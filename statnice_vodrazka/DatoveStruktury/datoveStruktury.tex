\documentclass[a4paper]{article}      % Comments after  % are ignored
\usepackage{amsmath,amssymb,amsfonts} % Typical maths resource packages
\usepackage[utf8]{inputenc}
\usepackage{czech}
\usepackage{graphicx}
\usepackage{latexsym}
\usepackage{listings}

\newtheorem{theorem}{Věta}[section]
\newtheorem{lemma}[theorem]{Lemma}
\newtheorem{proposition}[theorem]{Předpoklad}
\newtheorem{corollary}[theorem]{Důsledek}

\newenvironment{proof}[1][Důkaz]{\begin{trivlist}
\item[\hskip \labelsep {\bfseries #1}]}{\end{trivlist}}
\newenvironment{definition}[1][Definice]{\begin{trivlist}
\item[\hskip \labelsep {\bfseries #1}]}{\end{trivlist}}
\newenvironment{example}[1][Příklad]{\begin{trivlist}
\item[\hskip \labelsep {\bfseries #1}]}{\end{trivlist}}
\newenvironment{remark}[1][Pozorování]{\begin{trivlist}
\item[\hskip \labelsep {\bfseries #1}]}{\end{trivlist}}

\newcommand{\qed}{$\blacksquare$}

\begin{document}

\section*{Amortizovaná složitost}
Na rozdíl od složitosti v nejhorším případě se při amortizované složitosti započítává i "vhodnost" stavu
datové struktury pro provedení operace. Složitá operace zpravidla změní stav datové struktury tak, že k jejímu dalšímu použití
dojde až za delší dobu. Lze počítat více způsoby:

\begin{itemize}
\item \emph{Agregační metoda} - spočítáme nejhorší možný čas $T(n)$ posloupnosti $n$ operací. Amortizovaná složitost je potom:
\[
\frac{T(n)}{n}
\]
\item \emph{Účetní metoda} - za každou provedenou operaci je třeba zaplatit určitý obnos $c$ - \emph{amortizovaná cena}.
Jednoduché operace z tohoto obnosu spotřebují jen část (přebytek se uloží na účet),
složité mohou potřebovat víc (co chybí, to se bere z účtu). To co se doopravdy spotřebuje je \emph{skutečná cena} operace.
Pokud je stav na účtu vždy nezáporný, je $c$ amortizovaná složitost dané operace.
\end{itemize}

\begin{example}{Insert v dynamickém poli}
Pokud se pole zaplní, naalokujeme dvakrát tak veké a překopírujeme všechny prvky.
\begin{itemize}
\item \emph{amortizovaná cena} - určíme ji na $2k+1$ kde $k$ je nějaká konstanta.
\item \emph{skutečná cena} - stanovíme $k$ jako cenu za alokaci místa pro jeden prvek a jeho překopírování. Samotné vložení prvku do pole má cenu $1$.
\end{itemize}
Protože jsme to udělali chytře, tak nám vyjde, že nejhorší případ má skutečnou cenu $n*k$, kde $n$ je velikost plného pole před alokací.

Máme jistotu, že než k tomuto případu dojde, bude na účtu obnos $2n*k$. Protože nové pole alokujeme dvakrát větší, je to přesně tolik, kolik potřebujeme.

Amortizovaná složitost operace \verb+insert+ je tedy $O(2k+1)$, takže asymptoticky $O(1)$. Podmínka na nezáporný stav účtu je určitě splněna, protože každé "složité" operaci předchází dostatečně mnoho "jednoduchých", které účet stihnou naplnit. 
\end{example}

\section{Stromové vyhledávací struktury}
Binární stromy, haldy, trie a B-stromy

\subsection{Binární vyhledávací stromy}
Všechny vychází ze základního BVS. Používají se k reprezentaci podmnožiny $S \subseteq U$ nějakého lineárně uspořádaného univerza.
BVS umožňuje operace \verb+MEMBER+, \verb+INSERT+ a \verb+DELETE+, jejichž složitost závisí na výšce stromu.
Pro každý uzel $x$ BVS definuje:
\begin{itemize}
\item $key(x)$ - přiřazený prvek
\item $parent(x)$ - rodič uzlu
\item $left(x)$ - levý syn
\item $right(x)$ - pravý syn
\end{itemize}
Pro kořen $r$ platí $parent(r) = NIL$. Pro všechny listy $s$ platí $left(s)=NIL$ a $right(s)=NIL$.
Je-li $x$ vrchol a $left(x) = u, right(x)= v$, potom platí:
\[
key(u) < key(x) \leq key(v)
\]

\paragraph{Efektivní operace} - protože efektivita operací závisí na výšce stromu, chceme pracovat s vyváženými stromy, jejichž výška je logaritmická vzhledem k počtu uložených prvků.

\paragraph{AVL stromy}
Oproti BVS každý uzel obsahuje navíc vyvažovací informaci \verb+N.balance+ $\in \lbrace -1, 0, 1\rbrace$.
Význam je následující:
\begin{itemize}
\item $-1$ - levý podstrom je o 1 vyšší než pravý
\item $0$ - oba podstromy mají stejnou výšku
\item $1$ - pravý podstrom je o 1 vyšší než levý
\end{itemize}
Vyvažovací informace se používá při operacích \verb+INSERT+ a \verb+DELETE+ v rámci vyvažovacích procedur:
\begin{itemize}
\item jednoduchá rotace (levá/pravá)
\item dvojitá rotace (levá/pravá)
\end{itemize}
Operace \verb+INSERT+ vyvažuje jednou. Operace \verb+DELETE+ vyvažuje max. $log(|S|)$-krát. 
Logaritmická výška AVL stromu se dokazuje výpočtem $min(i)$ - \#vrcholů minimálního AVL stromu s výškou $i$ (složitě přes Fibonacciho čísla)
a $max(i)$ - \#vrcholů maximálního AVL stromu s výškou $i$ (jednoduše indukcí).

\paragraph{RB stromy}
Mají navíc jednobitovou vyvažovací informaci (červená/černá) v každém uzlu. Navíc musí každý RB strom splňovat následující invarianty:
\begin{itemize}
\item Délka \emph{černé cesty}(počet vrcholů na cestě označených černě) z kořene do libovolného listu je stejná.
\item Listy jsou označeny černě.
\item Červený vrchol je buď kořen, nebo syn černého otce.
\end{itemize}

Operace \verb+INSERT+ vloží vrchol jako červený a obarví jeho syny černě.
Následuje oprava obarvení směrem ke kořeni a poté jedno volání nějaké rotace.

Operace \verb+DELETE+ volá buď 2x jednoduchou rotaci, nebo 1x jednoduchou a 1x dvojitou rotaci.
Logaritmická výška RB stromu vzhledem k počtu uložených prvků je garantována díky invariantu černé cesty.

Ať je $b$ délka černé cesty. Potom pro počet vrcholů $n$ platí:
\[
\begin{array}{lrcl}
b=1 & 1 \leq & n & \leq 3\\
b=2 & 2^{2}-1 \leq & n & \leq 2^{4}-1\\
b=3 & 2^{3}-1 \leq & n & \leq 2^{6}-1\\
    &              &\ldots & \\
b   & 2^{b}-1 \leq & n & \leq 2^{2b}-1\\
\end{array}
\]

\subsection{Haldy}Stromová datová struktura s lokální podmínkou na vnitřní vrcholy. Je-li $v$ vnitřní vrchol a $n_1,\ldots,n_k$ jeho následníci, potom platí:
\[
\forall i: f(v) \leq f(n_i)
\]
Umožňuje reprezentovat množinu $S$ s uspořádáním daným funkcí $f:S \rightarrow \mathbb{R}$ a poskytuje efektivní operace:
\begin{itemize}
\item \verb+Min(H)+ - nalezení minimálního prvku z haldy
\item \verb+RemoveMin(H)+ - odebrání minimálního prvku z haldy
\item \verb+Insert(x,H)+ - vložení prvku $x$ do haldy $H$
\item \verb+Delete(x,H)+ - odebrání prvku $x$
\item \verb+Increase(x,a), Decrease(x,a)+ - zvětší/zmenší hodnotu prvku $x$ o $a$
\item \verb+Makeheap(S)+ - postaví haldu nad množinou $S$
\item \verb+Merge(H1,H2)+ - sloučí haldy $H1$ a $H2$ 
\end{itemize}

\paragraph{d-regulární haldy} mají následující vlastnosti:
\begin{itemize}
\item každý vnitřní uzel má $d$ následníků (i kořen), nebo je listem
\item pokud má uzel méně než $d$ následníků, jsou všichni následníci listy
\item pokud očíslujeme všechny uzly průchodem do šířky a uzel, který má méně než $d$ následníků má číslo $i$, potom všechny uzly s číslem menším než $i$ mají přesně $d$ následníků a uzly s větším číslem než $i$ jsou listy
\end{itemize}
Téměř všechny operace jsou založeny na procedurách \verb+Up(v)+ (složitost $O(log(|S|))$ - může stoupat přes všechny patra) a \verb+Down(v)+ (složitost $O(d*log(|S|))$ - musí hledat minimum z $d$ synů v každém patře cestou dolů), které "probublávají" prvky haldou buď nahoru do kořene, nebo dolů na dno haldy. Operace \verb+Makeheap(S)+ (nasyp všechno do pole a pro každý prvek počínaje kořenem udělej \verb+Down+) vyjde teoreticky lépe (v čase $O(n)$), ale pro malé haldy se spíš vyplatí opakování \verb+Insert(x)+ (to pak bude v čase $O(n*log\ n)$).

\paragraph{Leftist halda} je speciální případ 2-regulární haldy s dalšími vlastnostmi:
\begin{itemize}
\item umíme rozlišit levého a pravého syna libovolného vrcholu $v$
\item pro každý vrchol $v$ definujeme $npl(v)$ - délka pravé cesty (po kolika krocích dojdu z $v$ do listu, když půjdu vždy doprava)
\item když má vrchol jednoho syna, je to levý syn
\item když má vrchol dva syny (levý - $u$ a pravý - $v$) potom platí:
\[
npl(u) \geq npl(v)
\]
\end{itemize}
Vrchol $v$ který nemá žádného syna má $npl(v)=1$(!!). Visí na něm jakože dva listy, které nenesou žádnou hodnotu. V \verb+Merge+ se takové vrcholy považují za prázdnou haldu ($\emptyset$).
Většina operací je implementována na základě rekurzivní operace \verb+Merge(H1,H2)+ (složitost $O(log(|H1|+|H2|))$).
Následující jednoduchý fakt (dokazuje se indukcí podle velikosti podstromu):
\[
n \geq 2^{npl(root)}-1 
\]
lze použít k důkazu toho, že délka pravé cesty z kořene leftist haldy je omezena logaritmicky k počtu uložených prvků ($log(n+1) \geq npl(root)$). Z toho pak vyplývá složitost operace \verb+Merge+ od které se odvíjí složitosti ostatních operací. Operace \verb+Makeheap+ používá frontu, do které si iniciálně vloží jednoprvkové haldičky, které po dvou spojuje a hází na konec, dokud není ve frontě jen jedna halda.
Tímhle způsobem lze docílit složitosti $O(|S|)$ vzhledem k počtu prvků - složitost \verb+while+ cyklu,
ve kterém to všechno běží se počítá jako součet nekonečné řady.

\begin{definition}[Binomiální strom] je stromová struktura definovaná rekurzivně:
\begin{itemize}
\item $H_0$ je jednoprvkový binomiální strom
\item $H_i$ vznikne složením dvou disjunktních stromů $H_{i-1}$ (kořen jednoho z nich se stane kořenem nově vzniklého stromu)
\end{itemize}
Binomiální strom $H_i$ má následující vlastnosti:
\begin{itemize}
\item má $2^{i}$ vrcholů
\item jeho výška je $i$
\item jeho kořen má $i$ synů
\item podstromy jeho synů jsou binomiálními stromy $H_0,H_1,\ldots,H_{i-1}$  
\end{itemize}
\end{definition}

\paragraph{Binomiální halda} je uspořádaný seznam binomiálních stromů:
\[
H = [T_1,T_2,\ldots,T_k]
\]
V haldě nemohou být dva binomiální stromy stejného řádu ($T_i,T_j \in H\ \Rightarrow i\neq j $).
Všechny uzly v binomiálních stromech splňují lokální haldovou podmínku na uspořádání.
Normálně je většina operací založena na operaci \verb+Merge+ (zde se realizuje vyvažování - aby se zachovaly unikátní řády stromů v haldě).

\paragraph{Líná binomiální halda} je binomiální halda se zeslabenými předpoklady na unikátní řády stromů.
V seznamu stromů mohou být i stromy různých řádů - vyvažování se provádí pouze při operacích \verb+Min+ a \verb+Deletemin+.

\paragraph{Vyvažovací procedura} pracuje se seznamy stromů řádu $i$ (postupně pro všechna $i$). Prvky ze seznamu po dvou spojuje a vkládá do seznamu $i+1$. Pokud nějaký zbyde, tak ho hodí do haldy. Výsledkem je halda, která obsahuje stromy s unikátními řády.

\lstset{
	basicstyle=\small,
	stringstyle=\ttfamily,
	columns=fixed,
	basewidth=0.50em,
	breaklines=true,
	breakautoindent=true}
\begin{lstlisting}{frame=single}
vyvaz(L={O_1,O_2,..}) // O_i je seznam binomialnich stromu radu i (nektere seznamy mohou byt prazdne)
i=0
H={}
while L != {{},{},...} do    // dokud jsou v L neprazdne seznamy
  S=O_i              // aktualne zpracovavany seznam (stromy radu i)
  D=O_(i+1)          // seznam stromu s vyssim radem (i+1)
  while length(S) >= 2 do
    T1=head(S)       // prvni dva stromy ze seznamu (maji stejny rad i)
    T2=head(tail(S))
    S=tail(tail(S))  // zbytek bez prvnich dvou
    append(D,spoj(T1,T2)) // spoj stromy a zarad vysledek do seznamu pro vyssi rad
  done
  if length(S) > 0 then // zbyva posledni strom radu i (byl jich lichy pocet)
    append(H,head(S))   // vloz zbyly strom do vysledne haldy
	  O_i = {}  // vsechny stromy radu i jsou vyresene
  i=i+1
done 
\end{lstlisting}

\paragraph{Fibonacciho strom} je strom, ve kterém platí lokální haldová podmínka. Některé vrcholy (kromě kořene) mohou být označeny.
Označení znamená, že byl od vrcholu odtržen některý syn. Fibonacciho strom má řád $i$ pokud má jeho kořen $i$ synů.
Navíc každý Fibonacciho strom vznikl uvnitř nějaké Fibonacciho haldy, která byla na začátku prázdná posloupností haldových operací.

\paragraph{Fibonacciho halda} je tvořena seznamem Fibonacciho stromů.
Efektivita Fibonacciho haldy je založena na tom, že vznikla z prázdné haldy posloupností haldových operací.
Operace \verb+Merge, Insert, Min+ a \verb+Deletemin+ jsou stejné jako v binomiálních haldách s línou implementací.
Operace \verb+Increase+ a \verb+Decrease+ pracují s označením vrcholů. Amortizované složitosti:
\begin{itemize}
\item \verb+Insert+, \verb+Merge+ a \verb+Decrease+ - $O(1)$
\item \verb+Min+, \verb+Deletemin+, \verb+Delete+ a \verb+Increase+ - $O(log(n))$
\end{itemize}

\paragraph{Manipulace s hodnotami prvků} se provádí pouze v operacích \verb+INCREASE+ a \verb+DECREASE+.
Ve Fibonacciho haldách pracují s označením vrcholů a fungují zhruba takto:
\begin{itemize}
\item \verb+DECREASE+: když se sníží hodnota uzlu $x$, poruší se tím \emph{lokální haldová podmínka} vzhledem k rodiči $x$. Celý podstrom včetně $x$ se musí odříznout a vložit do haldy zvlášť (tím se zruší případné \emph{označení} vrcholu $x$, protože se stane kořenem) a vrchol $x$ se musí označit(jakože už přišel o jednoho syna). Pokud by se od $x$ odtrhával 2. syn, tak se potom musí odtrhnout i samotný vrchol $x$ (protože když někdo přijde o dva syny, tak to už je na čase aby se z něj stal kořen). To sice může vyvolat kaskádu odtrhávání, ale \emph{amortizovaně} to prý vychází dobře.
\item \verb+INCREASE+: když se zvýší hodnota uzlu $x$, tak se tím poruší haldová podmínka vzhledem k dětem $x$. Je potřeba všechny děti utrhat a vložit do haldy jako samostatné stromečky. Stejně tak samotný vrchol $x$. Odtržení $x$ ale zase může vyvolat kaskádu odtrhávání, protože rodič $x$ už může být označený.
\end{itemize}


\subsection{Trie}
Stromová datová struktura vhodná pro implementaci slovníku. 

\paragraph{Slovník}
Mějme abecedu $\Sigma$ velikosti $k$. Potom univerzum $U$ tvoří slova délky právě $l$
(nekonečná slova mít nemůžeme a kratší slova doplníme zprava mezerami). Chceme reprezentovat nějakou množinu slov $S \subseteq U$.

\paragraph{Základní varianta.}
\begin{itemize}
\item kořen odpovídá prázdnému slovu
\item pokud uzel odpovídá slovu \verb+wx+, potom jeho předek odpovídá slovu \verb+w+
\item pro každý list je definována funkce $N: L \rightarrow \lbrace 0,1 \rbrace$. Platí: $N(u)=1\ \Leftrightarrow$ uzel $u \in L$ odpovídá slovu $w \in S$.
\item pro každé slovo $w \in S$ existuje ve struktuře odpovídající list
\end{itemize}
Složitost operace \verb+MEMBER+ je $O(1)$, složitost operací \verb+INSERT+ a \verb+DELETE+ je $O(kl)$. Paměťová složitost v nejhorším případě je $O(|S|lk)$. Dobrá množina slov pro příklad je: \verb+THERE IS A TAVERN IN THE TOWN+

\subsection{B-stromy a jejich varianty}
Jde o speciální případ \textbf{(a,b)-stromů}. Je asi lepší, nejdřív popsat co je (a,b)-strom a potom popisovat modifikace
a vysvětlovat, k čemu jsou dobré.

\begin{definition}[(a,b)-strom]
pro $a \leq b$(kladná, přirozená) je kořenový strom $(T,t)$ pokud:
\begin{itemize}
	\item každý vnitřní vrchol $v \in T$ různý od kořene $t$ má minimálně $a$ a maximálně $b$ následníků
	(dále požadujeme: $2 \leq a,\ 2a-1 \leq b$ - to nám umožní reprezentovat množiny všech velikostí)
	\item kořen je buď list, nebo má minimálně $2$ a maximálně $b$ synů.
	\item všechny listy $l$ jsou vzhledem ke kořeni $t$ ve stejné hloubce
\end{itemize}
\end{definition}

\begin{remark}[Logaritmická výška] vzhledem k počtu listů $n$ (prvky reprezentované množiny $|S|=n$ jsou uloženy v listech!!):
\begin{itemize}
	\item $h=0$ - kořen je listem ($n=1$)
	\item $h=1$ pro $n$ platí $2 \leq n \leq b$
	\item $h=2$ pro $n$ platí $2a \leq n \leq b^{2}$
	\item $h=3$ pro $n$ platí $2a^{2} \leq n \leq b^{3}$
	\item atd.
\end{itemize}
Obecně platí: $2a^{h-1} \leq n \leq b^{h}$
\end{remark}

\paragraph{Struktura vrcholů v (a,b)-stromech.} Mějme vnitřní vrchol $v$. Potom $v=(S_{v},H_{v})$, kde:
\begin{itemize}
\item $S_{v}$ - je pole ukazatelů na syny velikosti $\rho_{v}$
\item $H_{v}$ - je uspořádané pole hodnot (velikosti $\rho_{v}-1$) takové, že:
$H_{v}[i]$ je největší hodnota z $S$ reprezentovaná v podstromu syna $S_{v}[i]$.
\end{itemize}
List $l$ obsahuje pouze klíč - $key(l)\in S$ a případně odkaz na data.

\paragraph{Algoritmy} operací pro (a,b)-stromy:
\begin{itemize}
	\item \textbf{Vyhledej} - sestupuje od kořene, v každém uzlu $v$ hledá index $i$ aby věděl do kterého $S_{v}[i]$
        má pokračovat. Ukazatel $S_{v}[i]$ ukazuje na podstrom reprezentující interval $[H_{v}[i-1],H_{v}[i])$.
	\item \textbf{Štěpení} - štěpení uzlu, který má víc než $b$ synů
	\item \textbf{Spojení} - spojení uzlů, z nichž jeden má $a$ synů a druhý méně než $a$
	\item \textbf{Přesun} - přesun prvku mezi dvěma uzly
\end{itemize}

\paragraph{Varianty B-stromů}
\begin{itemize}
\item \textbf{$B$-stromy} řádu $b$ jsou vlastně $(a,b)$-stromy s $a=\lceil \frac{b}{2}\rceil$
\item \textbf{$B^{*}$-stromy} - u vnitřních uzlů se odkládá štěpení (přehazováním k sousedovi) až do okamžiku,
kdy jsou dva sousední uzly plné ze $\frac{2}{3}$
\item \textbf{$B^{+}$-stromy} - mají listy propojené pomocí ukazatelů (dobré pro intervalové dotazy); využití v databázích a filesystémech
\item \textbf{redundantní} vs. \textbf{neredundantní} - podle toho, kde jsou odkazy na data (pouze v listech vs. ve vnitřních uzlech)
\end{itemize}

\section{Hašování}
Pomocí hašování se řeší problém\footnote{tzv. slovníkový problém}, jak reprezentovat podmnožinu $S$ nějakého univerza $U$ v tabulce velikosti $m$.
Přitom zpravidla platí, že $m$ je prvočíslo, $|S| < m $ a $|S| \ll |U|$.

Chceme množinu $S$ reprezentovat pomocí nějaké datové struktury, která podporuje efektivní operace:
\begin{itemize}
\item \textbf{MEMBER(x)} - platí $x \in S$ ?
\item \textbf{INSERT(x)} - pokud $x \notin S$, tak vloží $x$ do datové struktury. 
\item \textbf{DELETE(x)} - pokud $x \in S$, tak smaže $s$ z datové struktury.
\end{itemize}  

Nejjednodušší reprezentace $S$ by mohla být pomocí pole s přímým přístupem, které by mělo velikost univerza $|U|$. To je ale plýtvání pamětí, protože předpokládáme, že $|S| \ll |U|$. Na druhou stranu všechny požadované operace by mohly mít konstantní složitost.

Hašování je metoda, která se snaží zachovat konstatntní složitost operací a zároveň šetřit paměť.
Používá se při tom \emph{hašovací funkce}, pomocí které se zjišťuje pozice prvku v tabulce:
\[h: U \rightarrow \lbrace 0, 1, \ldots, m-1 \rbrace\]
Přitom může samozřejmě pro dva různé prvky $x,y \in U$ nastat situace kdy $h(x) = h(y)$. Tomu se říká kolize.
Hašovací algoritmy se jí snaží minimalizovat a pomocí nejrůznějších mechanismů vyřešit.

Základní předpoklady a požadavky pro hašování jsou:
\begin{enumerate}
\item Hašovací funkce $h$ je \textbf{snadno spočitatelná} (v čase $O(1)$, nezávisle na velikosti vstupu)
\item Hašovací funkce zajišťuje \textbf{rovnoměrné rozdělení $U$}
\[
\forall i,j \in \lbrace 0, \ldots, m-1\rbrace:\ -1 \leq |h^{(-1)}(i)| - |h^{(-1)}(j)| \leq 1
\]
Když zahašujeme pomocí $h$ celé univerzum $U$, tak počty prvků kolidující na libovolných dvou pozicích tabulky se budou lišit maximálně o 1.
\item Množina \textbf{$S$} ($|S|=n$) \textbf{je náhodně vybrána z univerza $U$} s pravděpodobností $\frac{1}{{|U|\choose n}}$.
\item Každý prvek z $U$ \textbf{má stejnou pravděpodobnost, že bude argumentem operace.} 
\end{enumerate}

\paragraph{Metoda separovaných řetězců} řeší problém kolizí tak, že místo jednoho prvku ukládá do tabulky odkaz na seznam prvků.
Seznam pak obsahuje všechny prvky, které na dané pozici tabulky kolidují.

Poznámky k metodám:
\begin{itemize}
\item \textbf{MEMBER} prohledává seznam na dané pozici (prvek může být v seznamu právě jednou, nebo vůbec)
\item \textbf{INSERT} (i \textbf{DELETE}) volá nejdřív metodu \textbf{MEMBER} a poté vloží nový prvek na konec seznamu (\textbf{DELETE} jenom smaže) 
\end{itemize}

\paragraph{Metoda uspořádaných řetězců} používá separované řetězce, které jsou navíc setříděné.
Lepší výsledky lze očekávat při \textbf{neúspěšném vyhledávání} - nemusí se procházet celý řetězec až do konce.
Při úspěšném vyhledávání je složitost přibližně stejná jako v případě separovaných řetězců bez uspořádání.

Nevýhoda metod využívajících separované řetězce spočívá v neefektivním využití paměti.
Když se všechny prvky nahašují na jednu pozici v tabulce, je to stejné, jako používat spojový seznam,
ale navíc ještě alokujeme paměť pro seznamy k ostatním pozicím v tabulce.

\paragraph{Hašování s přemísťováním} ukládá prvky do původní tabulky. Tabulka má navíc položky \emph{next} a \emph{prev},
které nám umožňují uložit spojový seznam přímo do hašovací tabulky.

\paragraph{Hašování se dvěma ukazateli} ukládá prvky do původní tabulky. V tabulce jsou navíc položky \emph{begin} a \emph{next}.
Pomocí \emph{begin} zjistíme, kde začíná řetězec prvků kolidujících na dané pozici a pomocí \emph{next} zjistíme pozici následujícího prvku.

\paragraph{Srůstající hašování} nezajišťuje oddělení řetězců. Tabulka má navíc pouze prvek \emph{next}.
Podle toho kam se vkládá prvek při operaci \emph{INSERT} rozlišujeme \textbf{E}ISCH a \textbf{L}ISCH
(\textbf{Early} Standart Coalescent Hashing a \textbf{Late} Standart Coalescent Hashing).
\begin{itemize}
\item EISCH vkládá \textbf{za první prvek}
\item LISCH vkládá \textbf{na konec}
\end{itemize}

Další varianty LICH, VICH a EICH používají přidané řádky tabulky, do kterých hašovací funkce nemůže nic zahašovat. Tyto přidané řádky
se přednostně využívají když se hledá volný řádek. Tento postup oddaluje srůstání řetězců. Jednotlivé varianty se opět liší tím,
kam se vkládá nový prvek:
\begin{itemize}
\item LICH $\rightarrow$ Late Insert CH: vkládání na konec řetězce
\item EICH $\rightarrow$ Early Insert CH: vkládání za první prvek
\item VICH $\rightarrow$ Variable Insert CH: za poslední prvek řetězce, který je v pomocné části - pokud takový existuje. Pokud neexistuje,tak za první prvek.
\end{itemize}

Metody, které využívají speciální položky v tabulce mají zvýšené nároky na paměť. Následující metody nepotřebují žádné speciální položky v tabulce. Kolize jsou řešeny pomocí dalších hašovacích funkcí.

\paragraph{Hašování s lineárním přidáváním} je metoda, která v případě kolize hledá nejbližší volný řádek. Když ho najde, tak tam prvek vloží.
Přitom je třeba kontrolovat, jestli už neprocházíme tabulku podruhé.

Nevýhodou je možnost vzniku shluků, které vedou ke zpomalení algoritmu.

\paragraph{Dvojité hašování} využívá dvě hašovací funkce - $h_1$ a $h_2$.

Při operaci \textbf{INSERT} se hledá nejmenší $i$ takové, že:
\[
(h_1(x) + i*h_2(x))\ mod\ m
\]
je index volného řádku.

\paragraph{Složitost hašování s řetězci} se počítá na základě předpokladů pro hašování (rychlá spočitatelnost $h$, rovnoměrné rozdělení univerza, náhodný výběr prvků, náhodný výběr hašované množiny). Přitom významnou roli hrají:
\begin{itemize}
\item faktor naplnění $\alpha = \frac{n}{m}$
\item očekávaná délka řetězců
\item očekávaný počet testů (při vyhledávání v úspěšném/neúspěšném případě)
\end{itemize}

\paragraph{Další otázky k hašování}
\begin{itemize}
\item hledání volného řádku lze zpravidla implementovat pomocí zásobníku
\item přeplnění lze řešit přehašováním celé tabulky. Přitom se rozhodujeme podle $\alpha$ a držíme tento parametr v mezích $<\frac{1}{4}, 1>$.
\item operace \verb+DELETE+ v metodách, které ho nepodporují: metoda \emph{falešného delete} - prvek neodstranit, pouze zapsat speciální hodnotu, že je tam volno (\verb+INSERT+ tam potom může vkládat)
\end{itemize}


\paragraph{Univerzální hašování} je metoda, která umožňuje obejít předpoklad rovnoměrného
rozložení dat pomocí náhodného výběru hašovací funkce. Hašovací funkce se přitom vybírá ze speciální množiny funkcí
(tzv. \emph{c-univerzální systém}), ve které je zaručeno, že se většina funkcí z této množiny chová "dobře".

\begin{definition}
Množina funkcí $H = \lbrace h_i | \forall i \in I: h_i: U \rightarrow \lbrace 0, \ldots, m-1 \rbrace \rbrace$ je \emph{c-univerzální systém funkcí}, pokud:
\[
\forall x,y \in U, x \neq y: |\lbrace i \in I\ |\ h_i(x) = h_i(y) \rbrace| \leq c*\frac{|I|}{m}
\]
\end{definition}

Existence c-univerzálních systémů:
\begin{enumerate}
\item předpoklad o univerzu: $U = \lbrace 0,1, \ldots, N-1\rbrace$, kde $N$ je prvočíslo
\item množina funkcí:
\[
h_{a,b}(x) = ((ax + b)\ mod\ N)\ mod\ m
\]
\item $a,b \in I = \lbrace 0, \ldots , N-1 \rbrace$, tedy celkem $N^{2}$ funkcí
\item zkoumáme kolik existuje dvojic $a,b$ takových, že pro nějaká $x,y \in U, x \neq y$:
\[
h_{a,b}(x) = h_{a,b}(y)
\]
\item soustava rovnic (regulární protože $x \neq y$):
\[
\begin{array}{lcl}
(ax+b)\ mod\ N & = & rm + i\\
(ay+b)\ mod\ N & = & sm + i\\
\end{array}
\]
Počet řešení soustavy s proměnnými $a$,$b$ a pevně danými $r,s,i$ představuje počet kolidujících funkcí pro danou dvojici $x$ a $y$.
\item Jsou-li $x$ a $y$ předem dány, zkoumáme počet různých kombinací pro $r,s$ a $i$.
Přitom $i \in \lbrace 0,1,\ldots,m-1 \rbrace$. Počet možností pro $r$ a $s$ je omezen výrazem $\lceil \frac{N}{m} \rceil$ (nejdříve modulo $N$ a potom modulo $m$). Počet kolidujících prvků tedy je tedy $m*\left( \lceil \frac{N}{m} \rceil \right)^{2}$.
\item Z nerovnosti:
\[
m*\left( \lceil \frac{N}{m} \rceil \right)^{2} \leq c*\frac{|I|}{m}
\]
stanovíme $c = \frac{\left( \lceil \frac{N}{m} \rceil \right)^{2}}{\left( \frac{N}{m} \right)^{2}}$
a ukážeme, že takové $c$ splňuje definici c-univerzálního systému.
\end{enumerate}

\paragraph{Perfektní hašování} je metoda, která řeší problém kolizí jejich úplnou eliminací - používá hašovací funkci, při jejímž použití nedochází ke kolizím. Pro daná vstupní data (množina $S$) se zkonstruuje perfektní hašovací funkce a ta se potom používá.
Přitom předpokládáme, že k operaci \textbf{INSERT} nebo \textbf{DELETE} téměř nedochází.

\begin{definition}
Mějme univerzum $U = \lbrace 0,1,\ldots, N-1 \rbrace$. Chceme hašovat množinu $S \subseteq U,\ |S|=n$ do hašovací tabulky velikosti $m$.
Soubor funkcí $H = \lbrace h | h: U \rightarrow \lbrace 0,1,\ldots, m-1 \rbrace \rbrace$ nazveme \textbf{$(N,m,n)$-perfektní}, pokud pro libovolnou $S \subseteq U,\ |S|=n$ existuje perfektní hašovací funkce $h \in H$.
\end{definition}

\paragraph{Odhad velikosti} $(N,m,n)$-perfektního systému $H$.

\paragraph{Důkaz existence} $(N,m,n)$-perfektního systému $H$.

\paragraph{Konstrukce} $(N,m,n)$-perfektního systému.
\begin{enumerate}
\item Použijeme funkci tvaru $h_{k}(x) = (k*x\ mod\ N) mod\ m$
\item Definujeme \emph{odchylku} od perfektnosti $b_{i}^{k} = |\lbrace x | x \in S: h_{k}(x) = i \rbrace|$.
Funkce $k$, která bude perfektní bude splňovat:
\[
\forall i \in 0,\ldots,m-1\ :\ b_{i}^{k} \leq 1
\]
\item Ukážeme, že pro perfektní hašovací funkci musí platit:
\[
\Sigma_{i=0}^{m-1} (b_{i}^{k})^{2} - n < 2
\]
\item Ukážeme (pomocí magie), že čísel $k$, které jsou perfektní je alespoň $\frac{N-1}{4}$, což nám umožní použít pravděpodobnostní algoritmus, který takové $k$ najde. (Očekávaný počet pokusů při hledání je 4)
\end{enumerate}

\section{Mapování datových struktur do stránek vnější paměti počítače}
Data jsou ve vnější paměti uložena ve stránkách. Předpokládáme, že na jednu stránku se vejde $b > 1$ záznamů. 
Při práci s vnější pamětí (HDD,DVD,...) načítáme data po stránkách. Protože práce s vnější pamětí představuje typicky časově náročnou operaci (v porovnání s operacemi ve vnitřní paměti), chceme minimalizovat počet I/O operací (\emph{read}/\emph{write}) na 1 stránku. 
Například při reprezentaci jednoduchých datových struktur (např. fronta nebo zásobník) lze použít \emph{buffer}.

\paragraph{Slovníkový problém na externí paměti:} chceme DS, která umožňuje operace \verb+INSERT+, \verb+DELETE+ a \verb+MEMBER+ při minimálním počtu I/O operací.

\paragraph{Schéma pro použití externího hašování.} Používáme pomocné struktury zvané \emph{adresář} a \emph{primární soubor}.
\begin{enumerate}
\item dostaneme dotaz a spočteme pomocí hašovací funkce pozici v \emph{adresáři} (ten se dá načíst do RAM)
\item na dané pozici z adresáře zjistíme číslo stránky \emph{primárního souboru}, kterou načteme z disku
\item v rámci stránky najdeme odkaz na fyzickou stránku, která obsahuje hledaná data
\end{enumerate}
Počet přístupů na disk při dotazu: 1x načtení stránky z primárního souboru + 1x načtení stránky se záznamem.
Taky je třeba načíst z disku adresář (to ale stačí 1x při inicializaci). 

\paragraph{Statické vs. dynamické hašování.}
Rozdíl je v tom, jestli máme fixovanou délku \emph{primárního souboru} nebo ne. \textbf{Statické hašování} používá \emph{primární soubor konstantní velikosti} a tímpádem nemá možnost adresovat víc stránek než dokáže primární soubor pojmout.
Naproti tomu \emph{dynamické hašování} nemá fixovaný počet stránek primárního souboru, takže se nemůže stát, že by nám stránky došly.

\subsection{Statické hašování}

\paragraph{Cormack} - \emph{perfektní} hašovací metoda.
Využívá \emph{dvě hašovací funkce} a \emph{adresář} s $d$ řádky.
Primární soubor má fixní počet řádků.
\begin{itemize}
\item primární hašovací funkce $h(k) \in \lbrace 0, \ldots, d-1 \rbrace$ vrací řádek adresáře pro daný klíč $k$
\item jeden řádek adresáře obsahuje trojici: $(p,i,r)$ - (adresa v primárním souboru, maximální použitá hašovací funkce, počet kolidujících záznamů)
\item sekundární hašovací funkce $h_{i}(k,r)$ pro dané $k,i,r$ jednoznačně určí relativní pozici prvku s klíčem $k$ v primárním souboru vzhledem k $p$
\end{itemize}
Při vkládání, nebo mazání prvku se musí přesunout celá kolidující skupina a najít nová hašovací funkce (perfektní v rámci skupiny).
Nová funkce se hledá pro danou množinu klíčů velikosti $r$ tak, že se mění $i$.

\paragraph{Larson-Kajla} - \emph{perfektní} hašovací metoda. Používá 2 hašovací funkce: $h(k,i) = (k + i) mod M$ ($M$ je počet stránek primárního souboru) a $s(k,i)=k^{i} mod b$ ($b$ je maximální počet záznamů ve stránce).
Nepotřebuje adresář, ale používá fixní počet stránek primárního souboru.

\subsection{Dynamické hašování}

\paragraph{Fagin} - rozšiřitelné hašování s 1 hašovací funkcí $h: U \rightarrow \lbrace 0,1 \rbrace^{*}$.

Metoda využívá \emph{adresář} který nese parametr $d$ - aktuální počet bitů rozlišovaných v rámci adresáře. Počet řádků v adresáři je $2^{d}$, což odpovídá počtu slov z množiny $\lbrace 0,1 \rbrace^{d}$.
Každý řádek odkazuje na nějakou stránku \emph{primárního souboru}.

Každá stránka $S$ v \emph{primárním souboru} nese parametr $p_{S}$ který udává počet bitů (délku prefixu), které jsou stejné pro všechny klíče
nahašované do $S$.

\paragraph{Litwin}

\section{Třídění ve vnitřní a vnější paměti}

\subsection{Vnitřní paměť}
\paragraph{Algoritmy založené na porovnávání}
\begin{itemize}
\item Quicksort - podle principu rozděl a panuj. Problém výběru tzv. \emph{pivota} (proměnná $a$).
	\lstset{
	basicstyle=\small,
	stringstyle=\ttfamily,
	columns=fixed,
	basewidth=0.50em,
	breaklines=true,
	breakautoindent=true,
	language=haskell}
\begin{lstlisting}{frame=single}
quick [] = []
quick (a:za) =  quick [m | m <- za, m < a]
                ++ [a]
                ++ quick [v | v <- za, v >= a]
\end{lstlisting}
V nejhorším případě složitost $O(n^2)$. Očekávaný případ $O(n*log\ n)$.
\item Mergesort - opět rozděl a panuj. Slévání krátkých posloupností do delších. Vstupní posloupnost $A$
	\begin{enumerate}
	\item vytvoř frontu $Q$ obsahující rostoucí podposloupnosti $A$
	\item dokud je ve frontě víc než jedna posloupnost, tak zavolej \verb+Merge(a,b)+ na první dvě a výsledek ulož do $Q$
	\item vrať poslední posloupnost co zbyla ve frontě $Q$
	\end{enumerate}
\textbf{Analýza:} procedura \verb+Merge+ se provede v každém kole cyklu tolikrát, že celková složitost je $O(n)$
(vždycky musí projít všechny prvky). Délky posloupností ve frontě: v prvním kole alespoň $1$, v druhém alespoň $2$ atd.
(v $i$-tém alespoň $2^{i-1}$) takže celkem bude $log_{2} n$ běhů cyklu. Dohromady tedy $O(n*log n)$
\item Heapsort - postavíme d-regulární haldu v čase $O(n)$ a $n$-krát provedeme operaci \verb+Deletemin+ v čase $O(log\ n)$
\end{itemize}

\paragraph{Ostatní}
\begin{itemize}
\item Radixsort
\end{itemize}

\subsection{Vnější paměť}
Pokud se nám data nevejdou do vnitřní paměti, tak si musíme ukládat mezivýsledky na disk. Lze použít \emph{mergesort} na souborech.
Předpokládejme, že chceme setřídit $900$ MB a máme k dispozici jen $100$ MB RAM.
\begin{enumerate}
\item Vytvořím setříděné kusy $C_1,\ldots,C_9$ velikosti $100$ MB (načtu, setřídím v paměti, uložím na disk)
\item Z každého kusu načtu do paměti $10$ MB a $10$ MB si nechám na buffer
\item Použiju 9-ti cestný mergesort:
	\begin{enumerate}
	\item ze začátku těch načtených kousků odebírám nejmenší prvek do bufferu
	\item když se zaplní buffer, zapíšu ho na disk a uvolním
	\item když se vyprázdní jeden z těch $10$ MB kousků, tak načtu z příslušného souboru pokračování
	\end{enumerate}
\item na konci budu mít na disku setříděný soubor
\end{enumerate} 
V prvním kroku můžu vytvořit delší než $100$ MB kusy, pokud použiju 2 $50$ MB haldy:
\begin{itemize}
\item načítám z původního $900$ MB souboru $P$ a stavím haldu $A$
\item když je $A$ plná, zapisuju do souboru $C_i$ vždy menší z prvků: vrchol $A$, počátek $P$ a větší použiju na stavbu druhé haldy $B$
\item když se vyprázdní jedna halda, načnu nový soubor $C_{i+1}$ a prohodím haldy $A$ a $B$  
\end{itemize}
U složitosti nás zajímá především počt přístupů na disk.
\end{document}

