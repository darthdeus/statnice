\documentclass[a4paper]{article}      % Comments after  % are ignored
\usepackage{amsmath,amssymb,amsfonts} % Typical maths resource packages
\usepackage[utf8]{inputenc}
\usepackage{czech}
\usepackage{graphicx}
\usepackage{latexsym}
\usepackage{listings}

\newtheorem{theorem}{Věta}[section]
\newtheorem{lemma}[theorem]{Lemma}
\newtheorem{proposition}[theorem]{Předpoklad}
\newtheorem{corollary}[theorem]{Důsledek}

\newenvironment{proof}[1][Důkaz]{\begin{trivlist}
\item[\hskip \labelsep {\bfseries #1}]}{\end{trivlist}}
\newenvironment{definition}[1][Definice]{\begin{trivlist}
\item[\hskip \labelsep {\bfseries #1}]}{\end{trivlist}}
\newenvironment{example}[1][Příklad]{\begin{trivlist}
\item[\hskip \labelsep {\bfseries #1}]}{\end{trivlist}}
\newenvironment{remark}[1][Pozorování]{\begin{trivlist}
\item[\hskip \labelsep {\bfseries #1}]}{\end{trivlist}}

\newcommand{\qed}{$\blacksquare$}

\begin{document}
\section{Architektura autonomního agenta}
Autonomní agent je cokoliv, co vnímá svoje prostředí pomocí senzorů a koná akce ovlivňující toto prostředí.
Agenta určuje jeho funkce - na základě posloupnosti vjemů rozhodne o nějaké akci.
\begin{itemize}
\item Inteligentní agent - autonomní jednání
\item Virtuální člověk - jako ve hře SIMS, vypadá a chová se "jako člověk"
\item Animat - umělá bytost ve virtuálním prostředí, je plausibilní (umožňuje provádět simulace)
\item Bot - simulace hráče v počítačové hře
\end{itemize}

\paragraph{Percepce.} Agent vnímá svoje prostředí skrze senzory. To může být kamera, nárazník, detektor reagující na světlo, nebo třeba
jen proměnná ve které si něco přečte. Skrze vjemy si agent může updatovat svůj vnitřní model světa.

\paragraph{Mechanismus výběru akcí.} Agentova funkce - nejjednodušší reprezentace je pomocí tabulky,
ta ale není v mnoha případech možná (příliš mnoho položek).

\paragraph{Paměť}

\paragraph{Psychologické inspirace}
ve vývoji autonomních agentů lze nalézt za pojmy jako například:
\begin{itemize}
\item Zpětnovazební učení (behaviorální psychologie)
\end{itemize}

% ---------------------------- %

\section{Metody řízení agentů}

\paragraph{Řídící architektury - Wooldridge}
\begin{enumerate}
\item \emph{Symbolická AI}. Agent jako knowledge-based symbolic system. Agent má k dispozici symbolickou reprezentaci světa.
Provádí logické operace, symbolickou manipulaci a pattern matching aby zjistil co má udělat. Hlavní problém je tvorba přesného popisu světa a realtime rozhodovací mechanismy - svět je příliš komplikovaný.

\item \emph{Reaktivní AI}. Agent s reaktivní architekturou nemá žádný komplikovaný vnitřní stav světa a nevyužívá symbolickou manipulaci. 
Jde o \emph{subsumční architekturu} - agent má zakódovánu sadu chování ve tvaru (podmínky,chování).
Tato sada je i hierarchicky uspořádaná (tedy něco jako POSH). Různá chování si mohou navzájem konkurovat.  

\item \emph{Hybridní AI}. Využití postupů z reaktivní i symbolické větve.
\end{enumerate}

\paragraph{Symbolické reaktivní plánování}

\paragraph{Konekcionistické reaktivní plánování.}

\paragraph{Hybridní přístupy}
Kombinace přístupu reaktivní AI a symbolické AI. Agent sestává z komponent.
\subparagraph{BDI}
Plánování není součástí architektury (záleží na programátorovi).
Agent může rozhodovat o tom, kolik času stráví \emph{plánováním} a kolik \emph{prováděním} plánů.
\begin{itemize}
\item \emph{Belief} - jaké má agent představy o světě (něco jako báze znalostí, včetně inferenčních pravidel)
\item \emph{Desire} - to co by agent rád dosáhl (může toho chtít víc a zvažovat pro a proti).
\item \emph{Intention} - to co se agent rozhodl udělat
\end{itemize}
Základní princip fungování BDI agenta:
\begin{enumerate}
\item initialize-state
\item repeat
	\begin{enumerate}
	\item options: option-generator(event-queue)
	\item selected-options: deliberate(options)
	\item update-intentions(selected-options)
	\item execute()
	\item get-new-external-events()
	\item drop-unsuccessful-attitudes()
	\item drop-impossible-attitudes()
	\end{enumerate}
\item end repeat
\end{enumerate}

\subparagraph{Soar}
State Operator And Result. Framework/architektura pro výstavbu kognitivních modelů.
V zásadě jde o produkční systém, který své konání zakládá na bázi znalostí.
\begin{itemize}
\item reprezentace stavového prostoru
\item reprezentace znalostí (odvozovací pravidla + univerzální fakta)
\item dočasná paměť pro objekty a jejich vlastnosti
\item mechanismus pro automatickou dekompozici cílů, učení a generování nových cílů
\end{itemize}
\verb+http://www.it.bton.ac.uk/staff/rng/teaching/notes/Soar.html+

Fáze v systému Soar:
\begin{enumerate}
\item uložení vstupů do pracovní paměti
\item aplikace if-then pravidel pro návrh vhodných akcí
\item výběr jedné akce na základě preferencí
\item na základě vybrané akce je vytvořen symbolický výstup pro zpracování v aktuátorech
\item aplikace akce (výstup z předchozí fáze se pošle na aktuátory)
\end{enumerate}

\paragraph{Srovnání s plánovacími technikami}
V BDI se nehledají detailní plány - každou chvíli se může něco změnit.

% ---------------------------- %

\section{Problém hledání cesty}
Pokud máme agenta, který se má pohybovat v nějakém prostředí, je třeba řešit problém
navigace. Různá řešení závisí na tom, jaké informace máme k dispozici:
\begin{itemize}
\item Známe mapu prostředí?
\item Známe svou pozici na mapě?
\item Známe pozici cíle/směr k cíli?
\end{itemize}

\paragraph{Navigační pravidla}
Pokud neznáme mapu prostředí, musíme se řídit podle jednoduchých navigačních pravidel.
Příkladem jsou \verb+Bug+ algoritmy, které předpokládají, že agent zná pouze směr k cíli a dovede v každém okamžiku zjistit,
jak je od cíle daleko.
\begin{itemize}
\item \verb+Bug1+ - označí si \emph{hitpoint} a \emph{leavepoint} pokud narazí na překážku. Musí ji objet celou.
\item \verb+Bug2+ - označí si \emph{hitpoint} a objíždí, dokud nenarazí na průsečík s přímkou k cíli, který je blíž než \emph{hitpoint}.
\end{itemize} 

\paragraph{Reprezentace terénu}
Vytvoření mapy navigačních bodů nebo oblastí. Poté se problém převede na hledání cesty v grafu. 
Přístupy:
\begin{itemize}
\item graf viditelnosti
\item Voronoi diagramy
\item lichoběžníková dekompozice
\end{itemize}

% ---------------------------- %

\section{Komunikace a znalosti v multiagentních systémech}
Jeden ze způsobů změny stavu světa v multiagentních systémech je i změna vnitřního stavu jiného agenta.
To se dá udělat \emph{přímo} (komunikace), nebo \emph{nepřímo} (změna stavu světa v okolí jiného agenta).

\paragraph{Komunikace} je proces, během kterého si agenti vyměňují informace formou \emph{řečových aktů} (elementární řečová zpráva).
Řečový akt je čtveřice (\textbf{odesilatel},\textbf{prijemce},\textbf{obsah},\textbf{typ}). Základní typy řečových aktů jsou:
\begin{itemize}
\item \emph{otázka}
\item \emph{nabídka}
\item \emph{zamítnutí}
\item \emph{informování}
\end{itemize}

\paragraph{Ontologie} je formální definice nějakého souboru znalostí. Typicky hierarchicky strukturovaná. Definuje nejenom \emph{třídy} objektů, ale i \emph{vlastnosti} těchto tříd a \emph{vztahy} mezi nimi.

Příkladem ontologie může být například hierarchie tříd nějakého programovacího jazyka.

Pro ontologie se používají nejrůznější jazyky - například XML.

\paragraph{Problém omezené racionality.} Každý agent, který se má racionálně rozhodovat v komplexním prostředí podléhá následujícím omezením:
\begin{enumerate}
\item v daném okamžiku disponuje pouze \emph{omezenou informací} o světě (např. vidí jenom na omezenou vzdálenost)
\item má pouze \emph{omezené prostředky} ke zpracování informací (např. RAM,CPU)
\item musí se rozhodovat v \emph{omezeném čase} (pokud jde o real-time agenta)
\end{enumerate}
V konečném důsledku je pro takového agenta nemožné se rozhodovat racionálně (tzn. volit vždy optimální akce).
Musí se tedy spolehnout na heuristiky a omezenou znalost prostředí.

\paragraph{Kripkeho sémantika možných světů}

% ---------------------------- %

\section{Etologické motivace}

\paragraph{Modely populační dynamiky}

\subparagraph{Analytické modely} představují nástroj matematické biologie ke zkoumání vývoje populací nejrůznějších živých organismů.
Sestavují se na základě \emph{empirických dat} (z pozorování). Analytický model je zpravidla tvořen diferenciálními rovnicemi, které popisují chování systému. Jejich nevýhodou je, že bývají příliš obecné.

Příklad: jednoduchá rovnice pro jednu populaci ($t$ - čas, $b$ - birth rate, $d$ - death rate):
\[
N(t+\Delta t) = N(t) + b*N(t)*\Delta t - d*N(t)*\Delta t
\]

Existují složitější modely, které lépe modelují konkrétní prostředí a pomáhají zodpovědět otázky jako např. \emph{Za jak dlouho dojde k vyhubení chráněného živočicha XY?}

Existují také modely pro chování více populací (model \emph{predátor-kořist}, \emph{hostitel-parazit} apod.).

\subparagraph{Simulační modely} jsou založeny na umělých agentech zvaných \emph{animati}, kteří jsou plausibilní - chovají se realisticky.
Hlavní nevýhodou simulačních modelů je vysoký počet parametrů.

% ---------------------------- %

\section{Metody pro učení agentů}
Předpokládejme agenta, který se má naučit nějakou funkci $f: Input \rightarrow Output$.
Učení agentů se využívá především tam, kde nelze pokrýt všechny možné eventuality tím, že je do agenta naprogramujeme předem.

Při strojovém učení potřebujeme vědět:
\begin{enumerate}
\item \emph{Jakou část agenta zlepšujeme?} (př. rozhodovací mechanismus pro výběr akcí)
\item \emph{Jakou odezvu má agent k dispozici?} (př. žádnou, otagovanou trénovací množinu)
\item \emph{V jaké podobě ukládáme naučená data?} (př. rozhodovací strom, pravděpodobnostní rozdělení, sada pravidel)
\end{enumerate}

Základní dělení metod strojového učení:
\begin{itemize}
\item \textbf{S učitelem} - máme: $(x_1,y_1),\ldots,(x_n,y_n)$ a chceme: $f(x_i)=y_i$ 
\item \textbf{Bez učitele} - máme: $\vec{x_1},\vec{x_2},\ldots,\vec{x_n}$ a chceme: $P(X=\vec{x})$ 
\item \textbf{Zpětnovazební učení} - máme: $s_0,a_1,s_1,a_2,\ldots$ kde některé stavy mohou být odměněny a chceme $\Pi(s_i)$ - nějakou sadu pravidel co dělat ve kterém stavu abychom posbírali co nejvíc odměn.
\end{itemize}

\paragraph{Učení s učitelem (supervised learning).} Je metoda, při které se používá následující schéma:
\begin{enumerate}
\item Vezmeme data (množina dvojic \textbf{(vstup,výstup)} - $\lbrace(x_1,y_1),(x_2,y_2),\ldots,(x_n,y_n)\rbrace$) a rozdělíme je na
dvě disjunktní množiny: \emph{trénovací} ($T$) a \emph{testovací} ($E$). Pro hledanou funkci $f$ platí: $\forall i:\ f(x_i)=y_i$
\item Použijeme nějaký \emph{učící algoritmus}, pro nalezení \emph{hypotézy} $h$, která aproximuje trénovací data.
O hypotéze řekneme, že je \emph{konzistentní} s množinou trénovacích dat $T$, pokud platí:
\[
\forall (x_i,y_i)\in T:\ h(x_i)=y_i
\]
\item kvalitu řešení (tj. nalezenou hypotézu $h$) otestujeme na množině testovacích dat $E$. Čím menší je suma $\Sigma_{(x_i,y_i)\in E}|h(x_i)-y_i|$ tím lepší hypotézu jsme našli.
\end{enumerate}
Pokud je obor hodnot hledané funkce diskrétní, jde o \emph{klasifikaci}. Je-li obor hodnot spojitý, jde o \emph{regresi}.

\paragraph{Učení bez učitele (unsupervised learning).} Je metoda, při které se agent snaží ze vstupních dat odvodit nějaké zákonitosti
(nemá při tom k dispozici správné výstupy a tímpádem ani zpětnou vazbu). To co se agent učí je pravděpodobnostní model vstupních dat, na základě kterého potom může klasifikovat nová data. Př. \emph{Bayesovské učení},\emph{EM algoritmus}

\paragraph{Zpětnovazební učení} je metoda, při které se agent učí, jak má volit akce, aby našel optimální strategii pro dané prostředí.
Jde o učení bez učitele. Agent sice dostává odezvu, ale přímo z prostředí, takže musí experimentovat, a zjišťovat, které stavy jsou \emph{dobré} a které \emph{špatné}. Něco jako, když hrajete hru, jejíž pravidla neznáte a po cca. 100 tazích Váš oponent řekne: "Prohráls saláte!".
Řeší se zde problém \emph{explorace} vs. \emph{exploatace}. Pokud agent zkouší nové akce, jejichž výsledek nezná, jde o \emph{exploraci}, pokud provádí akce, o kterých ví, že mu přinášejí užitek, jde o \emph{exploataci}. Problém je, že optimální strategie nemůže být ani čistě \emph{explorační} ani čistě \emph{exploatační} - hledá se vyvážený kompromis.

\begin{itemize}
\item \textbf{Pasivní učení} - agent má pevně danou strategii a učí se, jak jsou ohodnocené jednotlivé stavy, do kterých se dostane.
\item \textbf{Aktivní učení} - agent se učí jaké akce má volit a zároveň jak jsou ohodnocené.
\end{itemize}

\paragraph{Základní formy učení zvířat.}
\begin{itemize}
\item Habituace - úbytek reakce na opakování téhož podnětu (přitom je reakce na jiné podněty zachována - nejde tedy o únavu)
\item Senzitizace - nárůst reakce na určitý podnět (opak habituace)
\item Imprinting - proces učení vázaný k určité fázi vývoje jedince, který vede k trvalým a nezvratným změnám chování (kachny a Konrad Lorenz)
\item Klasické podmiňování - Pavlov a jeho psi. \emph{Nepodmíněný podnět} - sám vyvolává reakci, \emph{podmíněný podnět} - vyvolává reakci až když je spojen s nepodmíněným podnětem.
\item Operantní podmiňování - každá akce je tzv. \emph{operant}. Když dostane zvíře za provedení nějakého operantu odměnu, tak se upevní. Postupně se tak dá natrénovat komplikované chování složené z postupně podmíněných operantů.
\end{itemize}

% ---------------------------- %
\section{Umělá evoluce}
Hlavní myšlenka je využití myšlenky evoluce, která funguje v přírodě pro řešení problémů.
Často se používá tam, kde není známý žádný "rozumný" způsob řešení.

\paragraph{Genetické algoritmy}
Instance řešení problému se zakóduje do řetězce. Předem stanovený počet takových řetězců tvoří populaci.
Dále potřebujeme nějakou fitness funkci, pomocí které ohodnotíme řešení.

\paragraph{Jednoduchý GA:}
Základní varianta genetického algoritmu - lze přidat \emph{elitismus} a modifikovat jednotlivé kroky; tj. mechanismus výběru jedinců (\verb+Select+), nebo použité genetické operátory(\verb+Crossover+ a \verb+Mutate+).

\begin{lstlisting}{frame=single}
t := 0
Initialize G(0)		// inicializuj populaci 
do while not Done
      Evaluate G(t)		// vypocet fitness aktualni generace
      t := t + 1
      Select G(t) from G(t-1)	// vyber jedincu do dalsi populace
      Crossover G(t)		// krizeni
      Mutate G(t)		// mutace
loop

return best G(t)		// nejlepsi jedinec z posledni populace
\end{lstlisting}

Výhody GA:
\begin{itemize}
	\item nevyžadují žádné speciální znalosti o cílové funkci
	\item jsou odolné vůči sklouznutí do lokálního optima
	\item vykazují velmi dobré výsledky u problémů s rozsáhlými množinami přípustných řešení
	\item mohou být využity pro nejrozmanitější optimalizační problémy
\end{itemize}

Nevýhody GA:
\begin{itemize}
	\item mají problém s nalezením přesného optima
	\item vyžadují velké množství vyhodnocování cílové funkce
	\item jejich implementace není vždy přímočará 
\end{itemize}

\paragraph{Genetické a evoluční programování}
\begin{itemize}
\item Genetické programování - jedinec v populaci kóduje strukturu programu (strom)
\item Evoluční programování - struktura programu je dána. Evolučně se hledají číselné parametry programu.
\end{itemize}
% ---------------------------- %
\section{Základní přístupy a pojmy}
Předpokládejme, že řešení problému umíme zakódovat do nějakého řetězce.

\begin{itemize}
\item \emph{populace} - množina řetězců se zakódovaným řešením problému
\item \emph{fitness} - funkce, která každému jedinci z populace přiřadí číslo (kvalita řešení)
\item \emph{rekombinace} - uplatnění genetických operátorů: křížení, mutace
\item \emph{statická selekce} - ruleta, pořadová selekce, turnajová selekce
\item \emph{dynamická selekce} - mění se v průběhu algoritmu (něco na způsob Simulovaného žíhání)
\item \emph{ruletová selekce} - mechanismus výběru jedinců do další populace (rozložení na kole rulety podle fitness funkce)
\item \emph{turnaje} - mechanismus výběru jedinců (z náhodně vybrané dvojice postupuje lepší)
\item \emph{elitismus} - prevence proti odumírání nejlepších jedinců vlivem náhody
\end{itemize}

% ---------------------------- %

\section{Reprezentační schémata}
Předpokládejme, že máme binárně kódované řetězce délky $n$ jako jedince v populaci.

\begin{definition}[Schemata]
Schema je řetězec $s$ délky $m$ nad abecedou $\lbrace 1,0,* \rbrace$.\\
$o(s)$ je \emph{řád schematu} - počet pevných pozic (ne $*$) ve schematu\\
$d(s)$ je \emph{definiční délka schematu} - vzdálenost mezi první a poslední pevnou pozicí\\
$C(s,t)$ - četnost schematu $s$ v populaci $t$\\
$P_{s}(t)$ - množina jedinců pokrytých schematem $s$ v populaci $t$
$F(s)$ - fitness schématu (průměrná fitness všech pokrytých řetězců)
\end{definition}

\paragraph{Věta o schématech:} krátká nadprůměrná s malým řádem schémata v populaci GA se exponenciálně množí.\\
Ukáže se to postupnou analýzou toho co se se schématy děje během GA:
\begin{itemize}
\item \emph{selekce}: pravděpodobnost výběru jedince $x$ do nové populace:
\[
p^{sel}(x) = \frac{F(x)}{\Sigma_{u \in P(t)} F(u)}
\]
Pravděpodobnost vybrání schematu $s$ je potom: $p^{sel}(s)\frac{F(s)}{F(P(t))}$ (průměrná fitness schematu/fitnes celé populace)
\item \emph{křížení}: pravděpodobnost že schema bude zničeno křížením $p^{cro}(s)$. Pst. že se křížení trefí do schematu - $\frac{d(s)}{m-1}$, pst. že se vůbec bude křížit - $p_c$. Dohromady tedy:
\[
p^{cro}(s)=p_c*\frac{d(s)}{m-1}
\]
\item \emph{mutace}: pravděpodobnost, že schema nebude zničeno mutací $p^{mut}(s)$. Pst. že 1 bit \textbf{ne}bude zmutován: $1-p_m$ (kde $p_m$ je pst. mutace), pst. že nebude zmutován žádný bit schematu: $(1-p_m)^{o(s)} \sim 1 - p_{m}*o(s)$
(pravděpodobnost mutace totiž bývá hodně malá).
\end{itemize}
Dohromady se to všechno poskládá nějak takhle:
\[
C(s,t+1) = C(s,t)*p^{sel}(s)*(1-p^{cro}-p^{mut})
\]

\paragraph{Hypotéza o stavebních blocích} GA staví řešení z funkčních bloků reprezentovaných krátkými nadprůměrnými schématy.

% ---------------------------- %

\section{Pravděpodobnostní modely jednoduchého genetického algoritmu}
\begin{itemize}
\item \emph{konečná populace} - modelujeme pomocí Markovských řetězců
\item \emph{nekonečná populace} - ??
\end{itemize}

% ---------------------------- %

\section{Koevoluce, otevřená evoluce.} O \emph{otevřené evoluci} mluvíme, když fitness funkce nesměřuje k žádnému předem známému optimu.
Fitness funkce je definována jako relativní k evolvujícímu se systému, takže se s vývojem populace mění i nároky, které fitness funkce představuje.

\paragraph{Botstrap problem} - jak inicializovat výchozí populaci pro evoluční algoritmus?
Náhodně vygenerovaní jedinci jsou zpravidla všichni stejně "špatní".

\subsection{Koevoluce} je biologicky motivovaná samoorganizační metoda využívaná v evolučních algoritmech.
Příklady \emph{kompetitivních} modelů:
\begin{itemize}
\item \textbf{dravec vs. kořist}
\item \textbf{hostitel vs. parazit}
\end{itemize}
I při nezměněné fitness funkci je zajištěno zvyšování obtížnosti adaptace.

Může pomoci vyřešit \emph{bootstrap problem}.

Člověkem navržená fitness nehraje takovou roli - systém je do určité míry autonomní.

Nastavuje podmínky pro vznik evolučního soupeření mezi populacemi.

Kromě kompetitivních modelů je možné využít i kooperaci. Příklad s roboty co strkají objekty do určeného prostoru (definice fitness):
\begin{itemize}
\item individuální - každý robot dostane bod za malý objekt
\item kooperativní - jeden bod pro všechny za velký objekt (musí ho strkat minimálně 2)
\item altruistická - jeden bod pro robota za malý objekt, 1 bod pro všechny za velký objekt
\end{itemize}
Kooperativní evoluci je možné použít při řešení komplexních problémů, které lze rozdělit na podproblémy.
Evolucí se vyvíjí subpopulace pro každý podproblém a jednotlivé subpopulace spolu musí spolupracovat.
Dekompozice na subkomponenty je emergentní.

% ---------------------------- %

\section{Aplikace evolučních algoritmů}

\begin{itemize}
\item Genetické programování
\item Kreslení grafů
\item Koevoluce
\item Evoluce neuronových sítí (nastavení vah nebo i topologie)
\item Rozvrhování
\item Evoluční umění (hudba, obrazy, generování map do počítačových her)
\end{itemize}

\end{document}

