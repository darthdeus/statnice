\newpage
\section{Adaptivní agenti a evoluční algoritmy}
\begin{pozadavky}
\begin{pitemize}
\item Architektura autonomního agenta; percepce, mechanismus výběru akcí, paměť; psychologické inspirace.
\item Metody pro řízení agentů; řídící architektury podle Wooldridge, symbolické a konekcionistické reaktivní plánování, hybridní přístupy (Belief Desire Intention, Soar), srovnání s plánovacími technikami.
\item Problém hledání cesty; navigační pravidla, reprezentace terénu.
\item Komunikace a znalosti v multiagentních systémech, ontologie, problém omezené racionality, Kripkeho sémantika možných světů.
\item Etologické motivace, modely populační dynamiky.
\item Metody pro učení agentů; zpětnovazební učení, základní formy učení zvířat.
\item Umělá evoluce; genetické algoritmy, genetické a evoluční programování. 
\item Základní přístupy a pojmy: populace, fitness, rekombinace, genetické operátory; dynamická vs. statická selekce, mechanismus rulety, turnaje, elitismus.
\item Reprezentační schémata, hypotéza o stavebních blocích.
\item Pravděpodobnostní modely jednoduchého genetického algoritmu.
\item Koevoluce, otevřená evoluce.
\item Aplikace evolučních algoritmů (výběr akcí, evoluce expertních systému, konečných automatu, adaptace evolučních pravidel, neuroevoluce, řešení kombinatorických úloh).
\end{pitemize}
\end{pozadavky}
\subsection{Architektura autonomního agenta; percepce, mechanismus výběru akcí, paměť; psychologické inspirace}
\subsection{Metody pro řízení agentů; řídící architektury podle Wooldridge, symbolické a konekcionistické reaktivní plánování, hybridní přístupy (Belief Desire Intention, Soar), srovnání s plánovacími technikami}
\subsection{Problém hledání cesty; navigační pravidla, reprezentace terénu}
\subsection{Komunikace a znalosti v multiagentních systémech, ontologie, problém omezené racionality, Kripkeho sémantika možných světů}
\subsection{Etologické motivace, modely populační dynamiky}
\subsection{Metody pro učení agentů; zpětnovazební učení, základní formy učení zvířat}
\subsection{Umělá evoluce; genetické algoritmy, genetické a evoluční programování}
\subsection{Základní přístupy a pojmy: populace, fitness, rekombinace, genetické operátory; dynamická vs. statická selekce, mechanismus rulety, turnaje, elitismus}
\subsection{Reprezentační schémata, hypotéza o stavebních blocích}
\subsection{Pravděpodobnostní modely jednoduchého genetického algoritmu}
\subsection{Koevoluce, otevřená evoluce}
\subsection{Aplikace evolučních algoritmů (výběr akcí, evoluce expertních systému, konečných automatu, adaptace evolučních pravidel, neuroevoluce, řešení kombinatorických úloh)}