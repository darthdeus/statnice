\newpage
\section{Neuronové sítě}
\begin{pozadavky}
\begin{pitemize}
\item Neurofyziologické minimum: struktura neuronu, typy synapsí, hlavní části mozku.
\item Modely pro učení s učitelem: perceptron, algoritmus zpětného šíření, strategie pro urychlení učení, interní reprezentace znalostí, generalizace, regularizační techniky.
\item Asociativní paměti; Hebbovské učení, BAM, Hopfieldův model, energetická funkce a hledání suboptimálních řešení.
\item Stochastické modely; simulované žíhání, Boltzmannův stroj.
\item Klastrovací techniky a samoorganizace; k-means algoritmus, hierarchické shlukování, evoluční stromy.
\item Umělé neuronové sítě založené na principu učení bez učitele; Ojův algoritmus učení, laterální inhibice, Kohonenovy mapy a jejich varianty pro učení s učitelem, sítě typu ART.
\item Modulární, hierarchické a hybridní modely neuronových sítí; adaptivní směsi lokálních expertů, vícevrstvé Kohonenovy mapy, sítě se vstřícným šířením, RBF-sítě, kaskádová korelace.
\item Genetické algoritmy, věta o schématech.
\item Aplikace umělých neuronových sítí a evolučních technik (analýza dat, bioinformatika, zpracování obrazové informace, robotika a další).
\end{pitemize}
\end{pozadavky}
\subsection{Neurofyziologické minimum: struktura neuronu, typy synapsí, hlavní části mozku}
\subsection{Modely pro učení s učitelem: perceptron, algoritmus zpětného šíření, strategie pro urychlení učení, interní reprezentace znalostí, generalizace, regularizační techniky}
\subsection{Asociativní paměti; Hebbovské učení, BAM, Hopfieldův model, energetická funkce a hledání suboptimálních řešení}
\subsection{Stochastické modely; simulované žíhání, Boltzmannův stroj}
\subsection{Klastrovací techniky a samoorganizace; k-means algoritmus, hierarchické shlukování, evoluční stromy}
\subsection{Umělé neuronové sítě založené na principu učení bez učitele; Ojův algoritmus učení, laterální inhibice, Kohonenovy mapy a jejich varianty pro učení s učitelem, sítě typu ART}
\subsection{Modulární, hierarchické a hybridní modely neuronových sítí; adaptivní směsi lokálních expertů, vícevrstvé Kohonenovy mapy, sítě se vstřícným šířením, RBF-sítě, kaskádová korelace}
\subsection{Genetické algoritmy, věta o schématech}
\subsection{Aplikace umělých neuronových sítí a evolučních technik (analýza dat, bioinformatika, zpracování obrazové informace, robotika a další)}