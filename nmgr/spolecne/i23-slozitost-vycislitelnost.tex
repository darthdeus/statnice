\newpage
\section{Složitost}
\begin{pozadavky}
\begin{pitemize}
\item Metody tvorby algoritmů: rozděl a panuj, dynamické programování, hladový algoritmus.
\item Dolní odhady pro složitost třídění (rozhodovací stromy). 
\item Amortizovaná složitost. 
\item Úplné problémy pro třídu NP, Cook-Levinova věta.
\item Pseudopolynomiální algoritmy, silná NP-úplnost.
\item Aproximační algoritmy a schémata. 
\item Algoritmicky vyčíslitelné funkce, jejich vlastnosti, ekvivalence jejich různých matematických definic. 
\item Částečně rekurzivní funkce. 
\item Rekurzivní a rekurzivně spočetné množiny a jejich vlastnosti. 
\item Algoritmicky nerozhodnutelné problémy (halting problem). 
\item Věty o rekurzi a jejich aplikace: příklady, Riceova věta.
\end{pitemize}
\end{pozadavky}
\subsection{Metody tvorby algoritmů: rozděl a panuj, dynamické programování, hladový algoritmus}
\subsection{Dolní odhady pro uspořádání (rozhodovací stromy)}
\subsection{Amortizovaná složitost}
\subsection{Úplné problémy pro třídy NP, PSPACE, polynomiální hierarchie, pseudopolynomiální algoritmy}
\subsection{Aproximační algoritmy a schémata}
\subsection{Algoritmicky vyčíslitelné funkce, jejich vlastnosti, ekvivalence jejich různých matematických definic}
\subsection{Primitivně a částečně rekurzivní funkce}
\subsection{Rekurzivní a rekurzivně spočetné množiny a jejich vlastnosti}
\subsection{Algoritmicky nerozhodnutelné problémy}
\subsection{Věty o rekurzi a jejich aplikace}