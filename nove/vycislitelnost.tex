\documentclass{article}
\usepackage[utf8]{inputenc}
\usepackage{todonotes}
\usepackage{amsmath}
\usepackage{amsthm}
\usepackage{amssymb}

\newcommand{\inlinecode}{\texttt}

\title{Státnice - Vyčíslitelnost}
\author{Jakub Arnold}
\date{May 2019}

\theoremstyle{plain}
\newtheorem{thm}{Theorem}
\newtheorem{lemma}[thm]{Lemma}
\newtheorem{claim}[thm]{Claim}

\theoremstyle{plain}
\newtheorem{defn}{Definition}

\theoremstyle{remark}
\newtheorem*{cor}{Corollary}
\newtheorem*{rem}{Remark}
\newtheorem*{example}{Example}


\begin{document}

\maketitle

Problém \inlinecode{Helloworld}:
Instance: Kód programu $P$ a vstup $I$
Otazka: Je pravda, ze prvnich 12 znaku, ktere $P$ vypise, je "Hello, world"?

\section{Nerozhodnutelnost \inlinecode{Helloworld}}

Uvazme program $H$, ktery odpovida \emph{ano/ne} podle toho, jestli $P$ vypise \emph{Hello, world}. Predpokladame, ze vstup pro $P$ i $H$ je predavan na stdin, a vystup na stdout.

Upravime program $H$ na $H_1$ tak, ze misto \emph{ne} odpovida \emph{Hello, world}. Navic upravime $H_1$ na $H_2$ tak, aby prijmal na vstupu dohromady $P$ i $I$ jako jednu vec.

Potom se ptam na $H_2(H_2)$. Pokud $H_2$ vraci \emph{ano}, vypise \emph{Hello, world}, a pokud vraci \emph{Hello, world}, tak vraci \emph{ano} $\rightarrow$ spor. Z toho vyplyva, ze program $H_2$, $H_1$ a tedy ani $H$ nemuze existovat.

\todo{tady se vyuzije veta o rekurzi abysme mohli spojit vstupy}

\begin{defn}
    Jsme-li pomoci problemu $B$ schopni vyresit problem $A$, rikame, ze $A$ je \emph{prevoditelny} na $B$.
\end{defn}


% magic(x):
% 	return true pokud se x(x) zastavi, jinak false

% broken(x):
% 	if magic(x):
% 		cyklim se do nekonecna
% 	else:
% 		zastavim se


% broken(broken)

% - pokud se broken zastavi, tak se to cykli -> spor
% - pokd se broken cykli, tak se zastavi -> spor

\section{Turingovy stroje}

\begin{defn}
    \emph{Jednopaskovy deterministicky Turuinguv stroj} $M$ je petice
    $$M = (\mathcal{Q}, \Sigma, \delta, q_0, F),$$
    kde
    \begin{itemize}
        \item $\mathcal{Q}$ je konecna mnozina stavu
        \item $\Sigma$ je konecna paskova abeceda, ktera obsahuje znak $\lambda$ pro prazdne policko (casto budeme rozlisovat paskovou (vnitrni) a vstupni (vnejsi) abecedu)
        \item $\delta : \mathcal{Q} \times \Sigma \rightarrow \mathcal{Q} \times \Sigma \times \{ R, N, L \} \cup \{ \bot \}$ je \emph{prechodova funkce}, kde $\bot$ oznacuje nedefinovany prechod.
        \item $q_0 \in Q$ je \emph{pocatecni stav}.
        \item $F \subseteq Q$ je \emph{mnozina prijmacich stavu}.
    \end{itemize}
\end{defn}

\subsection{Konfigurace a displej turingova stroje}

Turinguv storoj sestava z \emph{ridici jednotky}, \emph{pasky} (potencialne nekonecne v obou smerech), a \emph{hlavy} pro cteni a zapis, ktera se pohybuje obema smery.

\emph{Displej} je dovjice $(q, a)$, kde $q \in \mathcal{Q}$ je aktualni stav a $a \in \Sigma$ je symbol pod hlavou. Na zaklade displeje TS rozhoduje, jaky dalsi krok ma vykonat.

\emph{Konfigurace} zachycuje stav vypoctu TS a sklada se ze ridici jednotky, slova na pasce a pozice hlavy na pasce.

\subsection{Vypocet Turingova stroje}

\emph{Vypocet} zahajuje TS $M$ v \emph{pocatecni konfiguraci}, tedy v pocatecnim stavu $q_0$ se vstupnim slovem zapsanym na pasce (nesmi obsahovat prazdne policko) a hlavou nad nejlevejsim symbolem vstupniho slova.

Pokud se $M$ nachazi ve stavu $q \in \mathcal{Q}$ a pod hlavou je symbol $a \in \Sigma$, pak:

\begin{itemize}
    \item Je-li $\delta(q, a) = \bot$, pak vypocet $M$ konci.
    \item Je-li $\delta(q, a) = (q', a', Z)$ kde $q' \in \mathcal{Q}, a' \in \Sigma$ a $Z \in \{ L, N, R\}$, pak $M$ prejde do stavu $q'$, zapise na pozici hlavy symbol $a'$, a posune hlavou dle $Z$ (doleva, zustane, doprava).
\end{itemize}

\subsection{Definice}

\begin{itemize}
    \item \emph{Slovo} nad abecedou $\Sigma$ je posloupnost znaku.
    \item \emph{Delku retezce} $w$ oznacujeme $|w| = k$ kde $k \in \mathbb{N}$.
    \item \emph{Prazdne slovo} oznacujeme pomoci $\epsilon$.
    \item \emph{Konkatenaci slov} $w_1, w_2$ piseme jako $w_1 w_2$.
    \item \emph{Jazyk} je mnozina slov nad abecedou $\Sigma$.
    \item \emph{Doplnek jazyka} $L$ oznacime pomoci $\bar{L} = \Sigma* \\ L$.
    \item \emph{Konkatenace dvou jazyku} je jazyk obsahujici konkatenace vsech jejich slov.
    \item \emph{Kleeneho uzaverem} jazyka $L$ je jazyk $L* = \{ w | (\exists k \in \mathbb{N}) (\exists w_1, \ldots, w_k \in L) w = w_1 w_1 \ldots w_k \}$.
    \item Rozhodovaci problem formalizujeme jako otazku, zda dana instance patri do jazyka kladnych instanci.
\end{itemize}

\subsection{Turingovsky rozhodnutelne jazyky}

\begin{itemize}
    \item TS $M$ \emph{prijma slovo} $w$, pokud vypocet $M$ nad vstupem $w$ skonci v prijmacim stavu.
    \item TS $M$ \emph{odmita slovo} $w$, pokud vypocet $M$ nad vstupem $w$ skonci ve stavu, ktery neni prijmaci.
    \item \emph{Jazyk slov prijmanych} TS $M$ znacime $L(M)$.
    \item Pokud TS $M$ nad vstupem $w$ konci, znacime $M(w)\downarrow$ (vypocet \emph{konverguje}).
    \item Pokud TS $M$ nad vstupen $w$ nekonci, znacime $M(w)\uparrow$ (vypocet \emph{diverguje}).
    \item Rekneme, ze jazyk $L$ je \emph{castecne rozhodnutelny} (tez \emph{rekurzivne spocetny}), pokud exsituje TS $M$, pro ktery $L = L(M)$. (tzn existuje TS, ktery jej prijma).
    \item Rekneme, ze jazyk $L$ je \emph{rozhodnutelny} (tez \emph{rekurzivni}), pokud existuje TS $M$, ktery se vzdy zastavi, a $L = L(M)$.
\end{itemize}

\subsection{Turingovsky vycislitelne funkce}

\begin{itemize}
    \item Turinguv stroj $M$ s paskovou abecedou $\Sigma$ pocita nejakou castecnou funkci $f_M : \Sigma* \rightarrow \Sigma*$.
    \item Pokud $M(w)\downarrow$ pro vstup $w \in \Sigma*$, je hodnota funkce $f_M(w)$ definovana, coz oznacime pomoci $f_M(w)\downarrow$, a jeji hodnota je slovo zapsane na vstupni pasce $M$ po ukonceni vypoctu nad $w$.
    \item Pokud $M(w)\uparrow$, pak je hondnota $f_M(w)$ nedefinovana, coz oznacime pomoci $f_M(w)\uparrow$.
    \item Funkce $f$ je \emph{turingovsky vycislitelna}, pokud existuje Turinguv stroj $M$, ktery ji pocita (kazda takova fce ma nekonecne mnoho TS, ktere ji pocitaji).
\end{itemize}

TS muze mit vice variant:

\begin{itemize}
    \item TS s jednosmerne nekonecnou paskou.
    \item TS s vice paskami (vstupni/vystupni/pracovni).
    \item TS s vice hlavami na paskach.
    \item TS s pouze binarni abecedou.
    \item Nedeterministicke TS.
\end{itemize}

Vsechny tyto varianty jsou ekvivalentni vyse definovanemu modelu v tom smyslu, ze prijmaji tu samou tridu jazyku, a vycisluji ty same funkce.

\end{document}
