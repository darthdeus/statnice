\section{Vektorové priestory}

\begin{poziadavky}
\begin{pitemize}
\item Základné vlastnosti vektorových priestorov, podpriestorov generovania, lineárna závislost a nezávislosť.
\item Veta o výmene
\item Konečne generované vektorové priestory, báza.
\item Lineárne zobrazenie.
\end{pitemize}
\end{poziadavky}

\noindent Ako zdroj pre vypracovanie otázky boli použité vlastné poznámky z prednášok Lineárna algebra Jiřího Fialu a suborkové texty.

\subsection{Definície}
\begin{definicia}
Nech $(T,+,\cdot)$ je teleso a $V$ je množina (jej prvky nazývame \emph{vektory}) s binárnou operáciou $+$ a $\cdot :T \times V\to V$ je zobrazenie, potom $(V,+,\cdot)$ sa nazýva \textbf{vektorový prostor} nad telesom $T$ ak je splnených nasledujúcich 8 axiomov.
\begin{penumerate}
\item[(SA)] $\forall u,v,w \in V: \quad (u+v)+w = u+(v+w)$ \hfill\textit{(asociativita súčtu)}
\item[(SK)] $\forall u,v \in V: \quad u+v = v+u$ \hfill\textit{(komutativita súčtu)}
\item[(S0)] $\exists {\bf 0}\in V: \quad u + {\bf 0} = {\bf 0} + u = u$ \hfill\textit{(neutrálný prvok súčtu)}
\item[(SI)] $\forall u \in V \, \exists -u \in V: \quad u + (-u) = {\bf 0}$ \hfill\textit{(inverzný prvok súčtu)}
\item[(NA)] $\forall a,b \in T \, \forall u \in V: \quad (a\cdot b) \cdot u = a \cdot (b \cdot u)$ \hfill\textit{(asociativita súčinu)}
\item[(N1)] $\forall u \in V: \quad 1 \cdot u = u$ \hspace{0.5cm} kde $1 \in T$ je jednotkový prvok telesa T
\item[(D1)] $\forall a,b \in T \, \forall u \in V: \quad (a+b) \cdot u = a \cdot u + b \cdot u$ \hfill\textit{(distributivita)}
\item[(D2)] $\forall a \in T \, \forall u,v \in V: \quad a \cdot (u+v) = a \cdot u + a \cdot v$ \hfill\textit{(distributivita)}
\end{penumerate}
\end{definicia}

\pagebreak[2]
\begin{prikladySK}
\begin{pitemize}
\item $\{{\bf 0}\}$ \dots  triviálny vektorový priestor
\item $T^{n}$ aritmetický vektorový priestor dimenzie $n$ nad telesom $T$. Ide o usporiadané $n$-tice, kde + je definované predpisom
$$(x_{1}, \dots, x_{n}) + (y_{1}, \dots, y_{n}) = (x_{1} + y_{1}, \dots ,x_{n} + y_{n})$$
a násobenie predpisom
$$\alpha(x_1, \dots, x_n)=(\alpha x_1, \dots, \alpha x_n)$$
\item Z každého telesa $T$ je možné vybudovať vektorový priestor rovnakej veľkosti $V=T^{1}$
\item $\mathbb{R}$, $\mathbb{Q}$, $\mathbb{C}$, $\mathbb{Z}_{p}$, \dots, $\mathbb{R}^{2}$, $\mathbb{Q}^{2}$, \dots 
\item Matice typu $m \times n$ nad $T$ (pre konkrétne $m$, $n$)
\item Polynomy nad $T$ (napríklad obmedzeného stupňa)
\end{pitemize}
\end{prikladySK}

\subsection{Vlastnosti vektorových priestorov}

\begin{pozorovanie}
\begin{penumerate}
\item ${\bf 0}$, $-u$ sú určené jednoznačne.
\item $\forall a \in T \ \forall u \in V: \quad a \cdot {\bf 0} = 0 \cdot u = {\bf 0}$
\item $\forall a \in T \ \forall u \in V: \quad a \cdot u = {\bf 0} \ \Rightarrow\ a = 0 \ \lor\ u = {\bf 0}$.
\end{penumerate}
\end{pozorovanie}

\begin{definicia}
Nech $(V,+, \cdot )$ je vektorový priestor nad telesom T a $U \subseteq V, U \not= \emptyset$ taká, že
\begin{pitemize}
\item $\forall u,v \in U: \quad u+v \in U$ \hfill\textit{(uzavretosť na súčet)}
\item $\forall u \in U \, \forall a \in T: \quad a \cdot u \in U$ \hfill\textit{(uzavretosť na súčin)}
\end{pitemize}
potom $(U,+, \cdot )$ nazývame \textbf{podpriestorom} $V$.
\end{definicia}

\begin{pozorovanie}
Podpriestor je tiež vektorový priestor.
\end{pozorovanie}

\begin{vetaSK}
Prienik ľubovolného systému podpriestorov je podpriestor.
\end{vetaSK}

\begin{definiciaN}{Lineárny obal, množina generátorov}
Nech $V$ je vektorový priestor nad telesom T a $X$ je podmnožina $V$, potom 
$$\mathcal{L}(X) = \bigcap\{U|X \subseteq U, U \,\mathrm{je\ podpriestor}\, V\}$$
je podpriestor $V$ generovaný $X$ nazývaný \textbf{lineárny obal $X$}. Množina $X$ sa potom nazýva \emph{systém generátorov} podpriestoru $\mathcal{L}(X)$. 

Keď $\mathcal{L}(X)=V$, potom $X$ je systém generátorov vektorového priestoru $V$.
\end{definiciaN}

\begin{definice}
\textbf{Spojení dvou podprostorů} je podprostor
$$W_1 \oplus W_2 = \mathcal{L}(W_1 \cup W_2)$$
\end{definice}

\begin{vetaSK}
Lineárny obal $\mathcal{L}(X)$ obsahuje všetky lineárne kombinácie vektorov z~X.
$$\mathcal{L}(X) = \{w\big|w = \sum_{i=1}^{n}{a_{i}u_{i}}, n \geq 0, n \,\mathrm{\textit{konečné}}, \forall i: \, a_{i} \in T, u_{i} \in X\}$$
Špeciálne v prípade, že $X=\emptyset$ a teda $n=0$, platí $\mathcal{L}(X)=\{0\}$.
\end{vetaSK}

\begin{definicia}
Nech $V$ je vektorový priestor nad telesom T, potom $n$-tica vektorov $v_{1}, \dots ,v_{n} \in V$ je \textbf{lineárne nezávislá}, ak rovnica
$$a_{1}v_{1} + a_{2}v_{2} + \dots + a_{n}v_{n} = {\bf 0}$$
má iba triviálne riešenie $a_{i} = 0$ pre všetky $i \in \{1, 2, \dots, n\}$

Nekonečná množina vektorov je lineárne nezávislá, ak každá jej konečná podmnožina je lineárne nezávislá.
\end{definicia}

\begin{pozorovanie}
$X$ je lineárne nezávislá práve keď $\forall u \in X: \, u \not\in \mathcal{L}(X \setminus \{u\})$
\end{pozorovanie}

\pagebreak[3]
\begin{veta}
\begin{penumerate}
\item Obsahuje-li systém $x_1, \dots, x_n$ nulový vektor, je závislý.
\item Obsahuje-li systém $x_1, \dots, x_n$ dva stejné vektory, je závislý.
\item Pro libovolná reálná čísla $\beta_2, \dots, \beta_n$ je systém $x_1, \dots, x_n$ lineárně závislý, právě když je systém $x_1+\sum_{i=2}^n \beta_i x_i, x_2, \dots, x_n$ lineárně závislý. \textit{(Inak povedané, ak pričítame k jednému vektoru ľubovolnú lineárnu kombináciu ostatných vektorov, nezmeníme tým ich lineárnu závislosť.)}
\end{penumerate}
\end{veta}

\begin{veta}
\begin{penumerate}
	\item Podsystém lineárně nezávislého systému je lineárně nezávislý.
	\item Nadsystém systému generátorů je systém generátorů.
\end{penumerate}
\end{veta}

\begin{definicia}
Nech $V$ je vektorový priestor. Množina $X \subseteq V$ sa nazýva \textbf{báza} vektorového priestoru $V$ ak
\begin{pitemize}
	\item je lineárne nezávislá
	\item $\mathcal{L}(X) = V$
\end{pitemize}
\textit{(Inak povedané, báza je lineárne nezávislý systém generátorov.)}
\end{definicia}

\begin{vetaSK}
Každý prvok vektorového priestoru možem vyjadriť ako lineárnu kombináciu prvkov jeho báze a toto vyjadrenie je jednoznačné.
\end{vetaSK}

\begin{definicia}
Vyjadrenie vektoru $u \in V$ vzhladom k báze $X$ sa nazýva \textbf{vektor súradníc}. Značí sa $[u]_{X}$.

$(x_{1}, \dots , x_{n}) = X$ je báza $V$, $u \in V: u = a_{1}x_{1} + a_{2}x_{2} + \dots +a_{n}x_{n}$

$[u]_{X} = (a_{1}, \dots , a_{n})$
\end{definicia}

\subsection{Veta o výmene}
\begin{lemmaN}{o výmene}
Nech $v_{1},v_{2}, \dots ,v_{n}$ je systém generátorov priestoru $V$ a pre $u \in V$ platí $u = a_{1}v_{1} + a_{2}v_{2} + \dots +a_{n}v_{n}$, potom platí 
$$\forall i: \, a_{i} \not= 0 \Rightarrow \mathcal{L}(v_{1},v_{2}, \dots, v_{i-1}, u, v_{i+1}, \dots, v_{n}) = V$$
\textit{(Inak povedané, vektor bázy ktorý sa ``podielal'' na vytvorení vektoru $u$, možme s $u$ zameniť.)}
\end{lemmaN}

\begin{dokaz}
Pre ľubovolné $w \in V$, môžme písať $w = b_{1}v_{1} + b_{2}v_{2} + \dots b_{n}v_{n}$.
Do tohoto vyjadrenia miesto $v_{i}$ dosadíme vyjadrenie $v_i$ z rovnice $u = a_{1}v_{1} + a_{2}v_{2} + \dots + a_{n}v_{n}$, čím dostaneme vyjadrenie ľubovolného $w$ pomocou $v_{1},v_{2}, \dots, v_{i-1}, u, v_{i+1}, \dots, v_{n}$, z čoho výplýva, že tieto vektory sú tiež systém generátorov.
\end{dokaz}

\pagebreak[3]
\begin{vetaSKN}{Steinitzova o výmene}
Nech $V$ je vektorový priestor, $X \subseteq V$ je lineárne nezávislá a $Y \subseteq V$je konečný systém generátorov. Potom existuje $Z \subseteq V$ také, že
\begin{pitemize}
	\item $|Z| = |Y|$
	\item $\mathcal{L}(Z) = V$
	\item $Z \setminus X \subseteq Y$
	\item $X \subseteq Z$
\end{pitemize}
\textit{(V krátkosti povedané - každý nezávislý systém vektorov $X$ je možné, pridaním vektorov zo systému generátorov $Y$, rozšíriť na systém generátorov $V$.)}
\end{vetaSKN}

\begin{dokaz}
Ak $X \subseteq Y$ sme hotoví a $Z=Y$.
Inak vezmeme $Y$ a postupne do neho začneme pridávať prvky z $X \setminus Y$.
Pri každom pridaní, podľa lemmy o výmene, jeden prvok z tejto množiny odstránime. Po poslednej iterácii získame hľadané $Z$.

(Pri každej iterácií vyhadzujeme jeden prvok, ktorý nepatrí do $X$, pretože $X$ je lineárne nezávislá.)
\end{dokaz}

\begin{dosledok}
Ak má $V$ konečnú bázu, majú všetky bázy rovnakú veľkosť.
\end{dosledok}

\begin{dosledok}
Ak má $V$ konečnú bázu, potom môžme každú lineárne nezávislú množinu $X$ doplniť na bázu.
\end{dosledok}

\begin{definicia}
Veľkosť bázy konečne generovaného priestoru $V$ sa nazýva \textbf{dimenzia} priestoru $V$. Značíme $\dim(V)$.
\end{definicia}

\begin{veta}
Buďte $W_1, W_2$ konečně generované podprostory vektorového
prostoru $V.$ Potom
$$\dim W_1+\dim W_2 = \dim(W_1 \cap W_2)+\dim(W_1 \oplus W_2)$$
\end{veta}

\subsection{Lineárne zobrazenie}

\begin{definiceN}{lineární zobrazení}
Mějme vektorové prostory $V,W$. Řekneme, že zobrazení $f: V \to W$ je \textbf{lineární}, jestliže pro libovolná $x,y \in V$ a $a, b \in T$ platí
$$f(a \cdot x + b \cdot y) = a \cdot f(x) + b \cdot f(y)$$
\end{definiceN}

\begin{definiceN}{lineární operátor}
Lineární zobrazení $f: V \rightarrow V$ se nazývá \emph{lineární operátor}.
\end{definiceN}

\begin{priklady}
\begin{penumerate}
    \item Identické zobrazení $V$ na $V$ (to je příklad lineárního operátoru).
    \item Buď $\alpha$ pevné reálné číslo. Zobrazení $V \to V$ dané předpisem $x \to \alpha x.$
    \item Derivace je lineární zobrazení z množiny reálných spojitých funkcí $C_1(J)$ do množiny reálných funkcí $F(J).$
    \item V $\mathbb{R}^2$ jsou lineární zobrazení např. zrcadlení $(x,y)\mapsto(-x,y)$ nebo zkosení $(x,y)\mapsto(x+y,y)$.
\end{penumerate}
\end{priklady}

\begin{definiceN}{Hodnost lineárního zobrazení}
Pro lineární zobrazení $f:U\to V$ mezi dvěma vekt. prostory definujeme \emph{jádro zobrazení} ($\mathrm{Ker\ }f$) jako množinu $\mathrm{Ker\ }f=f^{-1}[\{0\}]$. \emph{Obraz} zobrazení $f$ ($\mathrm{Im\ }f$) je množina $\mathrm{Im\ }f=f[U]$. Jako \emph{hodnost zobrazení} $f$ označíme číslo $\mathrm{dim}(\mathrm{Ker\ }f)$.
\end{definiceN}


\begin{vetaN}{Základní vlastnosti lineárního zobrazení}
Nechť $f: V \rightarrow W$ je lineární zobrazení. Potom platí:
\begin{penumerate}
	\item $f(\mathbf{0}_V) = \mathbf{0}_W$
	\item $\mathrm{Im\ }f$ je podprostor prostoru W
	\item $\mathrm{Ker\ }(f)$ je podprostor prostoru V
	\item $f$ je prosté, právě když $\mathrm{Ker\ }(f) = \{0\}$
	\item je-li $\dim V = \dim W$ a je-li zobrazení $f$ prosté, potom je $f$ bijekce a inversní zobrazení $f^{-1}: W \rightarrow V$ je opět lineární.
\end{penumerate}
\end{vetaN}

\begin{vetaN}{O dimenzi obrazu a jádra}
Pro $f:U\to V$ mezi dvěma vektorovými prostory konečné dimenze platí:
$$\mathrm{dim}(\mathrm{Ker\ }f)+\mathrm{dim}(\mathrm{Im\ }f)=\mathrm{dim}(U)$$
\end{vetaN}

\begin{vetaN}{Báze určuje lineární zobrazení}
Mějme dány vektorové prostory $V,W$ a bázi $B=b_1,\dots,b_n$ prostoru $V$. Potom pro každé lineární zobrazení $f:B\to W$ existuje právě jedno lineární zobrazení $g:V\to W$ takové, že $f(b_i)=g(b_i)\ \forall i\in\{1,\dots,n\}$.

\emph{Jiná formulace:} Pro libovolné vektory $y_1,\dots,y_n\in W$ existuje právě jedno lineární zobrazení $g:V\to W$ takové, že $g(b_i)=y_i\ \forall i\in\{1,\dots,n\}$.
\end{vetaN}


\begin{definiceN}{Isomorfismus}
Lineární zobrazení se nazývá \emph{isomorfismus}, existuje-li k němu inversní lineární zobrazení. Pokud existuje isomorfismus $V \to W$, říkáme, že prostory $V$ a $W$ jsou \emph{isomorfní}.
\end{definiceN}

\begin{veta}
Je-li lineární zobrazení bijektivní, je to isomorfismus.
\end{veta}

\begin{vetaN}{Isomorfismus vekt. prostorů nad $\mathbb{T}$}
Každý $n$-dimensionální vektorový prostor nad tělesem $\mathbb{T}$ je isomorfní vekt. prostoru $\mathbb{T}^n$ (tj. jehož prvky jsou uspořádané $n$-tice prvků z $\mathbb{T}$).
\end{vetaN}

\begin{vetaN}{Další vlastnosti lin. zobrazení}
\begin{penumerate}
    \item Je-li lineární zobrazení prosté, zachovává lineární nezávislost.
    \item Je-li na (surjekce), zachovává vlastnost \uv{být systémem generátorů}.
\end{penumerate}
\end{vetaN}

\begin{vetaN}{Skládání lineárních zobrazeni}
Nechť $f: U \rightarrow V$, $g: V \rightarrow W$ jsou lineární zobrazení. Potom složené zobrazení $g \circ f: U \rightarrow W$ definované předpisem
$$ (g \circ f)(x) = g(f(x)) \,\, \textit{pro } x \in U$$
je rovněž lineárním zobrazením.
\end{vetaN}


\begin{vetaN}{Sčítání a násobky lin. zobrazení}
Nechť $f, g$ jsou lineární zobrazení z vekt. prostoru $V$ do $W$, $\alpha$ skalár. Potom zobrazení $f+g: V \rightarrow W$ a $\alpha f: V \rightarrow W$ definovaná předpisem
$$(f+g)(x) = f(x) + g(x), \, x \in V$$
$$(\alpha f)(x) = \alpha f(x), \, x \in V$$
jsou lineární zobrazení $V$ do $W$.
\end{vetaN}

\begin{vetaN}{Množina lineárních zobrazení je vekt. prostor}
Množina lineárních zobrazení prostoru $V$ do prostoru $W$ s operacemi sčítání a násobení skalárem, definovanými v předchozí větě, tvoří vektorový prostor, který značíme $L(V,W)$.
\end{vetaN}

\begin{veta}
Nechť $\dim V=n$ a $\dim W=m$. Potom prostor $L(V,W)$ je isomorfní prostoru $\mathbb{R}^{m\times n}$. V důsledku toho je
$$\dim L(V,W) = mn$$
\end{veta}


