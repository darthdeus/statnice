\clearpage \documentclass[a4paper,12pt,notitlepage]{article}

\frenchspacing
\usepackage{a4}
\usepackage[pdftitle={Vypracovane otazky k bakalarskym statnicim}, pdfauthor={študenti MFF}, pdfdisplaydoctitle=true, colorlinks=false,unicode=true,pdfborder=0 0 0]{hyperref}
\usepackage{slovak}
\usepackage{ucs}
\usepackage[utf8x]{inputenc}

\title{Vypracovane otazky k bakalarskym statnicim}
\author{študenti MFF}

\usepackage{graphicx}
\usepackage{amsmath,amssymb,amsthm}
\usepackage{color}
\usepackage[left=3cm, right=3cm, top=3cm, bottom=3cm]{geometry} % nastavení dané velikosti okrajů


%Vacsina prostredi je dvojjazicne. V pripade, ze znenie napr pozorovania je pisane po slovensky, malo by byt po slovensky aj oznacenie.

\newenvironment{pozadavky}{\pagebreak[2]\noindent\textbf{Požadavky}\par\noindent\leftskip 10pt}{\par\bigskip}
\newenvironment{poziadavky}{\pagebreak[2]\noindent\textbf{Požiadavky}\par\noindent\leftskip 10pt}{\par\bigskip}

\newenvironment{definice}{\pagebreak[2]\noindent\textbf{Definice}\par\noindent\leftskip 10pt}{\par\bigskip}
\newenvironment{definiceN}[1]{\pagebreak[2]\noindent\textbf{Definice~}\emph{(#1)}\par\noindent\leftskip 10pt}{\par\bigskip}
\newenvironment{definicia}{\pagebreak[2]\noindent\textbf{Definícia}\par \noindent\leftskip 10pt}{\par\bigskip}
\newenvironment{definiciaN}[1]{\pagebreak[2]\noindent\textbf{Definícia~}\emph{(#1)}\par\noindent\leftskip 10pt}{\par\bigskip}

\newenvironment{pozorovani}{\pagebreak[2]\noindent\textbf{Pozorování}\par\noindent\leftskip 10pt}{\par\bigskip}
\newenvironment{pozorovanie}{\pagebreak[2]\noindent\textbf{Pozorovanie}\par\noindent\leftskip 10pt}{\par\bigskip}
\newenvironment{poznamka}{\pagebreak[2]\noindent\textbf{Poznámka}\par\noindent\leftskip 10pt}{\par\bigskip}
\newenvironment{poznamkaN}[1]{\pagebreak[2]\noindent\textbf{Poznámka~}\emph{(#1)}\par\noindent\leftskip 10pt}{\par\bigskip}
\newenvironment{lemma}{\pagebreak[2]\noindent\textbf{Lemma}\par\noindent\leftskip 10pt}{\par\bigskip}
\newenvironment{lemmaN}[1]{\pagebreak[2]\noindent\textbf{Lemma~}\emph{(#1)}\par\noindent\leftskip 10pt}{\par\bigskip}
\newenvironment{veta}{\pagebreak[2]\noindent\textbf{Věta}\par\noindent\leftskip 10pt}{\par\bigskip}
\newenvironment{vetaN}[1]{\pagebreak[2]\noindent\textbf{Věta~}\emph{(#1)}\par\noindent\leftskip 10pt}{\par\bigskip}
\newenvironment{vetaSK}{\pagebreak[2]\noindent\textbf{Veta}\par\noindent\leftskip 10pt}{\par\bigskip}
\newenvironment{vetaSKN}[1]{\pagebreak[2]\noindent\textbf{Veta~}\emph{(#1)}\par\noindent\leftskip 10pt}{\par\bigskip}

\newenvironment{dusledek}{\pagebreak[2]\noindent\textbf{Důsledek}\par\noindent\leftskip 10pt}{\par\bigskip}
\newenvironment{dosledok}{\pagebreak[2]\noindent\textbf{Dôsledok}\par\noindent\leftskip 10pt}{\par\bigskip}

\newenvironment{dokaz}{\pagebreak[2]\noindent\leftskip 10pt\textbf{Dôkaz}\par\noindent\leftskip 10pt}{\par\bigskip}
\newenvironment{dukaz}{\pagebreak[2]\noindent\leftskip 10pt\textbf{Důkaz}\par\noindent\leftskip 10pt}{\par\bigskip}

\newenvironment{priklad}{\pagebreak[2]\noindent\textbf{Příklad}\par\noindent\leftskip 10pt}{\par\bigskip}
\newenvironment{prikladSK}{\pagebreak[2]\noindent\textbf{Príklad}\par\noindent\leftskip 10pt}{\par\bigskip}
\newenvironment{priklady}{\pagebreak[2]\noindent\textbf{Příklady}\par\noindent\leftskip 10pt}{\par\bigskip}
\newenvironment{prikladySK}{\pagebreak[2]\noindent\textbf{Príklady}\par\noindent\leftskip 10pt}{\par\bigskip}

\newenvironment{algoritmusN}[1]{\pagebreak[2]\noindent\textbf{Algoritmus~}\emph{(#1)}\par\noindent\leftskip 10pt}{\par\bigskip}
%obecne prostredie, ktore ma vyuzitie pri specialnych odstavcoch ako (uloha, algoritmus...) aby nevzniklo dalsich x prostredi
\newenvironment{obecne}[1]{\pagebreak[2]\noindent\textbf{#1}\par\noindent\leftskip 10pt}{\par\bigskip}


\newenvironment{penumerate}{
\begin{enumerate}
  \setlength{\itemsep}{1pt}
  \setlength{\parskip}{0pt}
  \setlength{\parsep}{0pt}
  %\setlength{\topsep}{200pt}
  \setlength{\partopsep}{200pt}
}{\end{enumerate}}

\def\pismenka{\numberedlistdepth=2} %pouzit, ked clovek chce opismenkovany zoznam...

\newenvironment{pitemize}{
\begin{itemize}
  \setlength{\itemsep}{1pt}
  \setlength{\parskip}{0pt}
  \setlength{\parsep}{0pt}
}{\end{itemize}}

\definecolor{gris}{gray}{0.95}
\newcommand{\ramcek}[2]{\begin{center}\fcolorbox{white}{gris}{\parbox{#1}{#2}}\end{center}\par}
 \clearpage
\title{\LARGE Učební texty k státní bakalářské zkoušce \\ Správa počítačových systémů \\ Základy teoretické informatiky}
\begin{document}
\maketitle
\newpage
\setcounter{section}{0}
\section{Základy teoretické informatiky}
\begin{pozadavky}
\begin{pitemize}
\item Logika - jazyk, formule, sémantika, tautologie
\item Rozhodnutelnost, splnitelnost, pravdivost, dokazatelnost
\item Normální tvary výrokových formulí, prenexní tvary formulí predikátové logiky
\item Automaty - Chomského hierarchie, třídy automatů a gramatik, determinismus a nedeterminismus.
\end{pitemize}
\end{pozadavky}
\def\c#1{\mathcal{#1}}


\subsection{Logika -- jazyk, formule, sémantika, tautologie}

\subsubsection*{Jazyk}

\begin{obecne}{Logika prvního řádu}
Formální systém logiky prvního řádu obsahuje jazyk, axiomy, odvozovací pravidla, věty a důkazy. Jazyk prvního řádu zahrnuje:
\begin{pitemize}
    \item neomezeně mnoho proměnných $x_1,x_2,\dots $
    \item funkční symboly $f_1,f_2\dots $, každý má aritu  $n\geq 0$
    \item predikátové symboly $p_1,p_2\dots $, každý má aritu
    \item symboly pro logické spojky ($\neg,\vee,\&,\rightarrow,\leftrightarrow$)
    \item symboly pro kvantifikátory ($\forall,\exists$)
    \item může (ale nemusí) obsahovat binární predikát $"="$, který pak se pak ale musí chovat jako rovnost, tj. splňovat určité axiomy (někdy se potom proto řadí mezi logické symboly).
\end{pitemize}
Proměnné, logické spojky a kvantifikátory jsou \emph{logické symboly}, ostatní symboly se nazývají \emph{speciální}.
\end{obecne}

\begin{definiceN}{Jazyk výrokové logiky}
Jazyk výrokové logiky je jazyk prvního řádu, obsahující \emph{výrokové proměnné} (\uv{prvotní formule}), \emph{logické spojky} $\neg,\vee,\&,\rightarrow,\leftrightarrow$ a \emph{pomocné symboly} (závorky).
\end{definiceN}

\begin{definiceN}{Jazyk predikátové logiky}
Jazyk predikátové logiky je jazyk prvního řádu, obsahující proměnné, \emph{predikátové symboly} (s nenulovou aritou), \emph{funkční symboly} (mohou mít nulovou aritu), symboly pro logické spojky a symboly pro \emph{kvantifikátory}.
\end{definiceN}

\subsubsection*{Formule}

\begin{definiceN}{Formule výrokové logiky}
Pro jazyk výrokové logiky jsou následující výrazy formule:
\begin{penumerate}
    \item každá výroková proměnná
    \item pro formule $A,B$ i výrazy $\neg A$, $(A\vee B)$, $(A\& B)$, $(A\rightarrow B)$,
	$A\leftrightarrow B$
    \item každý výraz vzniknuvší konečným užitím pravidel 1. a 2.
\end{penumerate}
Množina formulí se nazývá \emph{teorie}.
\end{definiceN}

\begin{definiceN}{Term}
V predikátové logice je \emph{term}:
\begin{penumerate}
    \item každá proměnná
    \item výraz $f(t_1,\dots,t_n)$ pro $f$ $n$-ární funkční symbol a $t_1,\dots,t_n$ termy
    \item každý výraz vzniknuvší konečným užitím pravidel 1. a 2.
\end{penumerate}
Podslovo termu, které je samo o sobě term, se nazývá \emph{podterm}.
\end{definiceN}

\begin{definiceN}{Formule predikátové logiky}
V predikátové logice je formule každý výraz tvaru $p(t_1,\dots,t_n)$ pro $p$ predikátový symbol a $t_1,\dots,t_n$ termy. Stejně jako ve výrokové logice je formule i (konečné) spojení jednodušších formulí log. spojkami.
Formule jsou navíc i výrazy $(\exists x)A$ a $(\forall x)A$ pro formuli $A$ a samozřejmě cokoliv, co vznikne konečným užitím těchto pravidel.

Podslovo formule, které je samo o sobě formule, se nazývá \emph{podformule}.
\end{definiceN}

\begin{definiceN}{Volné a vázané proměnné}
Výskyt proměnné $x$ ve formuli je \emph{vázaný}, je-li tato součástí nějaké podformule tvaru $(\exists x)A$ nebo $(\forall x)A$. V opačném případě je \emph{volný}. Formule je \emph{otevřená}, pokud neobsahuje vázanou proměnnou, je \emph{uzavřená}, když neobsahuje volnou proměnnou. Proměnná může být v téže formuli volná i vázaná (např. $(x=z)\rightarrow(\exists x)(x=z)$).
\end{definiceN}

\subsubsection*{Sémantika}

\begin{definiceN}{Pravdivostní ohodnocení ve výrokové logice}
Výrokové proměnné samotné neanalyzujeme -- jejich hodnoty máme dány už z vnějšku, máme pro ně \emph{množinu pravdivostních hodnot} ($\{0,1\}$).

\emph{Pravdivostní ohodnocení} $v$ je zobrazení, které každé výrokové proměnné přiřadí právě jednu hodnotu z množiny pravdivostních hodnot. Je-li známo ohodnocení proměnných, lze určit \emph{pravdivostní hodnotu} $\overline{v}$ pro každou formuli (při daném ohodnocení) -- indukcí podle její složitosti, podle tabulek pro logické spojky. 
\end{definiceN}


\begin{definiceN}{Interpretace jazyka predikátové logiky}
\emph{Interpretace jazyka} je definována množinovou strukturou $\c{M}$, která ke každému symbolu jazyka a množině proměnných přiřadí nějakou množinu individuí. $\c{M}$ obsahuje:
\begin{pitemize}
    \item neprázdnou množinu individuí $M$.
    \item zobrazení $f_M:M^n\to M$ pro každý $n$-ární funkční symbol $f$
    \item relaci $p_M\subset M^n$ pro každý $n$-ární predikát $p$
\end{pitemize}

Interpretace termů se uvažuje pro daný jazyk $L$ a jeho interpretaci $\c{M}$. \emph{Ohodnocení proměnných} je zobrazení $e:X\to M$ (kde $X$ je množina proměnných). \emph{Interpretace termu} $t$ při ohodnocení $e$ - $t[e]$ se definuje následovně:
\begin{pitemize}
    \item $t[e]=e(x)$ je-li $t$ proměnná $x$
    \item $t[e]=f_M(t_1[e],\dots,t_n[e])$ pro term tvaru $f(t_1,\dots,t_n)$.
\end{pitemize}
Ohodnocení závisí na zvoleném $\c{M}$, interpretace termů při daném ohodnocení pak jen na konečně mnoha hodnotách z něj. Pokud jsou $x_1,\dots,x_n$ všechny proměnné termu $t$ a $e,e'$ dvě ohodnocení tak, že $e(x_i)=e'(x_i) \forall i\in\{1,\dots,n\}$, pak $t[e]=t[e']$.

\emph{Pozměněné ohodnocení} $y$ pro $x = m\in M$ je definováno: $$e(x/m)(y)=
    \begin{cases}
	m \text{(pro $y\equiv x$)} \\
	e(y) \text{(jinak)}
    \end{cases}
    $$
\end{definiceN}

\begin{definiceN}{Substituce, instance, substituovatelnost}
\emph{Substituce} proměnné za podterm v termu ($t_{x_1,\dots,x_n}[t_1,\dots,t_n]$) je současné nahrazení všech výskytů proměnných $x_i$ termy $t_i$. Je to term. 

\emph{Instance} formule je současné nahrazení všech volných výskytů nějakých proměnných za termy. Je to taky formule, vyjadřuje speciálnější tvrzení -- ne vždy ale lze provést substituci bez změny významu formule. Term je \emph{substituovatelný} do formule $A$ za proměnnou $x$, pokud pro $\forall y$ vyskytující se v $t$ žádná podformule formule $A$ tvaru $(\exists y)B$ ani $(\forall y)B$ neobsahuje volný výskyt $x$.
\end{definiceN}

\begin{definiceN}{Uzávěr formule}
Jsou-li $x_1,\dots,x_n$ všechny proměnné s volným výskytem ve formuli $A$, potom $(\forall x_1)\dots(\forall x_n)A$ je \emph{uzávěr} formule $A$.
\end{definiceN}

\subsubsection*{Tautologie}

\begin{definiceN}{Tautologie ve výrokové logice}
Formule je \emph{tautologie}, jestliže je pravdivá při libovolném ohodnocení proměnných ($\models A$).
\end{definiceN}

\begin{definiceN}{Tautologický důsledek}
Teorie $U$ je \emph{tautologický důsledek} teorie $T$, jestliže každý model $T$ je také modelem $U$ ($T\models U$). Model nějaké teorie ve výrokové logice je takové ohodnocení proměnných, že každá formule z této teorie je pravdivá. K modelům se ještě vrátíme v následující sekci.
\end{definiceN}




\def\c#1{\mathcal{#1}}
\def\Nat{\mathbb{N}}


\subsection{Rozhodnutelnost, splnitelnost, pravdivost a dokazatelnost}

Z těchto témat se rozhodnutelnosti budeme věnovat až jako poslední, protože k vyslovení některých vět budeme potřebovat pojmy, definované v částech o splnitelnosti, pravdivosti a dokazatelnosti.

\subsubsection*{Splnitelnost}

\begin{definiceN}{Splnitelnost}
Množina formulí $T$ ve výrokové logice je \emph{splnitelná}, jestliže existuje ohodnocení $v$ takové, že každá formule $\forall A\in T$ je pravdivá při $v$. Potom se $v$ nazývá \emph{model teorie} $T$ ($v\models T$).
\end{definiceN}


\subsubsection*{Pravdivost}

\begin{definiceN}{Pravdivá formule výrokové logiky}
Formule výrokové logiky $A$ je \emph{pravdivá} při ohodnocení $v$, je-li $\overline{v}(A)=1$, jinak je \emph{nepravdivá}. Je-li formule $A$ pravdivá při ohodnocení $v$, pak říkáme, že $v$ \emph{je model} $A$ ($v\models A$). 
\end{definiceN}

\begin{definiceN}{Tarského definice pravdy}
Pro daný (redukovaný, tj. jen se \uv{základními} log. spojkami) jazyk predikátové logiky $L$, $\c{M}$ jeho interpretaci, ohodnocení $e$ a $A$ formuli tohoto jazyka platí:
\begin{penumerate}
    \item $A$ je \emph{splněna v ohodnocení} $e$ ($\c{M}\models A[e]$), když:
    \begin{pitemize}
	\item $A$ je atomická tvaru $p(t_1,\dots,t_n)$, kde $p$ není rovnost a $(t_1[e],\dots,t_n[e])\in p_M$.
	\item $A$ je atomická tvaru $t_1 = t_2$ a $t_1[e]=t_2[e]$
	\item $A$ je tvaru $\neg B$ a $\c{M}\not\models B[e]$
	\item $A$ je tvaru $B\rightarrow C$ a $\c{M}\not\models B[e]$ nebo $\c{M}\models C[e]$
	\item $A$ je tvaru $(\forall x)B$ a $\c{M}\models B[e(x/m)]$ pro každé $m\in M$
	\item $A$ je tvaru $(\exists x)B$ a $\c{M}\models B[e(x/m)]$ pro nějaké $m\in M$
    \end{pitemize}
    \item $A$ je \emph{pravdivá v interpretaci} $\c{M}$ ($\c{M}\models A$), jestliže je $A$ splněna v $M$ při
	každém ohodnocení proměnných (pro uzavřené formule stačí jedno ohodnocení, splnění je vždy stejné)
   \end{penumerate}
\end{definiceN}

\begin{definiceN}{Logicky pravdivá formule predikátové logiky}
Formule $A$ je \emph{validní (logicky pravdivá)} ($\models A$), když je platná při každé interpretaci daného jazyka.
\end{definiceN}

\subsubsection*{Dokazatelnost}

\begin{definiceN}{Axiomy výrokové logiky}
Pro redukovaný jazyk výrokové logiky (po snížení počtu log. spojek na základní ($\rightarrow, \neg$)) jsou \emph{axiomy výrokové logiky} (schémata axiomů) všechny formule následujících tvarů:
\begin{pitemize}
    \item $(A\rightarrow (B\rightarrow A))$ 
	(A1 - \uv{implikace sebe sama})
    \item $(A\rightarrow (B\rightarrow C))\rightarrow ((A\rightarrow B) \rightarrow (A\rightarrow C))$ 
	(A2 - \uv{roznásobení})
    \item $(\neg B\rightarrow \neg A)\rightarrow(A\rightarrow B)$ 
	(A3 - \uv{obrácená negace implikace})
\end{pitemize}
\end{definiceN}

\begin{definiceN}{Odvozovací pravidlo výrokové logiky}
Výroková logika má jediné odvozovací pravidlo -- \emph{modus ponens}:
$$\frac{A,A\rightarrow B}{B} $$
\end{definiceN}

\begin{definiceN}{Důkaz ve výrokové logice}
Důkaz $A$ je konečná posloupnost formulí $A_1,\dots A_n$, jestliže $A_n = A$ a pro každé $i=1,..n$ je $A_i$ buď axiom, nebo je odvozená z předchozích pravidlem modus ponens. Existuje-li důkaz formule $A$, pak je tato \emph{dokazatelná} ve výrokové logice (je větou výrokové logiky - $\vdash A$).
\end{definiceN}

\begin{definiceN}{Důkaz z předpokladů}
Důkaz formule $A$ z předpokladů je posloupnost formulí $A_1,\dots A_n$ taková, že $A_n = A$ a $\forall i\in\{1,..n\}$ je $A_i$ axiom, nebo prvek množiny předpokladů $T$, nebo je odvozena z přechozích pravidlem modus ponens. Existuje-li důkaz $A$ z $T$, pak $A$ \emph{je dokazatelná} z $T$ - $T\vdash A$.
\end{definiceN}

\begin{vetaN}{o dedukci ve výrokové logice}
Pro $T$ množinu formulí a formule $A,B$ platí: $$T\vdash A\rightarrow B \mbox{ právě když } T,A\vdash B$$
\end{vetaN}


\begin{definice}
Množina formulí $T$ je \emph{sporná}, pokud je z předpokladů $T$ dokazatelná libovolná formule, jinak je $T$ \emph{bezesporná}. $T$ je \emph{maximální bezesporná} množina, pokud je $T$ bezesporná a navíc jediná její bezesporná nadmnožina je $T$ samo. Množina všech formulí dokazatelných z $T$ se značí $\mathit{Con}(T)$.
\end{definice}

\begin{vetaN}{Lindenbaumova}
Každou bezespornou množinu formulí výrokové logiky $T$ lze rozšířit na maximální bezespornou $S$, $T\subset S$.
\end{vetaN}

\begin{vetaN}{o bezespornosti a splnitelnosti}
Množina formulí výrokové logiky je bezesporná, právě když je splnitelná.
\end{vetaN}

\begin{vetaN}{Věta o úplnosti výrokové logiky}
Je-li $T$ množina formulí a $A$ formule, pak platí:
\begin{penumerate}
    \item $T\vdash A$ právě když $T\models A$
    \item $\vdash A$ právě když $\models A$ ($A$ je větou výrokové logiky, právě když je tautologie)
\end{penumerate}
Tedy výroková logika je bezesporná a jsou v ní dokazatelné právě tautologie.
\end{vetaN}

\begin{vetaN}{o kompaktnosti}
Množina formulí výrokové logiky je splnitelná, právě když je splnitelná každá její konečná podmnožina.
\end{vetaN}


\begin{definiceN}{Formální systém predikátové logiky}
Pracujeme s redukovaným jazykem (jen s log. spojkami $\neg,\rightarrow$ a jen s kvantifikátorem $\forall$). \emph{Schémata Axiomů predikátové logiky} vzniknou z těch ve výrokové logice prostým dosazením libovolných formulí predikátové logiky za výrokové proměnné. \emph{Modus ponens} platí i v pred. logice. Další axiomy a pravidla:
\begin{pitemize}
    \item \emph{schéma(axiom) specifikace}: $(\forall x)A\rightarrow A_x[t]$
    \item \emph{schéma přeskoku}: $(\forall x)(A\rightarrow B)\rightarrow (A\rightarrow(\forall x)B)$, pokud 
	proměnná $x$ nemá volný výskyt v $A$.
    \item \emph{pravidlo generalizace}: $\frac{A}{(\forall x)A}$
\end{pitemize}
Toto je formální systém pred. logiky \emph{bez rovnosti}. S rovností přibývá symbol $=$ a další tři axiomy.
\end{definiceN}


\begin{poznamkaN}{Vlastnosti formulí predikátové logiky}
\begin{penumerate}
    \item Je-li $A'$ instance formule $A$, pak jestliže platí $\vdash A$, platí i $\vdash A'$.
	(\emph{Věta o instancích})
    \item Je-li $A'$ uzávěr formule $A$, pak $\vdash A$ platí právě když $\vdash A'$. 
	(\emph{Věta o uzávěru})
\end{penumerate}
\end{poznamkaN}

\begin{vetaN}{o dedukci v predikátové logice}
Nechť $T$ je množina formulí pred. logiky, $A$ je uzavřená formule a $B$ lib. formule, potom $T\vdash A\rightarrow B$ právě když $T,A\vdash B$.
\end{vetaN}

\begin{definiceN}{Teorie, model}
Pro nějaký jazyk $L$ prvního řádu je množina $T$ formulí tohoto jazyka \emph{teorie prvního řádu}. Formule z $T$ jsou \emph{speciální axiomy} teorie $T$. Pro interpretaci $\c{M}$ jazyka $L$ je $\c{M}$ \emph{model teorie} $T$ ($\c{M}\models T$), pokud jsou všechny speciální axiomy $T$ pravdivé v $\c{M}$. Formule $A$ je \emph{sémantickým důsledkem} $T$: $T\models A$, jestliže je pravdivá v každém modelu teorie $T$.
\end{definiceN}

\begin{vetaN}{o korektnosti}
Je-li $T$ teorie prvního řádu a $A$ formule, potom platí:
\begin{penumerate}
    \item Jestliže $T\vdash A$, potom $T\models A$.
    \item Speciálně jestliže $\vdash A$, potom $\models A$.
\end{penumerate}
\end{vetaN}

\begin{vetaN}{o úplnosti v predikátové logice}
Nechť $T$ je teorie s jazykem prvního řádu $L$. Je-li $A$ lib. formule jazyka $L$, pak platí:
\begin{penumerate}
    \item $T\vdash A$ právě když $T\models A$
    \item $T$ je bezesporná, právě když má model.
\end{penumerate}
\end{vetaN}

\begin{definiceN}{Úplná teorie}
Teorie $T$ s jazykem $L$ prvního řádu je \emph{úplná}, je-li bezesporná a pro libovolnou uzavřenou formuli $A$ je jedna z formulí $A,\neg A$ dokazatelná v $T$.
\end{definiceN}

\begin{vetaN}{o kompaktnosti}
Teorie $T$ s jazykem $L$ prvního řádu má model, právě když každý její konečný fragment $T'\subset T$ má model. Tj. pro libovolnou formuli $A$ jazyka $L$ platí: $T\models A$ právě když $T'\models A$ pro nějaký konečný fragment $T'\subset T$.
\end{vetaN}

\subsubsection*{Rozhodnutelnost}

\begin{definiceN}{Rekurzivní funkce a množina}
\emph{Rekurzivní funkce} jsou všechny funkce popsatelné jako $f:\Nat^k\to\Nat$, kde $k\geq 1$, tedy všechny \uv{algoritmicky vyčíslitelné} funkce. Množina přirozených čísel je \emph{rekurzivní množina (rozhodnutelná množina)}, pokud je rekurzivní její charakteristická funkce (to je funkce, která určí, které prvky do množiny patří).
\end{definiceN}

\begin{definiceN}{Spočetný jazyk, kód formule}
\emph{Spočetný jazyk} je jazyk, který má nejvýš spočetně mnoho speciálních symbolů. Pro spočetný jazyk, kde lze efektivně (rekurzivní funkcí) očíslovat jeho speciální symboly, lze každé jeho formuli $A$ přiřadit její \emph{kód formule} - přir. číslo $\#A$.
\end{definiceN}

\begin{vetaN}{Churchova o nerozhodnutelnosti predikátové logiky}
Pokud spočetný jazyk $L$ prvního řádu obsahuje alespoň jednu konstantu, alespoň jeden funkční symbol arity $k>0$ a pro každé přirozené číslo spočetně mnoho predikátových symbolů, potom množina $\{\#A|A \text{ je uzavřená formule a }L\models A\}$ není rozhodnutelná.
\end{vetaN}

\begin{vetaN}{o nerozhodnosti predikátové logiky}
Nechť $L$ je jazyk prvního řádu bez rovnosti a obsahuje alespoň 2 binární predikáty. Potom je predikátová logika (jako teorie) s jazykem $L$ nerozhodnutelná.
\end{vetaN}

\begin{definiceN}{Tři popisy aritmetiky}
Je dán jazyk $L=\{0,S,+,\cdot\,\leq\}$.
\begin{pitemize}
    \item \emph{Robinsonova aritmetika} - "$Q$" s jazykem L má 8 násl. axiomů:
    \begin{penumerate}
	\item $S(x)\neq 0$
	\item $S(x)=S(y)\rightarrow x=y$
	\item $x\neq 0\rightarrow (\exists y)(x=S(y))$
	\item $x+0=x$
	\item $x+S(y)=S(x+y)$
	\item $x\cdot 0=0$
	\item $x\cdot S(y)=(x\cdot y)+x$
	\item $x\leq y\leftrightarrow (\exists z)(z+x=y)$
    \end{penumerate}

    \textit{Poznámka: Někdy, pokud není potřeba definovat uspořádání, se poslední axiom spolu se symbolem \uv{\leq} vypouští.}

    \item \emph{Peanova aritmetika} - "$P$" má všechny axiomy Robinsonovy kromě třetího, navíc má 
	\emph{Schéma(axiomů) indukce} - pro formuli $A$ a proměnnou $x$ platí: $A_x[0]\rightarrow 
	\{(\forall x)(A\rightarrow A_x[S(x)])\rightarrow(\forall x)A\}$.
    \item \emph{Úplná aritmetika} má za axiomy všechny uzavřené formule pravdivé v $\Nat$, je-li $\Nat$
	standardní model aritmetiky - \uv{pravdivá aritmetika}. \emph{Teorie modelu $\Nat$} je množina
	$Th(\Nat)=\{A|A\text{ je uzavřená formule a } N\models A\}$.
\end{pitemize}
Platí: $Q\subseteq P\subseteq Th(\Nat)$. $Q$ má konečně mnoho axiomů, je tedy rekurzivně axiomatizovatelná. $P$ má spočetně mnoho axiomů, kódy axiomů schématu indukce tvoří rekurzivní množinu. $Th(\Nat)$ není rekurzivně axiomatizovatelná.
\end{definiceN}


\begin{definiceN}{Množina kódů vět teorie}
Pro $T$ teorii s jazykem aritmetiky definujeme \emph{množinu kódů vět teorie} $T$ jako $Thm(T)=\{\#A|A \text{ je uzavřená formule a } T\vdash A\}$.
\end{definiceN}

\begin{definiceN}{Rozhodnutelná teorie}
Teorie $T$ s jazykem aritmetiky je \emph{rozhodnutelná}, pokud je množina $Thm(T)$ rekurzivní. V opačném případě je $T$ \emph{nerozhodnutelná}.
\end{definiceN}


\begin{vetaN}{Churchova o nerozhodnutelnosti aritmetiky}
Každé bezesporné rozšíření Robinsonovy aritmetiky $Q$ je nerozhodnutelná teorie.
\end{vetaN}

\begin{vetaN}{Gödel-Rosserova o neúplnosti aritmetiky}
Žádné bezesporné a rekurzivně axiomatizovatelné rozšíření Robinsonovy aritmetiky $Q$ není úplná teorie.
\end{vetaN}



\subsection{Normální tvary výrokových formulí, prenexní tvary formulí predikátové logiky}

\begin{poznamkaN}{Vlastnosti log. spojek}
Platí:
\begin{penumerate}
    \item $A\wedge B\vdash A$; $A,B\vdash A\wedge B$
    \item $A\leftrightarrow B\vdash A\rightarrow B$; $A\rightarrow B, B\rightarrow A\vdash A\leftrightarrow B$
    \item $\wedge$ je idempotentní, komutativní a asociativní.
    \item $\vdash(A_1\rightarrow\dots(A_n\rightarrow B)\dots)
	\leftrightarrow((A_1\wedge\dots\wedge A_n)\rightarrow B)$
    \item DeMorganovy zákony: $\vdash\neg(A\wedge B)\leftrightarrow(\neg A\vee\neg B)$;
	$\vdash\neg(A\vee B)\leftrightarrow(\neg A\wedge\neg B)$
    \item $\vee$ je monotonní ($\vdash A\rightarrow A\vee B$), idempotentní, komutativní a asociativní.
    \item $\vee$ a $\wedge$ jsou navzájem distributivní.
\end{penumerate}
\end{poznamkaN}

\begin{vetaN}{o ekvivalenci ve výrokové logice}
Jestliže jsou podformule $A_1\dots A_n$ formule $A$ ekvivalentní s $A'_1\dots A'_n$ ($\vdash A'_i \leftrightarrow A_i$) a $A'$ vytvořím nahrazením $A'_i$ místo $A_i$, je i $A$ ekvivalentní s $A'$. (Důkaz indukcí podle složitosti formule, rozborem případů $A_i$ tvaru $\neg B$, $B\rightarrow C$)
\end{vetaN}

\begin{lemmaN}{o důkazu rozborem případů}
Je-li $T$ množina formulí a $A,B,C$ formule, pak $T,(A\vee B)\vdash C$ platí právě když $T,A\vdash C$ a $T,B\vdash C$.
\end{lemmaN}

\begin{definiceN}{Normální tvary}
Výrokovou proměnnou nebo její negaci nazveme \emph{literál}. \emph{Klauzule} budiž disjunkce několika literálů. \emph{Formule v normálním konjunktivním tvaru (CNF)} je konjunkce klauzulí. \emph{Formule v disjunktivním tvaru (DNF)} je disjunkce konjunkcí literálů.
\end{definiceN}

\begin{vetaN}{o normálních tvarech}
Pro každou formuli $A$ lze sestrojit formule $A_k,A_d$ v konjunktivním, resp. disjunktivním tvaru tak, že $\vdash A\leftrightarrow A_d$, $\vdash A\leftrightarrow A_k$. (Důkaz z DeMorganových formulí a distributivity, indukcí podle složitosti formule)
\end{vetaN}

\subsubsection*{Prenexní tvary formulí predikátové logiky}

\begin{vetaN}{o ekvivalenci v predikátové logice}
Nechť formule $A'$ vznikne z $A$ nahrazením některých výskytů podformulí $B_1,\dots,B_n$ po řadě formulemi $B'_1,\dots,B'_n$. Je-li $\vdash B_1\leftrightarrow B'_1,\dots,\vdash B_n\leftrightarrow B'_n$, potom platí i $\vdash A\leftrightarrow A'$.
\end{vetaN}

\begin{definiceN}{Prenexní tvar}
Formule predikátové logiky $A$ je v \emph{prenexním tvaru}, je-li $$A\equiv (Q_1 x_1)(Q_2 x_2)\dots(Q_n x_n)B,$$ kde $n\geq 0$ a $\forall i\in\{1,\dots,n\}$ je $Q_i\equiv \forall$ nebo $\exists$, $B$ je otevřená formule a kvantifikované proměnné jsou navzájem různé. $B$ je \emph{otevřené jádro} $A$, část s kvantifikátory je \emph{prefix} $A$.
\end{definiceN}

\begin{definiceN}{Varianta formule predikátové logiky}
Formule $A'$ je \emph{varianta} $A$, jestliže vznikla z $A$ postupným nahrazením podformulí $(Q x)B$ (kde $Q$ je $\forall$ nebo $\exists$) formulemi $(Q y)B_x[y]$ a $y$ není volná v $B$. Podle \emph{věty o variantách} je varianta s původní formulí ekvivalentní.
\end{definiceN}

\begin{lemmaN}{o prenexních operacích}
Pro převod formulí do prenexního tvaru se používají tyto operace (výsledná formule je s původní ekvivalentní). Pro podformule $B$, $C$, kvantifikátor $Q$ a proměnnou $x$:
\begin{penumerate}
    \item podformuli lze nahradit nějakou její variantou
    \item $\vdash \neg(Q x)B\leftrightarrow(\overline{Q} x)\neg B$
    \item $\vdash (B\rightarrow (Q x)C)\leftrightarrow(Q x)(B\rightarrow C)$, pokud $x$ není volná v $B$
    \item $\vdash ((Q x)B\rightarrow C)\leftrightarrow(\overline{Q} x)(B\rightarrow C)$, pokud $x$ není 
	volná v $C$
    \item $\vdash ((Q x)B\wedge C)\leftrightarrow (Q x)(B\wedge C)$, pokud $x$ není volná v $C$
    \item $\vdash ((Q x)B\vee C)\leftrightarrow (Q x)(B\vee C)$, pokud $x$ není volná v $C$
\end{penumerate}
\end{lemmaN}

\begin{vetaN}{o prenexních tvarech}
Ke každé formuli $A$ predikátové logiky lze sestrojit ekvivalentní formuli $A'$, která je v prenexním tvaru. (Důkaz: indukcí podle složitosti formule a z prenexních operací, někdy je nutné přejmenovat volné proměnné)
\end{vetaN}

\def\implies{\Rightarrow}
\def\Nat{\mathbb{N}}
\def\Real{\mathbb{R}}
\def\isimplied{\Leftarrow}
\def\onlyif{\Leftrightarrow}
\def\c#1{\mathcal{#1}}


\subsection{Automaty -- Chomského hierarchie, třídy automatů a gramatik, determinismus a nedeterminismus.}


TODO: přesunout věty a popisy gramatik do sekce o Chomskeho hierarchii, za definici gramatiky (abychom vedeli s cim operujeme ;))

\subsubsection*{Třídy automatů a gramatik}

\begin{definiceN}{Konečný automat}
\emph{Konečný automat} je pětice $A = (Q,X,\delta,q_0,F)$, kde $Q$ je stavový prostor (množina všech možných stavů), $X$ je \emph{abeceda} (množina symbolů), $\delta$ je přechodová funkce $\delta: Q\times X\to Q$, $q_0\in Q$ je poč. stav a $F\subseteq Q$ množina koncových stavů.
\end{definiceN}

\begin{definice}
\emph{Slovo} $w$ je posloupnost symbolů v abecedě $X$. \emph{Jazyk} $L$ je množina slov, tedy $L\subseteq X^{\ast}$, kde $X^{\ast}$ je množina všech posloupností symbolů abecedy $X$. $\mathbf{\lambda}$ je prázdná posloupnost symbolů. \emph{Rozšířená přechodová funkce} je $\delta^{\ast}:Q\times X^{\ast} \to Q$ - tranzitivní uzávěr $\delta$. Jazyk rozpoznávaný konečným automatem -- \emph{regularní jazyk} je $L(A) = \{ w | w\in X^{\ast}, \delta^{\ast}(q_0,w)\in F \}$. \emph{Pravá kongruence} je taková relace ekvivalence na $X^{\ast}$, že $\forall u,v,w\in X^{\ast}: u \sim v \implies uw \sim vw$. Je \emph{konečného indexu}, jestliže $X^{\ast}/\sim$ má konečný počet tříd.
\end{definice}

\begin{vetaN}{Nerodova}
Jazyk $L$ nad konečnou abecedou $X$ je rozpoznatelný kon. automatem, právě když existuje pravá kongruence konečného indexu na $X^{\ast}$ tak, že $L$ je sjednocením jistých tříd rozkladu $X^{\ast}/\sim$.
\end{vetaN}
\begin{vetaN}{Pumping (iterační) lemma}
Pro jazyk rozpoznatelný kon. automatem $L$ existuje $n\in \Nat$ tak, že libovolné slovo $z\in L,|z|\geq n$ lze psát jako $uvw$, kde $|uv|\leq n$, $|v|\geq 1$ a $\forall i\geq 0: uv^{i}w\in L$.
\end{vetaN}

\begin{definice}
Dva automaty jsou \emph{ekvivalentní}, jestliže přijímají stejný jazyk. \emph{Homomorfismus (isomorfismus)} automatů je zobrazení, zachovávající poč. stav, přech. funkci i konc. stavy (+ prosté a na). Pokud existuje ismorfismus automatů $A\to B$, pak jsou tyto dva ekvivalentní (jen 1 implikace!). \emph{Dosažitelný stav} $q$ - $\exists w \in X^{\ast}: \delta^{\ast}(q_0,w) = q$. Relace ekvivalence je \emph{automatovou kongruencí}, pokud zachovává konc. stavy a přech. funkci. Ke každému automatu existuje \emph{redukt} - ekvivaletní automat bez nedosažitelných a navzájem ekvivalentních stavů. Ten je určen jednoznačně pro daný jazyk (až na isomorfismus), proto lze zavést normovaný tvar.
\end{definice}

\begin{poznamkaN}{Operace s jazyky}
S jazyky lze provádět množinové operace ($\cup, \cap$), rozdíl ($\{ w | w \in L_1 \& w \notin L_2 \}$), doplněk ($\{ w | w \notin L \}$), dále zřetězení ($L_1\cdot L_2 = \{ uv | u \in L_1, v \in L_2 \}$), mocniny ($L^0={\lambda}, L^{i+1} = L^{i}\cdot L$), iterace ($L^{\ast}=L^0 \cup L^1 \cup L^2 \cup ...$), otočení $L^R$, levý (i pravý) kvocient ($K\setminus L = \{ v | uv \in L, u \in K \}$) a derivace (kvocienty podle jednoslovného jazyka). Třída jazyků rozpoznatelných konečnými automaty je na tyto operace uzavřená.
\end{poznamkaN}

\begin{definiceN}{Regulární jazyky}
\emph{Třída reguláních jazyků} nad abecedou $X$ je nejmenší třída, která obsahuje $\emptyset$, $\forall x \in X$ obsahuje ${x}$ a je uzavřená na sjednocení, iteraci a zřetězení.
\end{definiceN}

\begin{vetaN}{Kleenova}
Jazyk je regulární, právě když je rozpoznatelný konečným automatem.
\end{vetaN}

\begin{definiceN}{Regulární výrazy}
\emph{Regulární výrazy} nad abecedou $X={x_1,...,x_n}$ jsou nejmenší množina slov v abecedě ${x_1,...,x_n,\emptyset,\lambda,^{+},\cdot,^{\ast},(,)}$, která obsahuje výrazy $\emptyset$ a $\lambda$ a $\forall i$ obsahuje $x_i$ a je uzavřená na sjednocení ($+$), zřetězení ($\cdot$) a iterace ($^{\ast}$). \emph{Hodnota reg. výrazu} $a$ je reg. jazyk $[a]$, lze takto reprezentovat každý reg. jazyk.
\end{definiceN}

\begin{definiceN}{Dvoucestné konečné automaty}
\emph{Dvoucestný konečný automat} je pětice $(Q,X,\delta,q_0,F)$, kde oproti kon. automatu je $\delta:Q\times X\to Q\times \{-1,0,1\}$ (tj. pohyb čtecí hlavy). Přijímá slovo, pokud výpočet začal vlevo v poč. stavu a čtecí hlava opustila slovo $w$ vpravo v konc. stavu (mimo slovo končí výpočet okamžitě).
\end{definiceN}
\begin{poznamka}
Jazyky přijímané dvoucestnými automaty jsou regulární - každý dvoucestný automat lze převést na (nedeterministický) konečný automat.
\end{poznamka}


\begin{definiceN}{Zásobníkové automaty}
\emph{Zásobníkový automat} je sedmice $M=(Q,X,Y,\delta,q_0,Z_0,F)$, kde proti konečným automatům je $Y$ abeceda pro symboly na zásobníku, $Z_0$ počáteční symbol na zásobníku a funkce instrukcí $\delta:Q\times(X\cup\{\lambda\})\times Y\ \to\ \c{P}(Q\times Y^{\ast})$. Je z principu nedeterministický; vždy se nahrazuje vrchol zásobníku, nečte ale pokaždé vstupní symboly. Instrukci $(p,a,Z)\to(q,w)$ lze vykonat, pokud je automat ve stavu $p$, na zásobníku je $Z$ a na vstupu $a$. Vykonání instrukce znamená změnu stavu, pokud $a\neq\lambda$, tak i posun čtecí hlavy a odebrání $Z$ ze zásobníku, kam se vloží $w$ (prvním písmenem nahoru). Výpočet končí buď přečtením slova, nebo v případě, že pro danou situaci není definována instrukce
(\emph{Situace} zás. automatu je trojice $(p,u,v)$, kde $p\in Q$, $u$ je nepřečtený zbytek slova a $v$ celý zásobník ).

Přijímat slovo je možné buď koncovým stavem (slovo je přečteno a automat v konc. stavu), nebo zásobníkem (slovo je přečteno a zásobník prázdný -- konc. stavy jsou v takovém případě nezajímavé - $F=\emptyset$).
\end{definiceN}

\begin{poznamka}
Pro zás. automat přijímající konc. stavem vždy existuje ekvivalentní automat ($L(A_1)=L(A_2)$) přijímající zásobníkem a naopak.
\end{poznamka}

\begin{veta}
Každý bezkontextový jazyk je rozpoznáván zásobníkovým automatem, přijímajícím prázdným zásobníkem. Stejně pro každý zásobníkový automat existuje bezkontextová gramatika, která generuje jazyk jím přijímaný.
\end{veta}

\begin{poznamkaN}{Vlastnosti bezkontextových gramatik}
Bezkontextová gramatika je \emph{redukovaná}, pokud $\forall X\in V_N$ existuje terminální slovo $w\in V_T^{\ast}$ tak, že $X\Rightarrow^{\ast} w$ a navíc $\forall X\in V_N, X\neq S$ existují slova $u,v$ tak, že $S\Rightarrow^{\ast}uXv$. Ke každé bezkontextové gramatice lze sestrojit ekvivalentní redukovanou.

Pro každé terminální slovo v bezkontextové gramatice existují derivace, které se liší jen pořadím použití pravidel (a prohozením některých pravidel dostanu stejné terminální slovo), proto lze zavést \emph{levé (pravé) derivace} - tj. \emph{kanonické} derivace. Pokud $X\Rightarrow^{\ast}w$, pak existuje i levá (pravá) derivace. Znázornění průběhu derivací je možné určit \emph{derivačním stromem} -- určuje jednoznačně pravou/levou derivaci.

Bezkontextová gramatika je \emph{víceznačná} (nejednoznačná), pokud v ní existuje slovo, které má dvě různé levé derivace; jinak je \emph{jednoznačná}. Jazyk je jednoznačný, pokud k němu existuje generující jednoznačná gramatika. Pokud je každá gramatika nějakého jazyka nejednoznačná, je tento \emph{podstatně nejednoznačný}. 
\end{poznamkaN}


\begin{definiceN}{Greibachové normální forma}
Gramatika je v \emph{Greibachové normální formě}, jsou-li všechna její pravidla ve tvaru $A\to au$, kde $a\in V_T$ a $u\in V_N^{\ast}$. Ke každému bezkontextovému jazyku existuje gramatika v G. normální formě tak, že $L(G)=L\setminus\{\lambda\}$. Každou bezkontextovou gramatiku lze převést do G. normální formy.
\end{definiceN}

\begin{poznamkaN}{Úpravy bezkontextových gramatik}
Spojením více pravidel ($A\to uBv, B\to w_1, ...B\to w_k$ se převede na $A\to uw_{1}v | ... | uw_{k}v$) dostanu ekvivalentní gramatiku. Stejně tak odstraněním levé rekurze (převod přes nový neterminál). 
\end{poznamkaN}

\begin{definiceN}{Chomského normální forma}
Pro gramatiku v \emph{Chomského normální formě} jsou všechna pravidla tvaru $X\to YZ$ nebo $X\to a$, kde $X,Y,Z\in V_N$, $a\in V_T$. Ke každému bezkontextovému jazyku $L$ existuje gramatika $G$ v Chomského normální formě tak, že $L(G)=L\setminus\{\lambda\}$
\end{definiceN}

\begin{poznamkaN}{Vlastnosti třídy bezkontextových jazyků}
Třída bezkontextových jazyků je uzavřená na sjednocení, zrcadlení, řetězení, iteraci a pozitivní iteraci, substituci a homomorfismus, inverzní homomorfismus a kvocient s regulárním jazykem. Není uzavřená na průnik a doplněk.
\end{poznamkaN}

\begin{definiceN}{Dyckův jazyk}
\emph{Dyckův jazyk} je definován nad abecedou ${a_1,a'_1,...a_n,a'_n}$ gramatikou $$S\to \lambda | SS | a_{1}Sa'_1 | ... | a_{n}Sa'_n$$ Je bezkontextový, popisuje správná uzávorkování a lze jím popisovat výpočty zásobníkových automatů, tedy i bezkontextové jazyky.
\end{definiceN}

\begin{definiceN}{Turingův stroj}
\emph{Turingův stroj} je pětice $T=(Q,X,\delta,q_0,F)$, kde $X$ je abeceda, obsahující symbol $\varepsilon$ pro prázdné políčko, přechodová funkce $\delta:(Q\setminus F)\times X\to Q\times X\times\{-1,0,1\}$ popisuje změnu stavu, zápis na pásku a posun hlavy. Výpočet končí, není-li definována žádná instrukce (spec. platí pro $q\in F$). \emph{Konfigurace} Turingova stroje jsou údaje, popisující stav výpočtu -- nejmenší souvislá část pásky, obsahující všechny neprázdné buňky a čtenou buňku, vnitřní stav a poloha hlavy. \emph{Krok výpočtu} je $uqv \vdash wpz$ pro $u$ část slova vlevo od akt. pozice na pásce, $v$ od čteného písmena dál a $q$ stav stroje. \emph{Výpočet} je posloupnost kroků, slovo $w$ je přijímáno, pokud $q_{0}w \vdash^{\ast} upv$, $p\in F$. Jazyky (množiny slov bez $\varepsilon$) přijímané Turingovými stroji jsou \emph{rekurzivně spočetné}.
\end{definiceN}

\begin{veta}
Každý jazyk typu 0 (s gramatikou s obecnými pravidly) je rekurzivně spočetný.
\end{veta}

\subsubsection*{Chomského hierarchie}

\begin{definiceN}{Přepisovací systém}
\emph{Přepisovací (produkční) systém} je dvojice $R=(V,P)$, kde $V$ je konečná abeceda a $P$ množina přepisovacích pravidel (uspořádaných dvojic prvků z $V^{\ast}$). Slovo $w$ se \emph{přímo přepíše} na $z$ ($w\Rightarrow z$), pokud $\exists u,v,x,y\in V: w = xuy, z = xvy, (u,v)\in P$. Derivace (\emph{odvození}) je zřetězení několika přímých přepsání.
\end{definiceN}

\begin{definiceN}{Generativní gramatika}
\emph{Generativní gramatika} je čtveřice $G=(V_N,V_T,S,P)$, kde $V_N$ je množina neterminálních symbolů, $V_T$ množina terminálních symbolů, $S$ startovací symbol ($S\in V_N$) a $P$ množina pravidel. \emph{Jazyk generovaný gramatikou} je $L(G)=\{w\in V_T^{\ast}, S\Rightarrow^{\ast}V\}$.
\end{definiceN}

\begin{definiceN}{Chomského hierarchie}
\emph{Chomského hierarchie} je rozdělení gramatik do 4 tříd podle omezení na pravidla:
\begin{pitemize}
    \item \emph{G0 (Rekurzivně spočetné jazyky)} mohou mít obecná pravidla.
    \item \emph{G1 (Kontextové jazyky/gramatiky)} - jen pravidla tvaru
	$\alpha X\beta \to \alpha\omega\beta$, kde $X\in V_N$ a $\alpha,\beta,\omega \in (V_N\cup V_T)^{\ast}$, navíc $|\omega|>0$. Může obsahovat i pravidlo $S\to\lambda$, ale pak se $S$ nesmí vyskytovat na pravé straně žádného pravidla.
    \item \emph{G2 (Bezkontextové jazyky/gramatiky)} - jen pravidla tvaru 
	$X \to \omega$, kde $X\in V_N$, $\omega\in(V_N\cup V_T)^{\ast}$.
    \item \emph{G3 (Regulární jazyky/pravé lineární gramatiky)} - jen pravidla typu
	$X\to\omega Y$ a $X\to\omega$, kde $\omega\in V_T^{\ast}$ a $X,Y\in V_N$. 
\end{pitemize}
Definuje uspořádání tříd jazyků podle inkluze, tedy $\c{L}0 \supset \c{L}1 \supset \c{L}2 \supset \c{L}3$.
\end{definiceN}

\begin{poznamka}
S $\c{L}1 \supset \c{L}2$ nastává problém, protože bezkontextové gramatiky umožňují pravidla tvaru $X\to\lambda$. Řešením je převod na \emph{nevypouštějící bezkontextové gramatiky} - takové bezkontextové gramatiky, které nemají pravidla typu $X\to\lambda$.
\end{poznamka}

\begin{vetaN}{o nevypouštějících bezkontextových gramatikách}
Ke každé bezkontextové $G$ existuje nevypouštějící bezkontextová $G_0$ tak, že $$L(G_0)=L(G)\setminus\{\lambda\}$$ Je-li $\lambda\in L(G)$, pak $\exists G_1$, t.ž. $L(G_1)=L(G)$ a jediné pravidlo v $G_1$ s $\lambda$ na pravé straně je $S\to\lambda$ a $S$ není v $G_1$ na pravé straně žádného pravidla.
\end{vetaN}

\begin{poznamkaN}{Lineární gramatiky}
Pro každou gramatiku typu G3 lze sestrojit konečný automat, který přijímá právě jazyk jí generovaný, stejně tak pro každý konečný automat lze sestrojit gramatiku G3. Levé lineární gramatiky také generují regulární jazyky, díky uzavřenosti na reverzi. \emph{Lineární gramatiky}, s pravidly typu $X\to uYv,X\to w$, kde $X,Y\in V_N, u,v,w\in V_T^{\ast}$, generují lineární jazyky - silnější než regulární jazyky.
\end{poznamkaN}

\begin{definiceN}{Separovaná a nevypouštějící gramatika}
\emph{Separovaná gramatika} je gramatika (obecně libovolné třídy), obsahující pouze pravidla tvaru $\alpha\to\beta$, kde buď $\alpha,\beta\in V_N^+$, nebo $\alpha\in V_N$ a $\beta\in V_T\cup\{\lambda\}$. \emph{Nevypouštějící (monotónní) gramatika} (také se neomezuje na konkrétní třídu) je taková, že pro každé pravidlo $u\to v$ platí $|u|\leq|v|$.
\end{definiceN}

\begin{poznamkaN}{Kontextové gramatiky}
Ke každé kontextové gramatice lze sestrojit ekvivalentní separovanou. Ke každé monotónní gramatice lze nalézt ekvivalentní kontextovou.
\end{poznamkaN}


\subsubsection*{Determinismus a nedeterminismus}

\begin{definiceN}{Nedeterministický konečný automat}
\emph{Nedeterministický konečný automat} je pětice $(Q,X,\delta,S,F)$, kde $Q$ je mn. stavů, $X$ abeceda, $F$ mn. konc. stavů, $S$ množina počátečních stavů a $\delta:Q\times X\to \c{P}(Q)$ je přechodová funkce. Slovo $w$ je takovým automatem přijímáno, pokud existuje posloupnost stavů $\{q_i\}_{i=1}^n$ tak, že $q_1\in S$, $q_{i+1}\in\delta(q_i,x_i)$, $q_{n+1}\in F$.
\end{definiceN}

\begin{poznamka}
Pro každý nedeterministický konečný automat $A$ lze sestrojit deterministický kon. automat $B$ tak, že jimi přijímané jazyky jsou ekvivalentní (může to znamenat exponenciální nárůst počtu stavů).
\end{poznamka}


\begin{definiceN}{Deterministický zásobníkový automat}
\emph{Deterministický zásobníkový automat} je $M=(Q,X,Y,\delta,q_0,Z_0,F)$ takové, že $\forall p\in Q, \forall a\in (X\cup\{\lambda\}), \forall Z\in Y$ platí $|\delta(p,a,Z)|\leq 1$ a navíc pokud pro nějaké $p, Z$ je $\delta(p,\lambda,Z)\neq\emptyset$, pak $\forall a\in X$ je $\delta(p,a,Z)=\emptyset$. 
\end{definiceN}

\begin{poznamka}
Deterministický zásobníkový automat je \uv{slabší} než nedeterministický, rozpoznává \emph{deterministické bezkontextové jazyky} koncovým stavem a \emph{bezprefixové bezkontextové jazyky} prázdným zásobníkem (takové jazyky, kde $u\in L(M)\implies \forall w\in X^{\ast}: uw\notin L(M)$) - když se poprvé zásobník automatu vyprázdní, výpočet určitě končí. 

Bezprefixové bezkontextové jazyky jsou vždy deterministické, opačně to neplatí. Deterministický bezkontextový jazyk lze na bezprefixový převést zřetězením s dalším symbolem, který není v původní abecedě. 

Regulární jazyky a bezprefixové bezkontextové jazyky jsou neporovnatelné množiny.
\end{poznamka}

\begin{definiceN}{Nedeterministický Turingův stroj}
\emph{Nedet. Turingův stroj} je pětice $T=(Q,X,\delta,q_0,F)$, kde oproti deterministickým je $\delta:(Q\setminus F)\times X \to \c{P}(Q\times X\times \{-1,0,1\})$. Přijímá slovo $w$, pokud existuje nějaký výpočet $q_{0}w\vdash^{\ast} upv$ tak, že $p\in F$.
\end{definiceN}

\begin{poznamka}
Nedeterministické Turingovy stroje přijímají právě rekurzivně spočetné jazyky, tj. nejsou silnější než deterministické. Výpočty nedet. stroje lze totiž díky nekonečnosti pásky simulovat deterministickým (např. prohledáváním do šířky).
\end{poznamka}

\begin{definiceN}{Lineárně omezený automat}
\emph{Lineárně omezený automat} je nedeterministický Turingův stroj s omezenou páskou (např. symboly $l$ a $r$, které nelze přepsat ani se dostat mimo jejich rozmezí). Slovo je přijímáno, pokud $q_{0}lwr \vdash^{\ast} upv$, kde $p\in F$. Prostor výpočtu je omezen délkou vstupního slova. Lineárně omezené automaty přijímají právě kontextové jazyky.
\end{definiceN}


\begin{poznamkaN}{Rozhodnutelnost}
Turingův stroj může nepřijmout slovo buď skončením výpočtu v nekoncovém stavu, nebo pokud výpočet nikdy neskončí. Turingův stroj \emph{rozhoduje jazyk} $L$, když přijímá právě slova tohoto jazyka a pro libovolné slovo je jeho výpočet konečný. Takové jazyky se nazývají \emph{rekurzivní}.

Problém zastavení výpočtu Turingova stroje je algoritmicky nerozhodnutelný (kvůli možnosti jeho simulace jiným Turingovým strojem). Pro bezkontextové jazyky je algoritmicky rozhodnutelné, zda dané slovo patří do jazyka. Pro bezkontextovou gramatiku nelze algoritmicky rozhodnout, zda $L(G)=X^{\ast}$. Pro dvě kontextové gramatiky je nerozhodnutelné, zda jejich jazyky mají neprázdný průnik.
\end{poznamkaN}


\end{document}
