\subsection{Konkrétní souborové systémy}

\subsubsection*{ext2 (second extended filesystem)}

\begin{pitemize}
  \item Prostor je rozdělen na bloky a ty jsou uspořádány do skupin bloků. To je kvůli
  snížení vnitřní fragmentace a minimalizaci pohyb hlav na disku.
  \item Každá skupina bloků obsahuje: superblock, mapu skupiny bloků, mapu inodů
  a bloky dat.
  \item \textbf{superblock}~-- obsahuje základní (nezbytné) informace o svazku. Jeho kopie
  je tedy v každé skupině bloků 
  \item \textbf{i-node}~-- obsahuje informace o souboru
  \begin{pitemize}
    \item počet linků na soubor
    \item vlastník, skupina
    \item přístupová práva
    \item typ souboru
    \item velikost souboru
    \item časy~-- modifikace, přístup\dots
    \item odkazy na datové bloky
    \item neobsahuje název souboru, ten je uložen v adresáři kam soubor patří
  \end{pitemize}
  \item velkost FS daná při formátování
\end{pitemize}

\subsubsection*{ext3}
\begin{pitemize}
  \item nadstavba na ext2 a tedy možnost přechodu mezi těmito dvěma FS bez
  nutnosti přenášení dat.
  \item hlavní novinka je, že přibyl žurnál. Jsou možné tři úrovně žurnálování:
  \begin{pitemize}
    \item \textbf{Journal}~-- (nejpomalejší, ale nejspolehlivější) metadata i
    obsah souborů jsou zapisovány do žurnálu, ještě před komitnutím do FS.
    \item \textbf{Ordered}~-- (středně rychlý, středně riskantní) 
    \item \textbf{Writeback}~-- (nejrychlejší, ale riskantní - v některých
    ohledech jako ext2) žurnálována jsou pouze metadata, ne již obsah souborů.
  \end{pitemize}
  \item při havárii systému okamžitá oprava podle žurnálu, FS se nemusí kontrolovat
  \item FS roste za běhu jak je potřeba
\end{pitemize}

\subsubsection*{ReiserFS}
\begin{pitemize}
  \item žunálování metadat
  \item růst velikosti FS za běhu, dle potřeby
  \item \uv{tail packing} pro snížení míry fragmentace
  \item metadata, adresářové položky, inody a konce (tail) souborů jsou uloženy
  v jednom \emph{B+ stromě} podle univerzálního ID
  \item mnohdy rychlejší než ext2 (ext3), kde jsou adresáře jako seznamy souborů
  a tak projití adresáře je lineární na rozdíl od logaritmického u stromu v ReiserFS.
\end{pitemize}

\subsubsection*{FAT32 (File Allocatio Table)}
\begin{pitemize}
  \item jednoduchá implementace $\Rightarrow$ téměř na každém OS
  \item žádná ochrana proti fragmentaci
  \item zjistit kolik je volného místa znamená projít lineárně celý disk
  \item rozdělení disku:
  \begin{pitemize}
    \item Boot sector
    \item FAT~-- dvě kopie FAT (mapy datové oblasti)
    \item Datová oblast~-- soubory a adresáře až do konce disku
  \end{pitemize}
\end{pitemize}
