\subsection{Rozhraní BSD Sockets}

\subsubsection*{Úvod}
\textbf{Berkeley (BSD) sockets} je rozhranie (API) na vyvíjanie aplikácií ktoré používajú medziprocesovú komunikáciu (napr. v rámci siete). De facto je to štandardná abstrakcia pre sieťové sockety. Primárnym jazykom tohto API je C, pre väčšinu ostatných však existujú podobné rozhrania.

BSD sockets je API umožňujúce komunikáciu medzi dvomi hostmi alebo procesmi na jednom počítači, používajúc koncepciu internetových socketov. Toto rozhranie je implicitné pre TCP/IP a je teda jednou zo základných technológií internetu. Programátori môžu využívať rozhrania socketov na troch úrovniach, najzákladnejšou z nich sú RAW sockety (aj keď túto úroveň sa využijú zväčša len na počítačoch implementujúcich technológie týkajúce sa už priamo internetu).

\subsubsection*{Hlavičkové súbory}
Berkeley sockets používajú viaceré hlavičkové súbory, okrem iného:
\begin{pitemize}
\item\textbf{sys/socket.h} Core BSD socket functions and data structures.
\item\textbf{netinet/in.h} AF\_INET and AF\_INET6 address families. Widely used on the Internet, these include IP addresses and TCP and UDP port numbers.
\item\textbf{sys/un.h} AF\_UNIX address family. Used for local communication between programs running on the same computer. Not used on networks.
\item\textbf{arpa/inet.h} Functions for manipulating numeric IP addresses.
\item\textbf{netdb.h} Functions for translating protocol names and host names into numeric addresses. Searches local data as well as DNS.
\end{pitemize}

\subsubsection*{TCP}
TCP poskytuje koncept spojenia. Proces vytvorí TCP socket pomocou volania socket() s parametrom PF\_INET(6) a SOCK\_STREAM.

\begin{obecne}{Server}
Vytvorenie jednoduchého TCP servera vyžaduje nasledujúce kroky:
\begin{pitemize}
\item Vytvorenie TCP socketu (pomocou volania \emph{socket()})
\item Pripojenie socketu na port, kde bude načúvať (\emph{bind()}; parametrami je sockaddr\_in štruktúra, v ktorej sa nastavuje sin\_family (AF\_INET-IPv4,\\AF\_INET6-IPv6) a sin\_port)
\item Pripravenie socketu na načúvanie na porte (\emph{listen()}).
\item Akceptovanie príchodzích pripojení pomocou \emph{accept()}. Táto funkcia blokuje volajúceho do príchodu pripojenia a vracia identifikátor príchodzieho spojenia, ktorý sa môže ďalej použiť. accept() je hneď možné volať na pôvodný identifikátor socketu na čakanie na ďalšie spojenia.
\item Komunikácia s klientom pomocou \emph{send()}, \emph{recv()} alebo \emph{read()} a \emph{write()}
\item Keď už socket nie je potrebný, je možné ho zavrieť pomocou \emph{close()}.
\end{pitemize}
\end{obecne}

\begin{obecne}{Klient}
Vytvorenie TCP klienta vyžaduje nasledujúce kroky:
\begin{pitemize}
\item Vytvorenie TCP socketu (pomocou volania \emph{socket()})
\item Pripojenie k serveru pomocou \emph{connect()}) (znovu sa používa štruktúra sockaddr\_in, vypĺňa sa sin\_family, sin\_port (ako pri serveri) + sin\_addr (adresa servera))
\item Komunikácia so serverom pomocou \emph{send()}, \emph{recv()} alebo \emph{read()} a \emph{write()}
\item Keď už socket nie je potrebný, je možné ho zavrieť pomocou \emph{close()}.
\end{pitemize}
\end{obecne}

\subsubsection*{UDP}
UDP je protokol bez spojenia (conectionless) a bez garancie doručenia správ. UDP balíky môžu (okrem správneho počtu/poradia) doraziť mimo poradia, môžu byť duplikované alebo nedoraziť ani raz. Vďaka minimálnym garanciám má UDP oproti TCP oveľa menšiu réžiu. Keďže tento protokol nevytvára spojenia, dáta sa prenášajú v datagramoch.

Adresovací priestor UDP (porty UDP) je úplne nezávislý na priestore portov TCP.

\begin{obecne}{Server}
Keďže sa nevytvárajú spojenia, po vytvorení socketu (ako pri TCP pomocou socket()+bind()) už aplikácia (server) rovno čaká príchodzie datagramy pomocou funkcie \emph{recvfrom()}. Na konci sa socket zatvára pomocou close().
\end{obecne}

\begin{obecne}{Klient}
U klienta je tiež oproti spojovanej verzii zjednodušenie - stačí vyrobiť socket (pomocou socket()) a potom už iba posielať datagramy pomocou \emph{sendto()}. Na konci sa socket zatvára pomocou close().
\end{obecne}

\subsubsection*{Najdôležitejšie funkcie}

\begin{pitemize}

\item \textbf{int socket(int domain, int type, int protocol)}
	\begin{pitemize}
		\item \emph{domain} (PF\_INET | PF\_INET6)
		\item \emph{type} (SOCK\_STREAM, SOCK\_DGRAM,\\SOCK\_SEQPACKET (spoľahlivé zoradené balíky),\\SOCK\_RAW (raw protokoly nad sieťovou vrstvou))
		\item \emph{protocol} (väčšinou IPPROTO\_IP, ďalšie sú v netinet/in.h)
	\end{pitemize}

	\item \textbf{struct hostent *gethostbyname(const char *name)\\
	struct hostent *gethostbyaddr(const void *addr, int len, int type)}
	\begin{pitemize}
		\item Vracia pointer na hostent štruktúru, ktorá popisuje internetového hosta zadaného pomocou mena alebo adresy (obsahuje buď informácie od name servera, alebo z lokálneho /etc/hosts súboru)...
	\end{pitemize}

	\item \textbf{int connect(int sockfd, const struct sockaddr *serv\_addr, socklen\_t addrlen)}
	\item \textbf{int bind(int sockfd, struct sockaddr *my\_addr, socklen\_t addrlen)}
	\item \textbf{int listen(int sockfd, int backlog)}
	\begin{pitemize}
		\item \emph{backlog} určuje maximálne koľko pripojení môže vo fronte čakať na akceptovanie...
	\end{pitemize}

	\item \textbf{int accept(int sockfd, struct sockaddr *cliaddr, socklen\_t *addrlen)}\\
	do \emph{cliaddr} sa vyplnia informácie o klientovi...
\end{pitemize}

\subsubsection*{Blokujúce a neblokujúce volania}
BSD sockety môžu fungovať v dvoch módoch - blokujúcich a neblokujúcich. V blokujúcom móde funkcie nevrátia riadenie programu, kým nie sú spracované všetky dáta - čo môže spôsobiť rôzne problémy (program \uv{zamrzne}, keď socket načúva; alebo keď socket čaká na dáta, ktoré neprichádzajú). Typicky sa nastavuje neblokujúci mód pomocou \emph{fcntl()} alebo \emph{ioctl()}
