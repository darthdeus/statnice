\subsection{Spolehlivost - spojované a nespojované protokoly, typy, detekce a oprava chyb}

\subsubsection*{Spolehlivost}

\textbf{Spolehlivost}:
\begin{pitemize}
	\item může být zajištěna na kterékoliv vrstvě (kromě fyzické)
	\item TCP/IP řeší na transportní (TCP), ISO/OSI očekává spolehlivost na všech (počínaje linkovou)
	\item větši režie, zpoždění při chybách 
\end{pitemize}

\textbf{Nespolehlivá komunikace}:
\begin{pitemize}
	\item menší režie, lepší odezva
	\item výhodné pro audio/video přenosy, kde lze tolerovat ztráty 
\end{pitemize}

\subsubsection*{Spojované a nespojované protokoly}

\textbf{Spojovaná komunikace}: stavová, virtuální okruhy, navazování a ukončení spojení. Viz TCP.

\textbf{Nespojovaná komunikace}: zasílání zpráv, datagramy (UDP), nestavová, bez navazování a ukončování.  Viz UDP.

\subsubsection*{Detekce a oprava chyb}
\begin{pitemize}
	\item schopnost poznat, že došlo k nějaké chybě při přenosu
	\item Hammingovy kódy - příliš velká redundance, nepoužívané
	\item potvrzování (ACK) - viz TCP/IP
		\begin{pitemize}
			\item příjemce si znovu nechá zaslat poškozená/nedoručená data
			\item podmínkou existence zpětného kanálu (alespoň half-duplex)
			\item jednotlivé vs. kontinuální
			\item kladné (ACK) a záporné (NAK)
			\item samostatné vs. nesamostatné (piggybacking)
			\item metoda okénka
			\item selektivní opakování vs. opakování s návratem 
		\end{pitemize}
	\item parita - příčná, podélná
	\item kontrolní součty
	\item cyklické redundantní součy (CRC)
	\item druhy chyb: pozměněná data, shluky chyb, výpadky dat
	\item při chybě nutno vyžádat si celý rámec znovu 
\end{pitemize}