\subsection{Třídy složitosti P a NP, převoditelnost, NP-úplnost}

TODO: doplnit, tohle asi nestaci na pohodovou zkousku (aspon podle toho co jsem vnimal, jsa svedek zkouseni u Dr. Yaghoba) -- chce to napr. nejaky priklad prevedeni jako delal Kucera, mozna jeste neco

\begin{definice}
  \begin{pitemize}
  \item \textbf{Úloha}~-- Pro dané zadání (vstup, \emph{instanci úlohy}) najít výstup s danými vlastnostmi.
  \item \textbf{Optimalizační úloha}~-- Pro dané zadání najít optimální (většinou
  nejmenší nebo největší) výstup s danými vlastnostmi.
  \item \textbf{Rozhodovací problém}~-- Pro dané zadání odpovědět ANO/NE.
  \end{pitemize}
\end{definice}

\begin{definiceN}{třída P}
 \textbf{Třídu složitosti P} (někdy též PTIME) tvoří problémy řešitelné
 sekvenčními deterministickými algoritmy v polynomiálním čase, tj. jejich časová
 složitost je $O(n^k)$. O algoritmech ve třídě P také říkáme, že jsou
 \textbf{efektivně řešitelné.}
\end{definiceN}

\begin{definiceN}{třída NP}
 \textbf{Třída NP} (NPTIME) je třída problémů řešitelných v polynomiálním čase
 sekvenčními nedeterministickými algoritmy. 
\end{definiceN}

\begin{poznamka}
  Ví se, že $P \subseteq NP$. Neví se však, zda $P \neq NP$. Předpokládá se to,
  ale ještě to nikdo nedokázal.
\end{poznamka}

\begin{obecne}{Příklady problémů ze třídy NP}
  \begin{pitemize}
    \item \textbf{KLIKA}(úplný podgraf)~-- Je dán neorientovaný graf $G$ a číslo
    $k$. Existuje v $G$ úplný podgraf velikosti aspoň $k$?
    \item \textbf{HK}(Hamiltonovská kružnice)~-- Je dán neorientovaný graf $G$.
    Existuje v $G$ Hamiltonovská kružnice?
    \item \textbf{SP}(Součet podmnožiny)~-- Jsou dána přirozená čísla $a_1,
    \dots, a_n,b$. Existuje podmnožina čísel $a_1,\dots,a_n$, jejíž součet je
    přesně $b$?
  \end{pitemize}
\end{obecne}

\begin{definiceN}{převody mezi rozhodovacími problémy}
  Nechť $A$, $B$ jsou dva rozhodovací problémy. Říkáme, že $A$ je
  \textbf{polynomiálně redukovatelný (převoditelný)} na $B$, pokud existuje
  zobrazení $f$ z množiny zadání problému $A$ do množiny zadání problému $B$ s
  následujícími vlastnostmi:
  \begin{pitemize}
    \item Nechť $X$ je zadání problému $A$ a $Y$ zadání problému $B$, takové, že
    $f(X)=Y$. Potom je $X$ kladné zadání problému $A$ právě tehdy, když $Y$ je
    kladné zadání problému $B$.
    \item Nechť $X$ je zadání problému $A$. Potom je zadání $f(X)$ problému $B$
    (deterministicky sekvenčně) zkonstruovatelné v polynomiálním čase vzhledem k
    velikosti $X$.
  \end{pitemize}
\end{definiceN}

\begin{definiceN}{NP-těžký problém}
  Problém $B$ je \textbf{NP-těžký}, pokud pro libovolný problém $A$ ze třídy NP
  platí, že $A$ je polynomiálně redukovatelný na $B$.
\end{definiceN}

\begin{definiceN}{NP-úplný problém}
  Problém je \textbf{NP-úplný}, pokud patří do třídy NP a je NP-těžký.
\end{definiceN}

\begin{obecne}{Důsledky}
  \begin{pitemize}
    \item Pokud je $A$ NP-těžký a navíc je $A$ polynomiálně redukovatelný na $B$, tak je
      $B$ taky NP-těžký.
    \item Pokud existuje polynomiální algoritmus pro nějaký NP-těžký problém, pak
      existují polynomiální algoritmy pro všechny problémy ve třídě NP.
  \end{pitemize}
\end{obecne}

\begin{vetaN}{Cook-Levin 1971}
 Existuje NP-úplný problém. (Dokázáno pro SAT)
\end{vetaN}
