\subsection{Neprocedurální programování, logické programování}

\subsubsection*{Neprocedurální programování}
\emph{Deklarativní programování} je postaveno na paradigmatu, podle něhož je program založen na tom, co se počítá a ne jak se to počítá. Je zde deklarován vstup a výstup a celý program je chápán jako funkce vyhodnocující vstupy podávající jediný výstup. Například i webovské stránky jsou deklarativní protože popisují, jak by stránka měla vypadat -- titulek, font, text a obrázky -- ale nepopisují, jak konkrétně zobrazit stránky na obrazovce.

\emph{Logické programování} a \emph{funkcionální programování} jsou poddruhy deklarativního programování. Logické programování využívá programování založené na vyhodnocování vzorů - tvrzení a cílů. Klasickým zástupcem jazyka pro podporu tohoto stylu je Prolog.

Tento přístup patří pod deklarativní programování stejně jako funkcionální programování, neboť deklaruje, co je vstupem a co výstupem, a nezabývá se jak výpočet probíhá. Naopak program jako posloupnost příkazů je paradigma imperativní.

\emph{Funkcionální programování} patří mezi deklarativní programovací principy.

Alonzo Church vytvořil formální výpočtový model nazvaný $\lambda$-kalkul. Tento model slouží jako základ pro funkcionální jazyky. Funkcionální jazyky dělíme na:
\begin{pitemize}
	\item typované - Haskell
	\item netypované - Lisp, Scheme
\end{pitemize}

Výpočtem funkcionálního programu je posloupnost vzájemně ekvivalentních výrazů, které se postupně zjednodušují. Výsledkem výpočtu je výraz v normální formě, tedy dále nezjednodušitelný. Program je chápán jako jedna funkce obsahující vstupní parametry mající jediný výstup. Tato funkce pak může být dále rozložitelná na podfunkce.

\subsubsection*{Prolog}

Prolog je logický programovací jazyk. Název Prolog pochází z francouzského programmation en logique (\uv{logické programování}). Byl vytvořen Alainem Colmerauerem v roce 1972 jako pokus vytvořit programovací jazyk, který by umožňoval vyjadřování v logice místo psaní počítačových instrukcí. Prolog patří mezi tzv. deklarativní programovací jazyky, ve kterých programátor popisuje pouze cíl výpočtu, přičemž přesný postup, jakým se k výsledku program dostane, je ponechán na libovůli systému.

Prolog je využíván především v oboru umělé inteligence a v počítačové lingvistice (obzvláště zpracování přirozeného jazyka, pro nějž byl původně navržen). Syntaxe jazyka je velice jednoduchá a snadno použitelná pravě proto, že byl původně určen pro počítačově nepříliš gramotné lingvisty.

Prolog je založen na \emph{predikátové logice prvního řádu} (konkrétně se omezuje na Hornovy klauzule). Běh programu je pak představován aplikací dokazovacích technik na zadané klauzule. Základními využívanými přístupy jsou \emph{unifikace}, \emph{rekurze} a \emph{backtracking}.

Interpret Prologu se snaží nalézt nejobecnější substituci, která splní daný cíl - tzn. nesubstituuje zbytečně, pokud nemusí (použití interních proměnných -- \_123 atd.). Za dvě proměnné může být substituována jedna interní proměnná (např. při hledání svislé úsečky -- konstantní X souřadnice) -- tomu se říká \emph{unifikace} proměnných. Pro proměnnou, jejíž hodnota může být libovolná, se v prologu užívá znak \uv{\_}. 

Datové typy v prologu se nazývají \emph{termy}. Základním datovým typem jsou \emph{atomy} (začínají malým písmenem, nebo se skládají ze speciálních znaků (\texttt{+ - * / \dots}, nebo jsou to znakové řetězce (\texttt{'text'})). Dále \uv{jsou} v prologu čísla (v komerčních implementacích i reálná), proměnné (velké písmeno) a struktury (definované rekursivně - pomocí funktoru dané arity a příslušným počtem termů, které jsou jeho argumenty -- \texttt{okamzik(datum(1,1,1999),cas(10,10))}). Posledním typem proměnných jsou seznamy, které jsou probírány později.

\medskip\textbf{Základní principy}:

Programování v Prologu se výrazně liší od programování v běžných procedurálních jazycích jako např. C. Program v prologu je databáze faktů a pravidel (dohromady se faktům a pravidlům paradoxně říká procedury), nad kterými je možno klást dotazy formou tvrzení, u kterých Prolog zhodnocuje jejich pravdivost (dokazatelnost z údajů obsažených v databázi).

Například lze do databáze uložit fakt, že Monika je dívka:
\begin{verbatim}
dívka(monika).
\end{verbatim}

Poté lze dokazatelnost tohoto faktu prověřit otázkou, na kterou Prolog odpoví yes (ano):
\begin{verbatim}
?- dívka(monika).
     yes.
\end{verbatim}

Také se lze zeptat na všechny objekty, o kterých je známo, že jsou dívky (středníkem požadujeme další výsledky):
\begin{verbatim}
?- dívka(X).
     X = monika;
     no.
\end{verbatim}

Pravidla (závislosti) se zapisují pomocí implikací, např.
\begin{verbatim}
syn(A,B) :- rodič(B,A), muž(A).
\end{verbatim}

Tedy: pokud B je rodičem A a zároveň je A muž, pak A je synem B. První části pravidla (tj. důsledku) se říká hlava a všemu co následuje za symbolem \texttt{:-} (tedy podmínkám, nutným pro splnění hlavy) se říká tělo. Podmínky ke splnění mohou být odděleny buď čárkou (pak jde o konjunkci, musejí být splněny všechny), nebo středníkem (disjunkce), přičemž čárky mají větší prioritu.

\medskip\textbf{Příklad}:

Typickou ukázkou základů programování v Prologu jsou rodinné vztahy.
\begin{verbatim}
sourozenec(X,Y) :- rodič(Z,X), rodič(Z,Y).
rodič(X,Y) :- otec(X,Y).
rodič(X,Y) :- matka(X,Y).
muž(X) :- otec(X,_).
žena(X) :- matka(X,_).
matka(marie,monika).
otec(jiří,monika).
otec(jiří,marek).
otec(michal,tomáš).
\end{verbatim}

Prázdný seznam je označen atomem $[]$, neprázdný se tvoří pomocí funktoru \texttt{'.'} (tečka) - \texttt{.(Hlava,Tělo)}. V praxi se to (naštěstí ;-)) takhle složitě rekurzivně zapisovat nemusí, stačí napsat \texttt{[a,b,c...]}, resp \texttt{[Začátek | Tělo]}, kde začátek je výčet prvků (ne seznam) stojících na začátku definovaného seznamu, a tělo je (rekurzivně) seznam (např. \texttt{[a,b,c|[]]}).

Aritmetické výrazy se samy o sobě nevyhodnocují, dokud jim to někdo nepřikáže. Takže např. predikát \texttt{5*1 = 5} by selhal. Vyhodnocení se vynucuje pomocí operátoru \emph{is} (pomocí = by došlo jen k unifikaci) - není to ale ekvivalent \uv{\texttt{=}} z jiných jazyků. Tento operátor se musí použít na nějakou volnou proměnnou a aritmetický výraz, s jehož hodnotou bude tato proměnná dále svázaná (jako např. \texttt{X is 5*1,X=5} uspěje).

Důležitý je i \emph{operátor řezu} (značíme vykřičníkem). Tento predikát okamžitě uspěje, ale při tom zakáže backtrackování přes sebe zpět. (\texttt{prvek1(X,[X|L]):-!. prvek1(X,[\_|L]):-prvek1(X,L).} - je-li prvek nalezen, je zakázán návrat = najde jen první výskyt prvku). Dále je důležitá negace (\texttt{not(P):- P, !, fail. not(P).} -- uspěje, pokud se nepodaří cíl P splnit). Řez tedy umožňuje ovlivňovat efektivitu prologovských programů, definovat vzájemně se vylučující použití jednotlivých klauzulí procedury, definovat negaci atd.

\subsubsection*{Haskell}
Haskell je standardizovaný funkcionální programovací jazyk používající zkrácené vyhodnocování, pojmenovaný na počest logika Haskella Curryho. Byl vytvořen v 80. letech 20. století. Posledním polooficiálním standardem je Haskell 98, který definuje minimální a přenositelnou verzi jazyka využitelnou k výuce nebo jako základ dalších rozšíření. Jazyk se rychle vyvíjí, především díky svým implementacím Hugs a GHC (viz níže).

Haskell je jazyk dodržující \emph{referenční transparentnost}. To, zjednodušeně řečeno, znamená, že tentýž (pod)výraz má na jakémkoliv místě v programu stejnou hodnotu. Mezi další výhody tohoto jazyka patří přísné \emph{typování proměnných}, které programátorovi může usnadnit odhalování chyb v programu. Haskell plně podporuje práci se soubory i standardními vstupy a výstupy, která je ale poměrně složitá kvůli zachování referenční transparentnosti. Jako takový se Haskell hodí hlavně pro algoritmicky náročné úlohy minimalizující interakci s uživatelem.

\medskip\textbf{Příklady}:

Definice funkce faktoriálu:
\begin{verbatim}
fac 0 = 1
fac n = n * fac (n - 1)
\end{verbatim}

Jiná definice faktoriálu (používá funkci product ze standardní knihovny Haskellu):
\begin{verbatim}
fac n = product [1..n]
\end{verbatim}

Naivní implementace funkce vracející n-tý prvek Fibonacciho posloupnosti:
\begin{verbatim}
fib 0 = 0 
fib 1 = 1 
fib n = fib (n - 2) + fib (n - 1)
\end{verbatim}

Elegantní zápis řadícího algoritmu quicksort:
\begin{verbatim}
qsort [] = []
qsort (pivot:tail) = 
  qsort left ++ [pivot] ++ qsort right
  where
    left = [y | y <- tail, y < pivot]
    right = [y | y <- tail, y >= pivot]
\end{verbatim}

TODO: popsat stráže (případy, otherwise), seznamy, řetězení, pattern matching u parametrů funkcí, lok. definice (where, let) -- patří to sem?

\subsubsection*{Lisp}

Lisp je funkcionální programovací jazyk s dlouhou historií. Jeho název je zkratka pro List processing (zpracování seznamů). Dnes se stále používá v oboru umělé inteligence. Nic ale nebrání ho použít i pro jiné účely. Používá ho například textový editor Emacs, GIMP či konstrukční program AutoCAD.

Další jazyky od něj odvozené jsou například Tcl, Smalltalk nebo Scheme.


\medskip\textbf{Syntaxe}:
Nejzákladnějším zápisem v Lispu je seznam. Zapisujeme ho jako:
\begin{verbatim}
(1 2 "ahoj" 13.2)
\end{verbatim}

Tento seznam obsahuje čtyři prvky:
\begin{pitemize}
	\item celé číslo 1
	\item celé číslo 2
	\item text \uv{ahoj}
	\item reálné číslo 13,2
\end{pitemize}

Jde tedy o uspořádanou čtveřici. Všimněte si, že závorky nefungují tak jako v matematice, ale pouze označují začátek a konec seznamu. Seznamy jsou v Lispu implementovány jako binární strom degenerovaný na jednosměrně vázaný seznam. Co se seznamem Lisp udělá, záleží na okolnostech.

\textbf{Příkazy}: Příkazy píšeme také jako seznam, první prvek seznamu je však název příkazu. Například sčítání provádíme příkazem +, což interpreteru zadáme takto:
\begin{verbatim}
(+ 1 2 3)
\end{verbatim}
Interpretr odpoví 6.

\textbf{Ukázka kódu}:
Program hello world lze zapsat několika způsoby. Nejjednoduší vypadá takto:
\begin{verbatim}
(format t "Hello, World!")
\end{verbatim}

Funkce se v Lispu definují pomocí klíčového slova defun:
\begin{verbatim}
(defun hello ()
        (format t "Hello, World!")
)
(hello)
\end{verbatim}

Na prvních dvou řádcích je definice funkce hello, na třetím řádku je tato funkce svým jménem zavolána.
Funkcím lze předávat i argumenty. V následujícím příkladu je ukázka funkce fact, která vypočítá faktoriál zadaného čísla:
\begin{verbatim}
(defun fact (n)
        (if (= n 0)
                1
                (* n (fact (- n 1)))
        )
)
\end{verbatim}
Pro výpočet faktoriálu čísla 6 předáme tuto hodnotu jako argument funkci fact:
\begin{verbatim}
(fact 6)
\end{verbatim}
Návratovou hodnotou funkce bude hodnota 720.

\subsubsection*{Logické programování}
TODO (není součástí otázek pro obor Programování)
