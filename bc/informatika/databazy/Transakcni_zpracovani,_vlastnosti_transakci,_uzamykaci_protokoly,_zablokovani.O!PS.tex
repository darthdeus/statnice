\subsection{Transakční zpracování, vlastnosti transakcí, uzamykací protokoly, zablokování}

\begin{definiceN}{Transakce}
\emph{Transakce} je jistá posloupnost nebo specifikace posloupnosti akcí práce s databází, jako
jsou čtení, zápis nebo výpočet, se kterou se zachází jako s jedním celkem.
\end{definiceN}

Hlavním smyslem používání transakcí, tj. \emph{transakčního zpracování}, je
udržení databáze v konzistentním stavu. Jestliže na sobě některé operace závisí,
sdružíme je do jedné transakce a tím zabezpečíme, že budou vykonány buď
všechny, nebo žádná. Databáze tak před i po vykonání transakce bude v
konzistentním stavu. Aby se uživateli transakce jevila jako jedna atomická
operace, je nutné zavést příkazy COMMIT a ROLLBACK. První z nich signalizuje
databázi úspěšnost provedení transakce, tj. veškeré změny v databázi se stanou
trvalými a jsou zviditelněny pro ostatní transakce, druhý příkaz signalizuje
opak, tj. databáze musí být uvedena do původního stavu.

Tyto příkazy většinou není nutné volat explicitně, např. příkaz COMMIT je vyvolán po
normálním ukončení programu realizujícího transakci. Příkaz ROLLBACK pro svou
funkci vyžaduje použití tzv. \emph{žurnálu} (logu) na nějakém stabilním
paměťovém médiu. Žurnál obsahuje historii všech změn databáze v jisté časové
periodě.

Jednoduchá transakce vypadá většinou takto:
\begin{penumerate}
  \item Začátek transakce,
  \item provedení několika dotazů -- čtení a zápisů (žádné změny v databázi nejsou zatím vidět pro
  okolní svět),
  \item Potvrzení (příkaz COMMIT) transakce (pokud se transakce povedla, změny
  v databázi se stanou viditelné).
\end{penumerate}
Pokud nějaký z provedených dotazů selže, systém by měl celou transakci zrušit a
vrátit databázi do stavu v jakém byla před zahájením transakce (operace ROLLBACK).

Transakční zpracování je také ochrana databáze před hardwarovými nebo
softwarovými chybami, které mohou zanechat databázi po částečném zpracování
transakce v nekonzistentním stavu. Pokud počítač selže uprostřed provádění
některé transakce, transakční zpracování zaručí, že všechny operace z
nepotvrzených (\uv{uncommitted}) transakcí budou zrušeny. 

\subsubsection*{Vlastnosti transakcí}

Podívejme se nyní na vlastnosti požadované po transakcích. Obvykle se používá
zkratka prvních písmen anglických názvů vlastností \textbf{ACID}~-- atomicity,
consistency, isolation (independence), durability. 
\begin{description}
  \item[atomicita] -- transakce se tváří jako jeden celek, musí buď proběhnout
  celá, nebo vůbec ne.
  \item[konzistence] -- transakce transformuje databázi z jednoho konzistentního
  stavu do jiného konzistentního stavu.
  \item[nezávislost] -- transakce jsou nezávislé, tj. dílčí efekty transakce
  nejsou viditelné jiným transakcím.
  \item[trvanlivost] -- efekty úspěšně ukončené (potvrzené,\uv{commited})
  transakce jsou nevratně uloženy do databáze a nemohou být zrušeny.
\end{description}

Transakce mohou být v uživatelských programech prováděny paralelně (spíše
zdánlivě paralelně, stejně jako je paralelismus multitaskingu na jednoprocesorových
strojích jen zdánlivý, zajistí to ale možnost paralelizace \uv{nedatabázových} 
akcí a pomalé transakce nebrzdí rychlé). Je
zřejmé, že posloupnost transakcí může být zpracována paralelně různým způsobem.
Každá transakce se skládá z několika akcí. Stanovené pořadí provádění akcí
více transakcí v čase nazveme \textbf{rozvrhem}.

Rozvrh, který splňuje následující podmínky, budeme nazývat \textbf{legální}:
\begin{pitemize}
  \item Objekt je nutné mít uzamknutý, pokud k němu chce transakce přistupovat.
  \item Transakce se nebude pokoušet uzamknout objekt již uzamknutý jinou
  transakcí (nebo musí počkat, než bude objekt odemknut).
\end{pitemize}

Důležitými pojmy pro paralelní zpracování jsou sériovost či uspořádatelnost.
\textbf{Sériové rozvrhy} zachovávají operace každé transakce pohromadě (a 
provádí se jen jedna transakce najednou). Pro $n$
transakcí tedy existuje $n!$ různých sériových rozvrhů. Pro získání korektního
výsledku však můžeme použít i rozvrhu, kde jsou operace různých transakcí
navzájem prokládány.
Přirozeným požadavkem na korektnost je, aby efekt paralelního zpracování
transakcí byl týž, jako kdyby transakce byly provedeny v nějakém sériovém rozvrhu.
Předpokládáme-li totiž, že každá transakce je korektní program, měl by vést
výsledek sériového zpracování ke konzistentnímu stavu. O systému zpracování
transakcí, který zaručuje dosažení konzistentního stavu nebo stejného stavu
jako sériové rozvrhy, se říká, že zaručuje \textbf{uspořádatelnost}.

Mohou se vyskytnout problémy, které uspořádatelnosti zamezují. Ty nazýváme \emph{konflikty}. Plynou z pořadí dvojic akcí různých transakcí na stejném objektu. Existují tři typy konfliktních situací:
\begin{penumerate}
    \item WRITE-WRITE -- přepsání nepotvrzených dat
    \item READ-WRITE -- neopakovatelné čtení
    \item WRITE-READ -- čtení nepotvrzených (\uv{uncommitted}) dat
\end{penumerate}

Řekneme, že rozvrh je \emph{konfliktově uspořádatelný}, je-li konfliktově ekvivalentní nějakému sériovému rozvrhu (tedy jsou v něm stejné, tj. žádné konflikty). Test na konfliktovou uspořádatelnost se dá provést jako test acykličnosti grafu, ve kterém konfliktní situace představují hrany a transakce vrcholy. Konfliktová uspořádatelnost je slabší podmínka než uspořádatelnost -- nezohledňuje ROLLBACK (\emph{zotavitelnost} -- zachování konzistence, i když kterákoliv transakce selže) a dynamickou povahu databáze (vkládání a mazání objektů). Zotavitelnosti se dá dosáhnout tak, že každá transakce $T$ je potvrzena až poté, co jsou potvrzeny všechny ostatní transakce, které změnily data čtená v $T$. Pokud v zotavitelném rozvrhu dochází ke čtení změn pouze potvrzených transakcí, nemůže dojít ani k jejich \emph{kaskádovému rušení}.

Při zpracování (i uspořádatelného) rozvrhu může dojít k situaci \emph{uváznutí} -- \emph{deadlocku}. To nastane tehdy, pokud jedna transakce $T_1$ čeká na zámek na objekt, který má přidělený $T_2$ a naopak. Situaci lze zobecnit i na více transakcí. Uváznutí lze buď přímo zamezit charakterem rozvrhu, nebo detekovat (hledáním cyklu v grafu čekajících transakcí, tzv. \uv{waits-for} grafu) a jednu z transakcí \uv{zabít} a spustit znova.

\medskip
K zajištění uspořádatelnosti a zotavitelnosti a zabezpečení proti kaskádovým rollbackům a deadlocku se používají různá schémata (požadavky na rozvrhy). Jedním z nich jsou uzamykací protokoly.

\subsubsection*{Uzamykací protokoly}

Vytváření rozvrhů a testování jejich uspořádatelnosti není pro praxi zřejmě ten
nejvhodnější způsob. Pokud ale budeme transakce konstruovat podle určitých
pravidel, tak za určitých předpokladů bude každý jejich rozvrh uspořádatelný.
Soustavě takových pravidel se říká \textbf{protokol}.

Nejznámější protokoly jsou založeny na dynamickém zamykání a odemykání objektů v
databázi. Zamykání (operace LOCK) je akce, kterou vyvolá transakce na objektu,
aby ho chránila před přístupem ostatních transakcí.

\begin{definiceN}{Dobře formovaná transakce}
Transakci nazveme \textbf{dobře formovanou} pokud podporuje přirozené požadavky
na transakce:
\begin{penumerate}
  \item transakce zamyká objekt, chce-li k němu přistupovat,
  \item transakce nezamyká objekt, který již je touto transakcí uzamčený,
  \item transakce neodmyká objekt, který není touto transakcí zamčený,
  \item po ukončení transakce jsou všechny objekty uzamčené touto transakcí
  odemčeny.
\end{penumerate}
\end{definiceN}

\paragraph{Dvoufázový protokol (2PL)} -- Dvoufázová transakce v první fázi
zamyká vše co je potřeba a od prvního odemknutí (druhá fáze) již jen odemyká co
měla zamčeno (již žádná operace LOCK). Tedy transakce musí mít všechny objekty
uzamčeny předtím, než nějaký objekt odemkne. Dá se dokázat, že pokud jsou
všechny transakce v dané množině transakcí dobře formované a dvoufázové, pak
každý jejich legální rozvrh je uspořádatelný.

Dvoufázový protokol zajišťuje uspořádatelnost, ale ne zotavitelnost ani
bezpečnost proti kaskádovému rušení transakcí nebo uváznutí.

\paragraph{Striktní dvoufázový protokol (S2PL)} -- Problémy 2PL jsou nezotavitelnost
a kaskádové rušení transakcí. Tyto nedostatky lze odstranit pomocí striktních
dvoufázových protokolů, které uvolňují zámky až po skončení transakce (COMMIT).
Zřejmá nevýhoda je omezení paralelismu. 2PL navíc stále nevylučuje možnost deadlocku.

\paragraph{Konzervativní dvoufázový protokol (C2PL)} -- Rozdíl oproti 2PL je
ten, že transakce žádá o všechny své zámky, ještě než se začne
vykonávat. To sice vede občas k zbytečnému zamykání (nevíme co přesně budeme
potřebovat, tak radši zamkneme víc), ale stačí to již k prevenci uváznutí
(deadlocku).

\subsubsection*{\uv{Vylepšení} zamykacích protokolů}

\paragraph{Sdílené a výlučné zámky} -- Nevýhodou 2PL je, že objekt může mít
uzamčený pouze jedna transakce. Abychom uzamykání provedli precizněji, je dobré
vzít na vědomí rozdíl mezi operacemi READ a WRITE. \emph{Výlučný zámek}
(W\_LOCK) může být aplikován na objekty jak pro operaci READ tak pro WRITE,
\emph{sdílený zámek} (R\_LOCK) uzamyká objekt, který chceme pouze číst. Jeden
objekt potom může být uzamčen sdíleným zámkem více transakcí a zvyšuje se tak
možnost paralelního zpracování. Budeme-li s těmito zámky zacházet stejně jako u
2PL, opět máme zaručenou uspořádatelnost rozvrhu, ovšem nikoliv absenci uváznutí.


\paragraph{Strukturované uzamykání (multiple granularity)} -- Objekty jsou v
tomto případě chápány hierarchicky dle relace \emph{obsahuje}. Například
databáze obsahuje soubory, které obsahují stránky a ty zase obsahují jednotlivé
záznamy. Na tuto hierarchii se můžeme dívat jako na strom, ve kterém každý
vrchol obsahuje své potomky. Když transakce zamyká objekt (vrchol) zamyká také
všechny jeho potomky. Protokol se tak snaží minimalizovat počet zámků, tím
snížit režii a zvýšit možnosti paralelního zpracování.


\subsubsection*{Alternativní protokoly}

\paragraph{Časová razítka} -- Další z protokolů zaručující uspořádatelnost je
využití časových razítek. Na začátku dostane transakce $T$ \emph{časové
razítko}~-- $TS(T)$ (časová razítka jsou unikátní a v čase rostou), abychom věděli
pořadí, ve kterém by měli být transakce vykonány. Každý objekt v databázi má
\emph{čtecí razítko}~-- $RTS(O)$ (read timestamp), které je aktualizováno, když je
objekt čten, a \emph{zapisovací razítko}~-- $WTS(O)$ (write timestamp), které je
aktualizováno, když nějaká transakce objekt mění.

Pokud chce transakce $T$ číst objekt $O$ mohou nastat dva případy:
\begin{pitemize}

  \item $TS(T) < WTS(O)$, tzn. někdo změnil objekt $O$ potom co byla spuštěna
  transakce $T$. V tomto případě musí být transakce zrušena a spouštěna znovu (a
  tedy s jiným časovým razítkem).

  \item $TS(T) > WTS(O)$, tzn. je bezpečné objekt číst. V tomto případě $T$
  přečte $O$ a $RTS(O)$ je nastaveno na $\max\{TS(T),\ RTS(O)\}$.

\end{pitemize}

Pokud chce transakce $T$ zapisovat do objektu $O$ rozlišujeme případy tři:
\begin{pitemize}

  \item $TS(T) < RTS(O)$, tzn. někdo četl $O$ poté co byla spuštěna $T$ a
  předpokládáme, že si pořídil lokální kopii. Nemůžeme tedy $O$ změnit, protože
  by lokální kopie přestala být platná a tedy je nutné $T$ zrušit a spustit
  znova.

  \item $TS(T) < WTS(O)$, tzn. někdo změnil $O$ po startu $T$. V tomto případě
  přeskočíme write operaci a pokračujeme dále normálně. $T$ nemusí být
  restartována.

  \item V ostatních případech $T$ změní $O$ a $WTS(O)$ je nastaveno na $TS(T)$.
\end{pitemize}

\paragraph{Optimistické protokoly} -- V situaci kdy se většina transakcí
neovlivňuje, je režie výše uvedených protokolů zbytečně velká a můžeme použít
takzvaný optimistický protokol. V protokolu můžeme rozlišit tři fáze.
\begin{penumerate}

  \item \textbf{Fáze čtení:} Čtou se objekty z databáze do lokální paměti a jsou
  na nich prováděny potřebné změny.

  \item \textbf{Fáze kontroly:} Po dokončení všech změn v lokální paměti je
  vyvolán pokus o zapsání výsledků do databáze. Algoritmus zkontroluje, zda
  nehrozí potenciální kolize s již potvrzenými transakcemi, nebo s některými
  právě probíhajícími. Pokud konflikt existuje, je třeba spustit algoritmus pro
  řešení kolizí, který se je snaží vyřešit. Pokud se mu to nepodaří, je využita
  poslední možnost a tou je zrušení a restartování transakce.

  \item \textbf{Fáze zápisu:} Pokud nehrozí žádné konflikty, jsou data z lokální
  paměti zapsány do databáze a transakce potvrzena.

\end{penumerate}


