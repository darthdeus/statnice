\clearpage \documentclass[a4paper,12pt,notitlepage]{article}

\frenchspacing
\usepackage{a4}
\usepackage[pdftitle={Vypracovane otazky k bakalarskym statnicim}, pdfauthor={študenti MFF}, pdfdisplaydoctitle=true, colorlinks=false,unicode=true,pdfborder=0 0 0]{hyperref}
\usepackage{slovak}
\usepackage{ucs}
\usepackage[utf8x]{inputenc}

\title{Vypracovane otazky k bakalarskym statnicim}
\author{študenti MFF}

\usepackage{graphicx}
\usepackage{amsmath,amssymb,amsthm}
\usepackage{color}
\usepackage[left=3cm, right=3cm, top=3cm, bottom=3cm]{geometry} % nastavení dané velikosti okrajů


%Vacsina prostredi je dvojjazicne. V pripade, ze znenie napr pozorovania je pisane po slovensky, malo by byt po slovensky aj oznacenie.

\newenvironment{pozadavky}{\pagebreak[2]\noindent\textbf{Požadavky}\par\noindent\leftskip 10pt}{\par\bigskip}
\newenvironment{poziadavky}{\pagebreak[2]\noindent\textbf{Požiadavky}\par\noindent\leftskip 10pt}{\par\bigskip}

\newenvironment{definice}{\pagebreak[2]\noindent\textbf{Definice}\par\noindent\leftskip 10pt}{\par\bigskip}
\newenvironment{definiceN}[1]{\pagebreak[2]\noindent\textbf{Definice~}\emph{(#1)}\par\noindent\leftskip 10pt}{\par\bigskip}
\newenvironment{definicia}{\pagebreak[2]\noindent\textbf{Definícia}\par \noindent\leftskip 10pt}{\par\bigskip}
\newenvironment{definiciaN}[1]{\pagebreak[2]\noindent\textbf{Definícia~}\emph{(#1)}\par\noindent\leftskip 10pt}{\par\bigskip}

\newenvironment{pozorovani}{\pagebreak[2]\noindent\textbf{Pozorování}\par\noindent\leftskip 10pt}{\par\bigskip}
\newenvironment{pozorovanie}{\pagebreak[2]\noindent\textbf{Pozorovanie}\par\noindent\leftskip 10pt}{\par\bigskip}
\newenvironment{poznamka}{\pagebreak[2]\noindent\textbf{Poznámka}\par\noindent\leftskip 10pt}{\par\bigskip}
\newenvironment{poznamkaN}[1]{\pagebreak[2]\noindent\textbf{Poznámka~}\emph{(#1)}\par\noindent\leftskip 10pt}{\par\bigskip}
\newenvironment{lemma}{\pagebreak[2]\noindent\textbf{Lemma}\par\noindent\leftskip 10pt}{\par\bigskip}
\newenvironment{lemmaN}[1]{\pagebreak[2]\noindent\textbf{Lemma~}\emph{(#1)}\par\noindent\leftskip 10pt}{\par\bigskip}
\newenvironment{veta}{\pagebreak[2]\noindent\textbf{Věta}\par\noindent\leftskip 10pt}{\par\bigskip}
\newenvironment{vetaN}[1]{\pagebreak[2]\noindent\textbf{Věta~}\emph{(#1)}\par\noindent\leftskip 10pt}{\par\bigskip}
\newenvironment{vetaSK}{\pagebreak[2]\noindent\textbf{Veta}\par\noindent\leftskip 10pt}{\par\bigskip}
\newenvironment{vetaSKN}[1]{\pagebreak[2]\noindent\textbf{Veta~}\emph{(#1)}\par\noindent\leftskip 10pt}{\par\bigskip}

\newenvironment{dusledek}{\pagebreak[2]\noindent\textbf{Důsledek}\par\noindent\leftskip 10pt}{\par\bigskip}
\newenvironment{dosledok}{\pagebreak[2]\noindent\textbf{Dôsledok}\par\noindent\leftskip 10pt}{\par\bigskip}

\newenvironment{dokaz}{\pagebreak[2]\noindent\leftskip 10pt\textbf{Dôkaz}\par\noindent\leftskip 10pt}{\par\bigskip}
\newenvironment{dukaz}{\pagebreak[2]\noindent\leftskip 10pt\textbf{Důkaz}\par\noindent\leftskip 10pt}{\par\bigskip}

\newenvironment{priklad}{\pagebreak[2]\noindent\textbf{Příklad}\par\noindent\leftskip 10pt}{\par\bigskip}
\newenvironment{prikladSK}{\pagebreak[2]\noindent\textbf{Príklad}\par\noindent\leftskip 10pt}{\par\bigskip}
\newenvironment{priklady}{\pagebreak[2]\noindent\textbf{Příklady}\par\noindent\leftskip 10pt}{\par\bigskip}
\newenvironment{prikladySK}{\pagebreak[2]\noindent\textbf{Príklady}\par\noindent\leftskip 10pt}{\par\bigskip}

\newenvironment{algoritmusN}[1]{\pagebreak[2]\noindent\textbf{Algoritmus~}\emph{(#1)}\par\noindent\leftskip 10pt}{\par\bigskip}
%obecne prostredie, ktore ma vyuzitie pri specialnych odstavcoch ako (uloha, algoritmus...) aby nevzniklo dalsich x prostredi
\newenvironment{obecne}[1]{\pagebreak[2]\noindent\textbf{#1}\par\noindent\leftskip 10pt}{\par\bigskip}


\newenvironment{penumerate}{
\begin{enumerate}
  \setlength{\itemsep}{1pt}
  \setlength{\parskip}{0pt}
  \setlength{\parsep}{0pt}
  %\setlength{\topsep}{200pt}
  \setlength{\partopsep}{200pt}
}{\end{enumerate}}

\def\pismenka{\numberedlistdepth=2} %pouzit, ked clovek chce opismenkovany zoznam...

\newenvironment{pitemize}{
\begin{itemize}
  \setlength{\itemsep}{1pt}
  \setlength{\parskip}{0pt}
  \setlength{\parsep}{0pt}
}{\end{itemize}}

\definecolor{gris}{gray}{0.95}
\newcommand{\ramcek}[2]{\begin{center}\fcolorbox{white}{gris}{\parbox{#1}{#2}}\end{center}\par}
 \clearpage
\title{\LARGE Učební texty k státní bakalářské zkoušce \\ Obecná informatika \\ Logika}
\begin{document}
\maketitle
\newpage
\setcounter{section}{0}
\section{Logika}
\begin{pozadavky}
\begin{pitemize}
\item Jazyk, formule, sémantika, tautologie.
\item Rozhodnutelnost, splnitelnost, pravdivost, dokazatelnost.
\item Věty o kompaktnosti a úplnosti výrokové a predikátové logiky.
\item Normální tvary výrokových formulí, prenexní tvary formulí predikátové logiky.
\end{pitemize}
\end{pozadavky}
\def\c#1{\mathcal{#1}}


\subsection{Logika -- jazyk, formule, sémantika, tautologie}

\subsubsection*{Jazyk}

\begin{obecne}{Logika prvního řádu}
Formální systém logiky prvního řádu obsahuje jazyk, axiomy, odvozovací pravidla, věty a důkazy. Jazyk prvního řádu zahrnuje:
\begin{pitemize}
    \item neomezeně mnoho proměnných $x_1,x_2,\dots $
    \item funkční symboly $f_1,f_2\dots $, každý má aritu  $n\geq 0$
    \item predikátové symboly $p_1,p_2\dots $, každý má aritu
    \item symboly pro logické spojky ($\neg,\vee,\&,\rightarrow,\leftrightarrow$)
    \item symboly pro kvantifikátory ($\forall,\exists$)
    \item může (ale nemusí) obsahovat binární predikát $"="$, který pak se pak ale musí chovat jako rovnost, tj. splňovat určité axiomy (někdy se potom proto řadí mezi logické symboly).
\end{pitemize}
Proměnné, logické spojky a kvantifikátory jsou \emph{logické symboly}, ostatní symboly se nazývají \emph{speciální}.
\end{obecne}

\begin{definiceN}{Jazyk výrokové logiky}
Jazyk výrokové logiky je jazyk prvního řádu, obsahující \emph{výrokové proměnné} (\uv{prvotní formule}), \emph{logické spojky} $\neg,\vee,\&,\rightarrow,\leftrightarrow$ a \emph{pomocné symboly} (závorky).
\end{definiceN}

\begin{definiceN}{Jazyk predikátové logiky}
Jazyk predikátové logiky je jazyk prvního řádu, obsahující proměnné, \emph{predikátové symboly} (s nenulovou aritou), \emph{funkční symboly} (mohou mít nulovou aritu), symboly pro logické spojky a symboly pro \emph{kvantifikátory}.
\end{definiceN}

\subsubsection*{Formule}

\begin{definiceN}{Formule výrokové logiky}
Pro jazyk výrokové logiky jsou následující výrazy formule:
\begin{penumerate}
    \item každá výroková proměnná
    \item pro formule $A,B$ i výrazy $\neg A$, $(A\vee B)$, $(A\& B)$, $(A\rightarrow B)$,
	$A\leftrightarrow B$
    \item každý výraz vzniknuvší konečným užitím pravidel 1. a 2.
\end{penumerate}
Množina formulí se nazývá \emph{teorie}.
\end{definiceN}

\begin{definiceN}{Term}
V predikátové logice je \emph{term}:
\begin{penumerate}
    \item každá proměnná
    \item výraz $f(t_1,\dots,t_n)$ pro $f$ $n$-ární funkční symbol a $t_1,\dots,t_n$ termy
    \item každý výraz vzniknuvší konečným užitím pravidel 1. a 2.
\end{penumerate}
Podslovo termu, které je samo o sobě term, se nazývá \emph{podterm}.
\end{definiceN}

\begin{definiceN}{Formule predikátové logiky}
V predikátové logice je formule každý výraz tvaru $p(t_1,\dots,t_n)$ pro $p$ predikátový symbol a $t_1,\dots,t_n$ termy. Stejně jako ve výrokové logice je formule i (konečné) spojení jednodušších formulí log. spojkami.
Formule jsou navíc i výrazy $(\exists x)A$ a $(\forall x)A$ pro formuli $A$ a samozřejmě cokoliv, co vznikne konečným užitím těchto pravidel.

Podslovo formule, které je samo o sobě formule, se nazývá \emph{podformule}.
\end{definiceN}

\begin{definiceN}{Volné a vázané proměnné}
Výskyt proměnné $x$ ve formuli je \emph{vázaný}, je-li tato součástí nějaké podformule tvaru $(\exists x)A$ nebo $(\forall x)A$. V opačném případě je \emph{volný}. Formule je \emph{otevřená}, pokud neobsahuje vázanou proměnnou, je \emph{uzavřená}, když neobsahuje volnou proměnnou. Proměnná může být v téže formuli volná i vázaná (např. $(x=z)\rightarrow(\exists x)(x=z)$).
\end{definiceN}

\subsubsection*{Sémantika}

\begin{definiceN}{Pravdivostní ohodnocení ve výrokové logice}
Výrokové proměnné samotné neanalyzujeme -- jejich hodnoty máme dány už z vnějšku, máme pro ně \emph{množinu pravdivostních hodnot} ($\{0,1\}$).

\emph{Pravdivostní ohodnocení} $v$ je zobrazení, které každé výrokové proměnné přiřadí právě jednu hodnotu z množiny pravdivostních hodnot. Je-li známo ohodnocení proměnných, lze určit \emph{pravdivostní hodnotu} $\overline{v}$ pro každou formuli (při daném ohodnocení) -- indukcí podle její složitosti, podle tabulek pro logické spojky. 
\end{definiceN}


\begin{definiceN}{Interpretace jazyka predikátové logiky}
\emph{Interpretace jazyka} je definována množinovou strukturou $\c{M}$, která ke každému symbolu jazyka a množině proměnných přiřadí nějakou množinu individuí. $\c{M}$ obsahuje:
\begin{pitemize}
    \item neprázdnou množinu individuí $M$.
    \item zobrazení $f_M:M^n\to M$ pro každý $n$-ární funkční symbol $f$
    \item relaci $p_M\subset M^n$ pro každý $n$-ární predikát $p$
\end{pitemize}

Interpretace termů se uvažuje pro daný jazyk $L$ a jeho interpretaci $\c{M}$. \emph{Ohodnocení proměnných} je zobrazení $e:X\to M$ (kde $X$ je množina proměnných). \emph{Interpretace termu} $t$ při ohodnocení $e$ - $t[e]$ se definuje následovně:
\begin{pitemize}
    \item $t[e]=e(x)$ je-li $t$ proměnná $x$
    \item $t[e]=f_M(t_1[e],\dots,t_n[e])$ pro term tvaru $f(t_1,\dots,t_n)$.
\end{pitemize}
Ohodnocení závisí na zvoleném $\c{M}$, interpretace termů při daném ohodnocení pak jen na konečně mnoha hodnotách z něj. Pokud jsou $x_1,\dots,x_n$ všechny proměnné termu $t$ a $e,e'$ dvě ohodnocení tak, že $e(x_i)=e'(x_i) \forall i\in\{1,\dots,n\}$, pak $t[e]=t[e']$.

\emph{Pozměněné ohodnocení} $y$ pro $x = m\in M$ je definováno: $$e(x/m)(y)=
    \begin{cases}
	m \text{(pro $y\equiv x$)} \\
	e(y) \text{(jinak)}
    \end{cases}
    $$
\end{definiceN}

\begin{definiceN}{Substituce, instance, substituovatelnost}
\emph{Substituce} proměnné za podterm v termu ($t_{x_1,\dots,x_n}[t_1,\dots,t_n]$) je současné nahrazení všech výskytů proměnných $x_i$ termy $t_i$. Je to term. 

\emph{Instance} formule je současné nahrazení všech volných výskytů nějakých proměnných za termy. Je to taky formule, vyjadřuje speciálnější tvrzení -- ne vždy ale lze provést substituci bez změny významu formule. Term je \emph{substituovatelný} do formule $A$ za proměnnou $x$, pokud pro $\forall y$ vyskytující se v $t$ žádná podformule formule $A$ tvaru $(\exists y)B$ ani $(\forall y)B$ neobsahuje volný výskyt $x$.
\end{definiceN}

\begin{definiceN}{Uzávěr formule}
Jsou-li $x_1,\dots,x_n$ všechny proměnné s volným výskytem ve formuli $A$, potom $(\forall x_1)\dots(\forall x_n)A$ je \emph{uzávěr} formule $A$.
\end{definiceN}

\subsubsection*{Tautologie}

\begin{definiceN}{Tautologie ve výrokové logice}
Formule je \emph{tautologie}, jestliže je pravdivá při libovolném ohodnocení proměnných ($\models A$).
\end{definiceN}

\begin{definiceN}{Tautologický důsledek}
Teorie $U$ je \emph{tautologický důsledek} teorie $T$, jestliže každý model $T$ je také modelem $U$ ($T\models U$). Model nějaké teorie ve výrokové logice je takové ohodnocení proměnných, že každá formule z této teorie je pravdivá. K modelům se ještě vrátíme v následující sekci.
\end{definiceN}




\def\c#1{\mathcal{#1}}
\def\Nat{\mathbb{N}}


\subsection{Rozhodnutelnost, splnitelnost, pravdivost a dokazatelnost}

Z těchto témat se rozhodnutelnosti budeme věnovat až jako poslední, protože k vyslovení některých vět budeme potřebovat pojmy, definované v částech o splnitelnosti, pravdivosti a dokazatelnosti.

\subsubsection*{Splnitelnost}

\begin{definiceN}{Splnitelnost}
Množina formulí $T$ ve výrokové logice je \emph{splnitelná}, jestliže existuje ohodnocení $v$ takové, že každá formule $\forall A\in T$ je pravdivá při $v$. Potom se $v$ nazývá \emph{model teorie} $T$ ($v\models T$).
\end{definiceN}


\subsubsection*{Pravdivost}

\begin{definiceN}{Pravdivá formule výrokové logiky}
Formule výrokové logiky $A$ je \emph{pravdivá} při ohodnocení $v$, je-li $\overline{v}(A)=1$, jinak je \emph{nepravdivá}. Je-li formule $A$ pravdivá při ohodnocení $v$, pak říkáme, že $v$ \emph{je model} $A$ ($v\models A$). 
\end{definiceN}

\begin{definiceN}{Tarského definice pravdy}
Pro daný (redukovaný, tj. jen se \uv{základními} log. spojkami) jazyk predikátové logiky $L$, $\c{M}$ jeho interpretaci, ohodnocení $e$ a $A$ formuli tohoto jazyka platí:
\begin{penumerate}
    \item $A$ je \emph{splněna v ohodnocení} $e$ ($\c{M}\models A[e]$), když:
    \begin{pitemize}
	\item $A$ je atomická tvaru $p(t_1,\dots,t_n)$, kde $p$ není rovnost a $(t_1[e],\dots,t_n[e])\in p_M$.
	\item $A$ je atomická tvaru $t_1 = t_2$ a $t_1[e]=t_2[e]$
	\item $A$ je tvaru $\neg B$ a $\c{M}\not\models B[e]$
	\item $A$ je tvaru $B\rightarrow C$ a $\c{M}\not\models B[e]$ nebo $\c{M}\models C[e]$
	\item $A$ je tvaru $(\forall x)B$ a $\c{M}\models B[e(x/m)]$ pro každé $m\in M$
	\item $A$ je tvaru $(\exists x)B$ a $\c{M}\models B[e(x/m)]$ pro nějaké $m\in M$
    \end{pitemize}
    \item $A$ je \emph{pravdivá v interpretaci} $\c{M}$ ($\c{M}\models A$), jestliže je $A$ splněna v $M$ při
	každém ohodnocení proměnných (pro uzavřené formule stačí jedno ohodnocení, splnění je vždy stejné)
   \end{penumerate}
\end{definiceN}

\begin{definiceN}{Logicky pravdivá formule predikátové logiky}
Formule $A$ je \emph{validní (logicky pravdivá)} ($\models A$), když je platná při každé interpretaci daného jazyka.
\end{definiceN}

\subsubsection*{Dokazatelnost}

\begin{definiceN}{Axiomy výrokové logiky}
Pro redukovaný jazyk výrokové logiky (po snížení počtu log. spojek na základní ($\rightarrow, \neg$)) jsou \emph{axiomy výrokové logiky} (schémata axiomů) všechny formule následujících tvarů:
\begin{pitemize}
    \item $(A\rightarrow (B\rightarrow A))$ 
	(A1 - \uv{implikace sebe sama})
    \item $(A\rightarrow (B\rightarrow C))\rightarrow ((A\rightarrow B) \rightarrow (A\rightarrow C))$ 
	(A2 - \uv{roznásobení})
    \item $(\neg B\rightarrow \neg A)\rightarrow(A\rightarrow B)$ 
	(A3 - \uv{obrácená negace implikace})
\end{pitemize}
\end{definiceN}

\begin{definiceN}{Odvozovací pravidlo výrokové logiky}
Výroková logika má jediné odvozovací pravidlo -- \emph{modus ponens}:
$$\frac{A,A\rightarrow B}{B} $$
\end{definiceN}

\begin{definiceN}{Důkaz ve výrokové logice}
Důkaz $A$ je konečná posloupnost formulí $A_1,\dots A_n$, jestliže $A_n = A$ a pro každé $i=1,..n$ je $A_i$ buď axiom, nebo je odvozená z předchozích pravidlem modus ponens. Existuje-li důkaz formule $A$, pak je tato \emph{dokazatelná} ve výrokové logice (je větou výrokové logiky - $\vdash A$).
\end{definiceN}

\begin{definiceN}{Důkaz z předpokladů}
Důkaz formule $A$ z předpokladů je posloupnost formulí $A_1,\dots A_n$ taková, že $A_n = A$ a $\forall i\in\{1,..n\}$ je $A_i$ axiom, nebo prvek množiny předpokladů $T$, nebo je odvozena z přechozích pravidlem modus ponens. Existuje-li důkaz $A$ z $T$, pak $A$ \emph{je dokazatelná} z $T$ - $T\vdash A$.
\end{definiceN}

\begin{vetaN}{o dedukci ve výrokové logice}
Pro $T$ množinu formulí a formule $A,B$ platí: $$T\vdash A\rightarrow B \mbox{ právě když } T,A\vdash B$$
\end{vetaN}


\begin{definice}
Množina formulí $T$ je \emph{sporná}, pokud je z předpokladů $T$ dokazatelná libovolná formule, jinak je $T$ \emph{bezesporná}. $T$ je \emph{maximální bezesporná} množina, pokud je $T$ bezesporná a navíc jediná její bezesporná nadmnožina je $T$ samo. Množina všech formulí dokazatelných z $T$ se značí $\mathit{Con}(T)$.
\end{definice}

\begin{vetaN}{Lindenbaumova}
Každou bezespornou množinu formulí výrokové logiky $T$ lze rozšířit na maximální bezespornou $S$, $T\subset S$.
\end{vetaN}

\begin{vetaN}{o bezespornosti a splnitelnosti}
Množina formulí výrokové logiky je bezesporná, právě když je splnitelná.
\end{vetaN}

\begin{vetaN}{Věta o úplnosti výrokové logiky}
Je-li $T$ množina formulí a $A$ formule, pak platí:
\begin{penumerate}
    \item $T\vdash A$ právě když $T\models A$
    \item $\vdash A$ právě když $\models A$ ($A$ je větou výrokové logiky, právě když je tautologie)
\end{penumerate}
Tedy výroková logika je bezesporná a jsou v ní dokazatelné právě tautologie.
\end{vetaN}

\begin{vetaN}{o kompaktnosti}
Množina formulí výrokové logiky je splnitelná, právě když je splnitelná každá její konečná podmnožina.
\end{vetaN}


\begin{definiceN}{Formální systém predikátové logiky}
Pracujeme s redukovaným jazykem (jen s log. spojkami $\neg,\rightarrow$ a jen s kvantifikátorem $\forall$). \emph{Schémata Axiomů predikátové logiky} vzniknou z těch ve výrokové logice prostým dosazením libovolných formulí predikátové logiky za výrokové proměnné. \emph{Modus ponens} platí i v pred. logice. Další axiomy a pravidla:
\begin{pitemize}
    \item \emph{schéma(axiom) specifikace}: $(\forall x)A\rightarrow A_x[t]$
    \item \emph{schéma přeskoku}: $(\forall x)(A\rightarrow B)\rightarrow (A\rightarrow(\forall x)B)$, pokud 
	proměnná $x$ nemá volný výskyt v $A$.
    \item \emph{pravidlo generalizace}: $\frac{A}{(\forall x)A}$
\end{pitemize}
Toto je formální systém pred. logiky \emph{bez rovnosti}. S rovností přibývá symbol $=$ a další tři axiomy.
\end{definiceN}


\begin{poznamkaN}{Vlastnosti formulí predikátové logiky}
\begin{penumerate}
    \item Je-li $A'$ instance formule $A$, pak jestliže platí $\vdash A$, platí i $\vdash A'$.
	(\emph{Věta o instancích})
    \item Je-li $A'$ uzávěr formule $A$, pak $\vdash A$ platí právě když $\vdash A'$. 
	(\emph{Věta o uzávěru})
\end{penumerate}
\end{poznamkaN}

\begin{vetaN}{o dedukci v predikátové logice}
Nechť $T$ je množina formulí pred. logiky, $A$ je uzavřená formule a $B$ lib. formule, potom $T\vdash A\rightarrow B$ právě když $T,A\vdash B$.
\end{vetaN}

\begin{definiceN}{Teorie, model}
Pro nějaký jazyk $L$ prvního řádu je množina $T$ formulí tohoto jazyka \emph{teorie prvního řádu}. Formule z $T$ jsou \emph{speciální axiomy} teorie $T$. Pro interpretaci $\c{M}$ jazyka $L$ je $\c{M}$ \emph{model teorie} $T$ ($\c{M}\models T$), pokud jsou všechny speciální axiomy $T$ pravdivé v $\c{M}$. Formule $A$ je \emph{sémantickým důsledkem} $T$: $T\models A$, jestliže je pravdivá v každém modelu teorie $T$.
\end{definiceN}

\begin{vetaN}{o korektnosti}
Je-li $T$ teorie prvního řádu a $A$ formule, potom platí:
\begin{penumerate}
    \item Jestliže $T\vdash A$, potom $T\models A$.
    \item Speciálně jestliže $\vdash A$, potom $\models A$.
\end{penumerate}
\end{vetaN}

\begin{vetaN}{o úplnosti v predikátové logice}
Nechť $T$ je teorie s jazykem prvního řádu $L$. Je-li $A$ lib. formule jazyka $L$, pak platí:
\begin{penumerate}
    \item $T\vdash A$ právě když $T\models A$
    \item $T$ je bezesporná, právě když má model.
\end{penumerate}
\end{vetaN}

\begin{definiceN}{Úplná teorie}
Teorie $T$ s jazykem $L$ prvního řádu je \emph{úplná}, je-li bezesporná a pro libovolnou uzavřenou formuli $A$ je jedna z formulí $A,\neg A$ dokazatelná v $T$.
\end{definiceN}

\begin{vetaN}{o kompaktnosti}
Teorie $T$ s jazykem $L$ prvního řádu má model, právě když každý její konečný fragment $T'\subset T$ má model. Tj. pro libovolnou formuli $A$ jazyka $L$ platí: $T\models A$ právě když $T'\models A$ pro nějaký konečný fragment $T'\subset T$.
\end{vetaN}

\subsubsection*{Rozhodnutelnost}

\begin{definiceN}{Rekurzivní funkce a množina}
\emph{Rekurzivní funkce} jsou všechny funkce popsatelné jako $f:\Nat^k\to\Nat$, kde $k\geq 1$, tedy všechny \uv{algoritmicky vyčíslitelné} funkce. Množina přirozených čísel je \emph{rekurzivní množina (rozhodnutelná množina)}, pokud je rekurzivní její charakteristická funkce (to je funkce, která určí, které prvky do množiny patří).
\end{definiceN}

\begin{definiceN}{Spočetný jazyk, kód formule}
\emph{Spočetný jazyk} je jazyk, který má nejvýš spočetně mnoho speciálních symbolů. Pro spočetný jazyk, kde lze efektivně (rekurzivní funkcí) očíslovat jeho speciální symboly, lze každé jeho formuli $A$ přiřadit její \emph{kód formule} - přir. číslo $\#A$.
\end{definiceN}

\begin{vetaN}{Churchova o nerozhodnutelnosti predikátové logiky}
Pokud spočetný jazyk $L$ prvního řádu obsahuje alespoň jednu konstantu, alespoň jeden funkční symbol arity $k>0$ a pro každé přirozené číslo spočetně mnoho predikátových symbolů, potom množina $\{\#A|A \text{ je uzavřená formule a }L\models A\}$ není rozhodnutelná.
\end{vetaN}

\begin{vetaN}{o nerozhodnosti predikátové logiky}
Nechť $L$ je jazyk prvního řádu bez rovnosti a obsahuje alespoň 2 binární predikáty. Potom je predikátová logika (jako teorie) s jazykem $L$ nerozhodnutelná.
\end{vetaN}

\begin{definiceN}{Tři popisy aritmetiky}
Je dán jazyk $L=\{0,S,+,\cdot\,\leq\}$.
\begin{pitemize}
    \item \emph{Robinsonova aritmetika} - "$Q$" s jazykem L má 8 násl. axiomů:
    \begin{penumerate}
	\item $S(x)\neq 0$
	\item $S(x)=S(y)\rightarrow x=y$
	\item $x\neq 0\rightarrow (\exists y)(x=S(y))$
	\item $x+0=x$
	\item $x+S(y)=S(x+y)$
	\item $x\cdot 0=0$
	\item $x\cdot S(y)=(x\cdot y)+x$
	\item $x\leq y\leftrightarrow (\exists z)(z+x=y)$
    \end{penumerate}

    \textit{Poznámka: Někdy, pokud není potřeba definovat uspořádání, se poslední axiom spolu se symbolem \uv{\leq} vypouští.}

    \item \emph{Peanova aritmetika} - "$P$" má všechny axiomy Robinsonovy kromě třetího, navíc má 
	\emph{Schéma(axiomů) indukce} - pro formuli $A$ a proměnnou $x$ platí: $A_x[0]\rightarrow 
	\{(\forall x)(A\rightarrow A_x[S(x)])\rightarrow(\forall x)A\}$.
    \item \emph{Úplná aritmetika} má za axiomy všechny uzavřené formule pravdivé v $\Nat$, je-li $\Nat$
	standardní model aritmetiky - \uv{pravdivá aritmetika}. \emph{Teorie modelu $\Nat$} je množina
	$Th(\Nat)=\{A|A\text{ je uzavřená formule a } N\models A\}$.
\end{pitemize}
Platí: $Q\subseteq P\subseteq Th(\Nat)$. $Q$ má konečně mnoho axiomů, je tedy rekurzivně axiomatizovatelná. $P$ má spočetně mnoho axiomů, kódy axiomů schématu indukce tvoří rekurzivní množinu. $Th(\Nat)$ není rekurzivně axiomatizovatelná.
\end{definiceN}


\begin{definiceN}{Množina kódů vět teorie}
Pro $T$ teorii s jazykem aritmetiky definujeme \emph{množinu kódů vět teorie} $T$ jako $Thm(T)=\{\#A|A \text{ je uzavřená formule a } T\vdash A\}$.
\end{definiceN}

\begin{definiceN}{Rozhodnutelná teorie}
Teorie $T$ s jazykem aritmetiky je \emph{rozhodnutelná}, pokud je množina $Thm(T)$ rekurzivní. V opačném případě je $T$ \emph{nerozhodnutelná}.
\end{definiceN}


\begin{vetaN}{Churchova o nerozhodnutelnosti aritmetiky}
Každé bezesporné rozšíření Robinsonovy aritmetiky $Q$ je nerozhodnutelná teorie.
\end{vetaN}

\begin{vetaN}{Gödel-Rosserova o neúplnosti aritmetiky}
Žádné bezesporné a rekurzivně axiomatizovatelné rozšíření Robinsonovy aritmetiky $Q$ není úplná teorie.
\end{vetaN}



\subsection{Věty o kompaktnosti a úplnosti výrokové a predikátové logiky}

TODO: všechno

\subsection{Normální tvary výrokových formulí, prenexní tvary formulí predikátové logiky}

\begin{poznamkaN}{Vlastnosti log. spojek}
Platí:
\begin{penumerate}
    \item $A\wedge B\vdash A$; $A,B\vdash A\wedge B$
    \item $A\leftrightarrow B\vdash A\rightarrow B$; $A\rightarrow B, B\rightarrow A\vdash A\leftrightarrow B$
    \item $\wedge$ je idempotentní, komutativní a asociativní.
    \item $\vdash(A_1\rightarrow\dots(A_n\rightarrow B)\dots)
	\leftrightarrow((A_1\wedge\dots\wedge A_n)\rightarrow B)$
    \item DeMorganovy zákony: $\vdash\neg(A\wedge B)\leftrightarrow(\neg A\vee\neg B)$;
	$\vdash\neg(A\vee B)\leftrightarrow(\neg A\wedge\neg B)$
    \item $\vee$ je monotonní ($\vdash A\rightarrow A\vee B$), idempotentní, komutativní a asociativní.
    \item $\vee$ a $\wedge$ jsou navzájem distributivní.
\end{penumerate}
\end{poznamkaN}

\begin{vetaN}{o ekvivalenci ve výrokové logice}
Jestliže jsou podformule $A_1\dots A_n$ formule $A$ ekvivalentní s $A'_1\dots A'_n$ ($\vdash A'_i \leftrightarrow A_i$) a $A'$ vytvořím nahrazením $A'_i$ místo $A_i$, je i $A$ ekvivalentní s $A'$. (Důkaz indukcí podle složitosti formule, rozborem případů $A_i$ tvaru $\neg B$, $B\rightarrow C$)
\end{vetaN}

\begin{lemmaN}{o důkazu rozborem případů}
Je-li $T$ množina formulí a $A,B,C$ formule, pak $T,(A\vee B)\vdash C$ platí právě když $T,A\vdash C$ a $T,B\vdash C$.
\end{lemmaN}

\begin{definiceN}{Normální tvary}
Výrokovou proměnnou nebo její negaci nazveme \emph{literál}. \emph{Klauzule} budiž disjunkce několika literálů. \emph{Formule v normálním konjunktivním tvaru (CNF)} je konjunkce klauzulí. \emph{Formule v disjunktivním tvaru (DNF)} je disjunkce konjunkcí literálů.
\end{definiceN}

\begin{vetaN}{o normálních tvarech}
Pro každou formuli $A$ lze sestrojit formule $A_k,A_d$ v konjunktivním, resp. disjunktivním tvaru tak, že $\vdash A\leftrightarrow A_d$, $\vdash A\leftrightarrow A_k$. (Důkaz z DeMorganových formulí a distributivity, indukcí podle složitosti formule)
\end{vetaN}

\subsubsection*{Prenexní tvary formulí predikátové logiky}

\begin{vetaN}{o ekvivalenci v predikátové logice}
Nechť formule $A'$ vznikne z $A$ nahrazením některých výskytů podformulí $B_1,\dots,B_n$ po řadě formulemi $B'_1,\dots,B'_n$. Je-li $\vdash B_1\leftrightarrow B'_1,\dots,\vdash B_n\leftrightarrow B'_n$, potom platí i $\vdash A\leftrightarrow A'$.
\end{vetaN}

\begin{definiceN}{Prenexní tvar}
Formule predikátové logiky $A$ je v \emph{prenexním tvaru}, je-li $$A\equiv (Q_1 x_1)(Q_2 x_2)\dots(Q_n x_n)B,$$ kde $n\geq 0$ a $\forall i\in\{1,\dots,n\}$ je $Q_i\equiv \forall$ nebo $\exists$, $B$ je otevřená formule a kvantifikované proměnné jsou navzájem různé. $B$ je \emph{otevřené jádro} $A$, část s kvantifikátory je \emph{prefix} $A$.
\end{definiceN}

\begin{definiceN}{Varianta formule predikátové logiky}
Formule $A'$ je \emph{varianta} $A$, jestliže vznikla z $A$ postupným nahrazením podformulí $(Q x)B$ (kde $Q$ je $\forall$ nebo $\exists$) formulemi $(Q y)B_x[y]$ a $y$ není volná v $B$. Podle \emph{věty o variantách} je varianta s původní formulí ekvivalentní.
\end{definiceN}

\begin{lemmaN}{o prenexních operacích}
Pro převod formulí do prenexního tvaru se používají tyto operace (výsledná formule je s původní ekvivalentní). Pro podformule $B$, $C$, kvantifikátor $Q$ a proměnnou $x$:
\begin{penumerate}
    \item podformuli lze nahradit nějakou její variantou
    \item $\vdash \neg(Q x)B\leftrightarrow(\overline{Q} x)\neg B$
    \item $\vdash (B\rightarrow (Q x)C)\leftrightarrow(Q x)(B\rightarrow C)$, pokud $x$ není volná v $B$
    \item $\vdash ((Q x)B\rightarrow C)\leftrightarrow(\overline{Q} x)(B\rightarrow C)$, pokud $x$ není 
	volná v $C$
    \item $\vdash ((Q x)B\wedge C)\leftrightarrow (Q x)(B\wedge C)$, pokud $x$ není volná v $C$
    \item $\vdash ((Q x)B\vee C)\leftrightarrow (Q x)(B\vee C)$, pokud $x$ není volná v $C$
\end{penumerate}
\end{lemmaN}

\begin{vetaN}{o prenexních tvarech}
Ke každé formuli $A$ predikátové logiky lze sestrojit ekvivalentní formuli $A'$, která je v prenexním tvaru. (Důkaz: indukcí podle složitosti formule a z prenexních operací, někdy je nutné přejmenovat volné proměnné)
\end{vetaN}


\end{document}
