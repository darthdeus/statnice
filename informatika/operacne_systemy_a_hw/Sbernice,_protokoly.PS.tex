\subsection{Sběrnice, protokoly}

\begin{pitemize}
	\item \textbf{Struktura sběrnice}: datové linky, adresové linky, řídící linky
	\item \textbf{Synchronní přenos} (vznik události je dán hodinovým signálem) vs. \textbf{asynchronní přenos} (vznik události je určen předcházející událostí~-- napr. signalizáciou začiatku dát) 
	\item \textbf{Parametry sběrnice}: 
	\begin{pitemize}
	  \item \emph{datová šířka}~-- počet přenášených bitů v jednom okamžiku,
	  \item \emph{kapacita}~-- počet bitů přenesených za čas,
	  \item \emph{rychlost}~-- kapacita sběrnice normovaná k jednotce informace. 
	\end{pitemize}  
	\item \textbf{Řízení požadavků}: 
	\begin{pitemize}
	  \item \emph{centrální}~-- náhodné, dle pořadí vzniku požadavků, prioritní,
	  \item \emph{distribuované}~-- kolizní (CSMA/CD), token bus, prioritní linka (daisy chain).
	\end{pitemize} 
	\item \textbf{Přenos dat po sběrnici} může probíhat buď za účasti procesoru (zdroj $\rightarrow$ CPU $\rightarrow$ cíl), nebo bez. Bez procesoru to může být např.:
	\begin{pitemize}
		\item dávkový režim~-- domluva mezi CPU a řadičem na době obsazení sběrnice (napr. pomocou zdvihnutia \uv{lock flagu} na zbernici)
		\item kradení cyklů~-- řadič na dobu přenosu \uv{uspí} procesor (nelze uspat na dlouho, je to technicky náročnější)
		\item transparentní režim~-- řadič rozezná, kdy procesor nepoužívá sběrnici, obvykle nelze větší přenosy najednou
		\item DMA (Direct Memory Access)~-- speciální jednotka pro provádění přenosů dat (mezi zařízeními a pamětí)
	\end{pitemize}
	Jednou z technik, používaných k přenosu dat po sběrnici řadiči DMA, je \emph{scatter-gather}. Znamená to, že v rámci jednoho přenosu se zpracovává víc ne nutně souvislých bloků dat. 
	\begin{pitemize}
	    \item \emph{scatter}~-- DMA řadič v rámci 1 přenosu uloží z 1 místa data na několik různých míst (např hlavičky TCP/IP - jinak zbytečné kopírování)  
	    \item \emph{gather}~-- např. při stránkování paměti - načítání stránek, které fyzicky na disku nemusí být u sebe, složení na 1 místo do paměti.
	\end{pitemize}
\end{pitemize}

Příklady sběrnic:
\begin{pitemize}
	\item ISA, EISA
	\item ATA, ATAPI~-- UltraDMA, Serial-ATA (SATA)
	\item SCSI (Small Computer System Interface)
	\item PCI, PCI-X, PCI Express
	\item AGP (Advanced Graphics Port)
	\item USB (Universal Serial Bus)
	\item FireWire (IEEE 1394)
	\item RS485
	\item $I^{2}C$
\end{pitemize}

Příbuzné sběrnic:
\begin{pitemize}
	\item IrDA
	\item Bluetooth
	\item Wi-Fi, WiMAX 
\end{pitemize}
