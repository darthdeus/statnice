\subsection{Vztah OS a HW, obsluha přerušení}

\begin{obecne}{Zjištění změny stavu I/O zařízení:}
\begin{pitemize}
	\item \emph{asynchronní přerušení}~-- zašle zařízení
	\item \emph{polling}~-- peridická kontrola stavu zařízení
\end{pitemize}
\end{obecne}

\begin{obecne}{Druhy přerušení:}
\begin{pitemize}
	\item \emph{synchronní}~-- záměrně (instrukce TRAP~-- vstup do OS), výjimky (nesprávné chování procesu)~-- zpracuje se okamžitě
	\item \emph{asynchronní}~-- vnější událost (např. příchod dat)~-- zpracuje se po dokončení aktuální instrukce 
\end{pitemize}
\end{obecne}

\begin{obecne}{Obsluha přerušení:}
\begin{pitemize}
	\item OS se ujme řízení
	\item uloží se stav CPU (obsah registrů, čítač, ...)
	\item analyzuje se přerušení, vyvolá se příslušná obsluha (pokud není přerušení blokováno)
	\item obslouží se přerušení (např. se zavolá obslužná procedura)
	\item obnoví se stav CPU a aplikace pokračuje, popř. může dojít k přeplánování 
\end{pitemize}
\end{obecne}

\begin{obecne}{I/O software (vrstvy):}
\begin{pitemize}
	\item uživatelský I/O software
	\item I/O nezávislý subsystém
	\item ovladače zařízení
	\item obsluha přerušení 
\end{pitemize}
\end{obecne}

\begin{obecne}{Cíle I/O software:}
\begin{pitemize}
	\item nezávislost zařízení~-- programy nemusí vědět, s jakým přesně pracují
	\item jednotné pojmenování (/dev)
	\item připojení (mount)~-- vyměnitelná zařízení
	\item obsluha chyb
\end{pitemize}
\end{obecne}
