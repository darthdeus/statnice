\subsection{Činnost systému při spouštění a ukončování, konfigurace}

TODO: tohle je jenom copy \& paste z Wiki

\subsubsection*{Spouštění systému}

\begin{pitemize}
  \item start zavaděče (LILO, GRUB)
  \item nahrání kernelu
  \item spuštění kernelu, detekce HW, spuštění ovladačů
  \item mount root readonly
  \item spuštění procesu init
  \item kontrola disků
  \item re-mount root read-write
  \item start-up skript (/etc/init.d)
  \item běžící systém, spuštění konzolí (getty) 
  \item při spouštění možnost aktivovat single-user režim 
  \item úrovně běhu (SystemV, Linux):
  \begin{pitemize}
    \item 0 - systém zastaven
    \item 1 - single-user
    \item 2 - multi-user, bez sítě a bez NFS
    \item 3 - multi-user
    \item 5 - multi-user + X11
    \item 6 - reboot
    \item konfigurace v /etc/inittab 
    \end{pitemize}
\end{pitemize}

\subsubsection*{Vypnutí systému}

\begin{pitemize}
    \item ukončení programů, zastavení služeb, signál TERM, po chvíli KILL
    \item vypráznění diskových cache, uložení dat, odpojení disků
    \item vypnutí napájení
    \item shutdown - možno nastavit čas vypnutí a zaslat oznámení všem přihlášeným 
\end{pitemize}

\subsubsection*{Konfigurace}

\begin{pitemize}
    \item na Unixových systémech se konfigurace ve většině případů provádí editací textových souborů
    \item většina konfigurace schována v adresáři /etc
\end{pitemize}
