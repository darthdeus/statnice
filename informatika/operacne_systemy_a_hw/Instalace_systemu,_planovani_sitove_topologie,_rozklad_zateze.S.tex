\subsection{Instalace systému, plánování síťové topologie, rozklad zátěže}

TODO: tohle je jenom copy\&paste z Wiki

\subsubsection*{Instalace systému}

\begin{pitemize}
    \item existuje mnoho variant Unixu a množství distribucí
    \item distribuce = jádro + balík dalších programů, nástrojů, rozšíření, ...
    \item FreeBSD, OpenBSD, Solaris, Linux (Debian, Fedora, Gentoo), ...
    \item software rozdělen do balíčků - volba balíčků k instalaci
    \item balíčkovací a instalační nástroje - RPM, apt, yum, emerge, ...
    \item rozdělení disku - /boot, /, /var/log apod.
    \item volba filesystému
    \item konfigurace sítě
    \item nastavení hesla roota
    \item po instalaci - kontrola běžících procesů 
\end{pitemize}

\subsubsection*{Plánování síťové topologie}

\begin{pitemize}
    \item volba technologií - drátové, bezdrátové
    \item volba protokolů - dnes zřejmě TCP/IP
    \item rozvržení lokální sítě, IP adresy (lokální, veřejné), DHCP, klientské stanice
    \item servery - poštovní, souborové, DNS, zálohovací, webové, aplikační, proxy, ...
    \item routery, směrování, propojení do Internetu
    \item sdílení informací o uživatelských účtech (NIS, YP)
    \item sdílení dat (NFS)
    \item redundance klíčových serverů a routerů (CARP, master/slave)
    \item zabezpečení (firewall), šifrování, DMZ (demilitarizovaná zóna), VPN 
\end{pitemize}

\subsubsection*{Rozklad zátěže}

\begin{pitemize}
    \item protokol CARP, nástroje CARP a UCARP
    \item překlad IP adres Round-Robin (NAT)
\end{pitemize}
