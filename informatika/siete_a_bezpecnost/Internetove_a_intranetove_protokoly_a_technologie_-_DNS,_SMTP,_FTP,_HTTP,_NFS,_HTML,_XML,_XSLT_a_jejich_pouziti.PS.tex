\subsection{Internetové a intranetové protokoly a technologie -- DNS, SMTP, FTP, HTTP, NFS, HTML, XML, XSLT a jejich použití}

\subsubsection*{DNS (Domain Name System)}
\begin{pitemize}
	\item Řešení které umožňuje používat symbolická jména místo číselných adres realizované pomocí DNS serverů, distribuované řešení -- výpadek i více serverů nevyřadí službu z provozu
	\item hierarchický (stromový) prostor jmen. Kořenem je tzv. kořenová doména –- tečka, pod ní se v hierarchii nachází domény nejvyšší úrovně (com, edu, cz, uk, \dots), strom je možné rozdělit do zón a její správu svěřit někomu dalšímu -- právě možnost delegování pravomocí a distribuovaná správa tvoří klíč. vlastnosti DNS a stojí za jeho úspěchem
	\item DNS servery  
	\begin{pitemize}
		\item Primární -- zde data vznikají, zde se musí také provádět změny, každá doména obsahuje právě jeden
		\item Sekundární -– automatická kopie primárního, průběžně si aktualizuje data, slouží jednak jako záloha pro případ výpadku prim. serveru a také pro rozkládání zátěže, každá doména musí mít alespoň jeden sekundární server
		\item Pomocný (caching only) -– slouží jako vyrovnávací paměť pro snížení zátěže celého systému, uchovává si odpovědi a poskytuje je při opakování dotazů dokud jim nevyprší životnost
	\end{pitemize}
	\item Odpovědi z primárního a ze sekundárních serverů jsou autoritatvní – platné. Z pomocného serveru je odpověd neautoritativní -- klient může požádat o autoritativní odpověď.

	\item soubor hosts
	\item doména, zóna, delegace, TLD, ccTLD, gTLD
	\item syntaxe jmen, FQDN (plně kvalifikované...)
	\item IDN
	\item iterativní dotaz, rekurzivní dotaz, cachování, resolvery, TTL, autoritativní odpověď
	\item Resource Records (RR) -- jednotka informace v DNS
		\begin{pitemize}
			\item formát: \texttt{ [name type class TLL rdlength rdata] }
			\item class -- dnes vždy IN (internet)
			\item TTL -- time to live (doba platnosti záznamu)
			\item types -- A (IPv4 adresa), NS (hostname nameserveru), MX (mailserver), PTR (domain name pointer -- reverse DNS), AAAA (IPv6 adresa), SPF, TXT, SRV, ... 
			\item rdlength -- délka dat, rdata -- vlastní data
		\end{pitemize}
	\item DNS protokol -- TCP/UDP, truncation 
\end{pitemize}

\subsubsection*{SMTP (Simple Mail Transfer Protocol)}
\begin{pitemize}
	\item Internetový protokol určený pro přenos zpráv elektronickě pošty mezi stanicemi. Protokol zajišťuje doručení pošty pomocí přímého spojení mezi odesílatelem a adresátem; zpráva je doručena do tzv. poštovní schránky adresáta, ke které potom může uživatel kdykoli (offline) přistupovat a vybírat zpráva pomocí protokolů POP3 popř. IMAP.
	\item Funguje nad protokolem TCP, používá port 25
	\item Pro netextové přenosy je využíván standart MIME (řeší problém národních abeced, formátování, příloh \dots) -- rozšíření formátu zpráv -- Quoted-Printable, Base64, mime-type
	\item SMTP -- protokol pro přenos zpráv
	\item RFC822 -- formát zpráv -- 7-bitová data
	\item POP3, IMAP -- stahování zpráv ze schránky
	\item struktura zprávy -- hlavička + tělo + přílohy
	\item e-mailové adresy, MX záznamy, priority 
\end{pitemize}

\subsubsection*{FTP (File Transfer Protocol)}
\begin{pitemize}
	\item Jeden ze sady protokolů TCP/IP, netransparetní řešení – uživatel si uvědomuje že se soubor nachází na vzdáleném počítači, typicky se vzdálený soubor celý přenese na místní počítač a zde se s ním pracuje ; v rámci TCP/IP  existuje zjednodušená verze -- TFTP
	\item Dva režimy – textový a binární ; vychází z modelu klient/server, klient je typicky aplikační program, server je obvykle systémový proces (démon,\dots)
	\item Jednotný formát pro potřeby přenosu dat, veškeré konverze provadí koncové uzly
	\item Používají se 2 různá spojení – řídící (přenos příkazů – FTP má vlastní řídící jazyk) a datové(přenos souborů), řídící navazuje klient, datové navazuje server (okrem passive módu)
	\item příkazy -- řízení přístupu, nastavení parametrů, výkonné příkazy
	\item TFTP 
\end{pitemize}

\subsubsection*{NFS (Network File System)}
\begin{pitemize}
	\item Jeden ze sady protokolů TCP/IP, slouží pro transparetní sdílení souborů (uživatel/aplikace si neuvědomuje že se soubor nachází na vzdáleném počítači), typicky se vzdálený soubor chová \uv{tváří} jako místní soubor a také se s ním tak pracuje
	\item Je použitelný na různých platformách
	\item Bezestavový protokol (v4 už je ale stavový), díky tomu je velmi robustní – to je důvod jeho úspěšnosti
	\item Využívá protokoly RPC (vzdálené volání procedur) a XDR (definuje jednotný způsob reprezentace přenášených dat nezávislý na konkrétní architektuře přijemce a odesílatele)
	\item mount server
\end{pitemize}

\subsubsection*{HTTP (Hyper-Text Transfer Protocol)}
\begin{pitemize}
	\item Protokol původně určený pro výměnu hypertextových dokumentů ve formátu HTML, v současné době užíván i pro přenos dalších informací – pomocí rozšíření MIME umí přenášet jakýkoli soubor (podobně jako SMTP)
	\item Existuje bezpečnejší verze HTTPS, umožňuje přenášená data šifrovat
	\item Funguje systémem dotaz-odpověď, při zaslání více dotazů není možné rozpoznat zda spolu souvisí -- HTTP je bezestavový protokol (nepříjemná vlastnost pro implementaci složitejších procesů přes HTTP) – proto byl rozšířen o HTTP cookies (umožňují uchovávat info o stavu na počítači uživatele)
	\item verze 0.9 -- bez hlaviček, minimální možnosti
	\item verze 1.0 -- rozšiřující hlavičky, podpora MIME
	\item verze 1.1 -- virtuální servery, jedno spojení pro více přenosů (keep-alive), komprimace dat
	\item GET, HEAD, POST
	\item cookies
	\item cachování 
\end{pitemize}

\subsubsection*{HTML (Hyper-Text Markup Language)}
\begin{pitemize}
	\item značkovací jazyk pro hypertext, definuje obsah, nikoliv vzhled
	\item CSS
	\item statické a dynamické HTML dokumenty -- CGI, ISAPI, NSAPI, ASP, PHP
	\item skripty, Java, ActiveX 
\end{pitemize}

\subsubsection*{XML (eXtensible Markup Language)}
\begin{pitemize}
	\item Značkovací jazyk vyvinut a standardizován konsorciem W3C, umožnuje snadné vytváření konkrétních značkovacíh jazyků pro různé účely
	\item Určen především pro výměnu dat mezi aplikacemi a pro publikování dokumentů, umožňuje popsat strukturu dokumentu z hlediska věcného obsahu, nezabývá se sám o sobě vzhledem dokumentu, vzhled dokumentu se definuje připojeným stylem
	\item Pomocí různých stylů je možné provést transformaci do jiného typu dokumentu, nebo jiné XML struktury (výsledkem může být např. HTML, PostScript,\dots)
\end{pitemize}

\subsubsection*{XSLT (eXtensible Style Sheet Language Transformations)}
\begin{pitemize}
	\item Transformace sloužící pro převod dat ve formátu XML do lib. jiného požadovaného formátu (nejčastěji HTML,jiného XML, ale také PDF,či RTF, \dots), struktura výstupu není definována přímo standardem – je závislá na procesoru XSLT (program který provede transformaci)
	\item K provedení transformace jsou třeba 2 soubory: 
	\begin{pitemize}
		\item Soubor, který obsahuje zdrojová data, která budou transformována.\\Struktura tohoto souboru vyjma obecných vlastností XML není blíže specifikována.
		\item Soubor, obsahující vzorec pro transformaci, napsaný v jazyce XSL
	\end{pitemize}
\end{pitemize}
