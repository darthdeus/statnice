\subsection{Síťové, systémové a adresářové služby, vzdálený přístup}

TODO: tohle je jenom copy\&paste z Wiki

\subsubsection*{Síťové služby}

\begin{pitemize}
    \item DNS dává jména IP a MAC adresám
    \item DHCP přiděluje IP adresy
    \item autentizační služby
    \item e-mail
    \item tisk
    \item NFS (Network File System) umožňuje sdílení souborů a zdrojů 
\end{pitemize}

\subsubsection*{Systémové služby}

\subsubsection*{Adresářové služby}
Speciální aplikace pro ukládání záznamů. Typicky se vyhledává hodně a data se
mění málo a jednoduše (bez transakcí). Používá se pro uložení údajů o lidech a
zdrojích (tiskárny).

Pro přístup se používá standard LDAP, který byl navržen jako odlehčená verze protokolu X.500 ze světa ISO/OSI.

\begin{obecne}{Implementace adresářových služeb}
  \begin{pitemize}
    \item NIS od Sun-u (příkazy začínají na yp kvůli starému názvu Yellow Pages)
    \item Active Directory od MS
    \item OpenLDAP, Kerberos (open-source) 
  \end{pitemize}
\end{obecne}

\subsubsection*{Vzdálený přístup}

\begin{pitemize}
    \item SSH - Secure SHell 
    \item Telnet
\end{pitemize}
